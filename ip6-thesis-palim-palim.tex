\documentclass[12pt,a4paper]{report}

% -- Imports biblatex and defines bib file --
\usepackage[backend=bibtex,style=numeric,language=german,sorting=none]{biblatex}
\addbibresource{references.bib}
% http://tug.ctan.org/info/biblatex-cheatsheet/biblatex-cheatsheet.pdf

% -- Language --
\usepackage[utf8]{inputenc}
\usepackage[german]{babel}

% -- Images --
\usepackage{graphicx}
\graphicspath{ {./images/} }

% -- Continuous figure numbering --
\usepackage{chngcntr}
\counterwithout{figure}{chapter}

% -- Code blocks --
\usepackage{listings}
\usepackage{color}
\definecolor{lightgray}{rgb}{.9,.9,.9}
\definecolor{darkgray}{rgb}{.4,.4,.4}
\definecolor{purple}{rgb}{0.65, 0.12, 0.82}

% -- Code blocks Stlye --
\renewcommand\lstlistingname{Codefragment}
\renewcommand\lstlistlistingname{Codefragmente}

\lstdefinelanguage{JavaScript}{
  keywords={typeof, new, true, false, catch, function, return, null, catch, switch, var, if, in, while, do, else, case, break},
  keywordstyle=\color{blue}\bfseries,
  ndkeywords={class, export, boolean, throw, implements, import, this},
  ndkeywordstyle=\color{darkgray}\bfseries,
  identifierstyle=\color{black},
  sensitive=false,
  comment=[l]{//},
  morecomment=[s]{/*}{*/},
  commentstyle=\color{purple}\ttfamily,
  stringstyle=\color{red}\ttfamily,
  morestring=[b]',
  morestring=[b]"
}
\lstset{
   language=JavaScript,
   backgroundcolor=\color{lightgray},
   extendedchars=true,
   basicstyle=\footnotesize\ttfamily,
   showstringspaces=false,
   showspaces=false,
   numbers=left,
   numberstyle=\footnotesize,
   numbersep=9pt,
   tabsize=2,
   breaklines=true,
   showtabs=false,
   captionpos=b
}

\newcommand{\paragraphwithnewline}[1]{\paragraph{#1}\mbox{}\\}

\begin{document}


\begin{titlepage}
\paragraph{Titelblatt}
\end{titlepage}

\chapter*{Projektinformationen}
Titel: Palim-Palim\\
Projektnummer: 21FS\_I4DS08

\paragraphwithnewline{Projekt-Team}
Daniel Obrist, 8iCbb\\
daniel.obrist@students.fhnw.ch\\\\
Severin Peyer, 8iCbb\\
severin.peyer@students.fhnw.ch

\paragraphwithnewline{Auftraggeber und Betreuung FHNW}
Marco Soldati\\
Fachhochschule Nordwestschweiz FHNW\\
Hochschule für Technik\\
Bahnhofstrasse 6\\
CH-5210 Windisch\\
+41 56 202 77 31\\
marco.soldati@fhnw.ch\\\\
Tabea Iseli\\
Fachhochschule Nordwestschweiz FHNW\\
Hochschule für Technik\\
Bahnhofstrasse 6\\
CH-5210 Windisch\\
+41 56 202 86 53\\
tabea.iseli@fhnw.ch\\

\paragraphwithnewline{Zeitbudget}
Das Projekt wird im Rahmen des Frühlingssemesters 2021 durchgeführt. Nominell sind für das Projekt 360~Arbeitsstunden pro Teammitglied veranschlagt. 

\paragraphwithnewline{Wichtigste Daten}
Beginn:	22. Februar 2021\\ 
Ende:	20. August 2021 \\

\begin{abstract}	
	\begin{itemize}
 		\item Ergebnisse aus den Spieletests, der Entwicklung und der Literaturrechereche zusammengefasst erläutern.
 		\item Erst am Schluss schreiben!
 		\item Wichtig für den ersten Eindruck
	\end{itemize}
\end{abstract}


\tableofcontents

\break

\chapter{Einleitung}
\paragraph{Teil 1 / Was wurde erreicht?}
\emph{\begin{itemize}
\color{blue}
 \item Beschreibung des Videospiels Palim-Palim (inkl. Screenshot)
 \item Aufstellung der Forschungsfragen und den Erkenntnissen
\end{itemize}}

\paragraph{Teil 2 / Warum wurde es gemacht?}
\emph{\begin{itemize}
\color{blue}
 \item Ausgangslage inkl. Forschungsstand
 \item Relevanz der Problemstellung
 \item Was ist das Umfeld?
\end{itemize}}

\paragraph{Teil 3 / Wie wurde es gemacht?}
\emph{\begin{itemize}
\color{blue}
 \item Grobe Beschreibung der angewendeten Methodik
 \begin{itemize}
 	\item Spielentwicklung (Architektur, Technologien)
 	\item Spieletests (Methoden)
 \end{itemize}
\end{itemize}}
 
\paragraph{Teil 4}
\emph{\begin{itemize}
\color{blue}
	\item Aufbau des Dokuments und Überleitung in den theoretischen Teil
\end{itemize}}

Palim-Palim ist ein interaktives Multiplayer-Videospiel für Kinder und betagte Menschen. Zwei Spielende können damit Kinder-Kaufladen in einer virtuellen Umgebung spielen. Es verfügt über einen Video-Chat als inhärente Game-Mechanik, um die Kommunikation zwischen den Spielerinnen und Spielern zu fördern.\\\\
Mit dem Spiel wird untersucht, welchen Einfluss ein Video-Chat in Videospielen auf die User Experience und die Kommunikation zwischen den Spielenden hat. Im Speziellen werden die folgenden Fragestellungen untersucht:
\begin{itemize}
	\item Welche Wirkung hat ein integrierter Video-Chat auf das Spielerlebnis von betagten Menschen?
	\item Welche Wirkung hat ein integrierter Video-Chat auf das Spielerlebnis von Kindern? 
	\item Welche Auswirkungen hat die Art und Weise, wie der Video-Chat in das Spiel integriert ist, auf das Spielerlebnis? 
	\item Welche Wirkung hat ein integrierter Video-Chat auf die Kommunikation zwischen den Spielenden?
	\item Wird ein integrierter Video-Chat aktiv als Kommunikationsmittel zur Bewältigung von Spielaufgaben genutzt?
	\item Welche Auswirkungen hat die Art und Weise, wie der Video-Chat in das Spiel integriert ist, auf die Förderung der Kommunikation?
\end{itemize}

\paragraph{\emph{\color{blue} Erkenntnisse}}

Video-Chats werden heutzutage schon von vielen Familien benutzt, um mit Ihren Verwandten zu kommunizieren. Auch in der Kommunikation mit den Grosseltern wird dabei immer mehr auf Video-Telefonie gesetzt. Dies gilt besonders für Zeiten, in denen Besuche im Alters- oder Pflegeheim auf Grund kursierenden Viren wie Corona schwierig oder unmöglich werden. Betagte Menschen schätzen und nutzen diese moderne Art der Kommunikation mit der Familie auch immer mehr – besonders weil es öfters auf ältere Benutzergruppen angepasste Angebote gibt \cite{glaab_silver_2015}.\\\\
			Allerdings hat die Video-Telefonie immer noch Grenzen, welche die Interaktionen unnatürlich und teilweise entfremdend wirken lassen. Vor allem für Kinder ist es schwierig, Gesprächsthemen und Kommunikationswege zu finden, die sich so lustig und verbindend anfühlen, wie die Zeit mit der Grossmutter oder dem Grossvater im echten Leben. Oft sind Kinder vom Gespräch schnell gelangweilt – sie würden lieber etwas spielen \cite{tulloch_7_2020}. Spielen kann ein Mittel sein, um Kinder besser einzubeziehen und die Interaktion mit ihnen zu unterstützen, wie bisherige Studien zu dem Thema Video-Calls mit Eltern und Kindern zeigen \cite{follmer_video_2010}.\\\\
			Ergänzend zum Video-Telefonie-Aspekt gibt es bereits viel Literatur bezüglich generationenübergreifender Computerspiele \cite{chua_lets_2013}, \cite{de_la_hera_benefits_2017}, \cite{soldati_create_2020}. Erkenntnisse aus einer Studie von Derboven et al \cite{derboven_designing_2012} deuten ausserdem drauf hin, dass in einem Multiplayer-Videospiel die zusätzliche Kommunikationsfunktionalität durch einen Video-Chat oft sowohl von älteren als auch von jüngeren Personen begrüsst wird. Allerdings bietet die Forschung bisher keine detaillierte Studie mit Kindern und betagten Personen in diesem Zusammenhang.\\\\
			Am Institut für Data Science (I4DS) der Fachhochschule Nordwestschweiz wird im Rahmen des Projekts Myosotis schon seit einigen Jahren daran gearbeitet, Video-Spiele zu entwickeln, welche in unterhaltsamer Weise die soziale Interaktion zwischen betagten Menschen und ihren Angehörigen unterstützen \cite{soldati_fhnw_2015}. Dabei wurden schon etliche Spiele umgesetzt und getestet \cite{soldati_create_2020}. Mit einer Integration eines Video-Chats in ein Spiel hat sich jedoch bisher noch kein Team explizit auseinandergesetzt. Palim-Palim schliesst diese Lücke und zeigt wertvolle Erkenntnisse über die Kombination von Video-Chats und Video-Spielen mit Kindern und betagten Personen.\\\\
			Um die formulierten Fragen zu beantworten, wurden mehrere Varianten eines Videospiels implementiert, in welchen ein Video-Stream zu einem Teil des Spiels wird. Anschliessend wurden mit allen Varianten Spieltests durchgeführt, um herauszufinden, ob die Integration eines Video-Chats in einem Video-Spiel einen positiven oder negativen Einfluss auf die User Experience sowie die Kommunikation zwischen den Spielenden hat. Die Arbeit fokussiert sich speziell auch auf die Art und Weise, wie ein Video-Stream in ein Spiel integriert werden kann.

\paragraph{\emph{\color{blue} Grobe Systemarchitektur, verwendete Methoden und Konzepte}}

 

% -- Theoretischer Teil --
\chapter{Umfeldanalyse und Zielgruppe}
\emph{
	\begin{itemize}
		\color{blue}
		\item Beschreibung des Umfelds / Anwendungsdomäne (betagte Personen und Kinder)
		\item Persona?
	\end{itemize}
}
Die verwendeten Begriffe Kind und betagte Person werden dabei im Rahmen des Projekts wie folgt definiert:\\
\subparagraph{Kind} Person zwischen fünf und acht Jahren. 
\subparagraph{Betagte Person} Person ab einem Alter von 65 Jahren ohne grössere mentale Beeinträchtigung. 
Diese beiden Personengruppen stellen zugleich die Zielgruppen für die durchzuführenden Spieltests dar.
\chapter{Intergenerationelles Spielen}
\emph{
	\begin{itemize}
		\color{blue}
		\item Forschungsstand (Welche Methoden/Ansätze werden angewendet?)
		\item Bisherige Erkenntnisse zu intergenerationellem Spielen
	\end{itemize}
}
\chapter{Videochats in Videospielen}
Videochats sind ein wichtiges Kommunikationsmittel um zwei oder mehr Menschen über weite Distanzen miteinander zu verbinden. Auch betagte Menschen schätzen und nutzen diese moderne Art der Kommunikation mit der Familie auch immer mehr – besonders weil es öfters auf ältere Benutzergruppen angepasste Angebote gibt \cite{glaab_silver_2015}.\\\\
Während Videochats im geschäftlichen Umfeld bereits recht gut erforscht sind, gibt es im privaten Bereich noch Lücken, wie auch Batcheller et al \cite{batcheller_testing_2007} in ihrer Arbeit zum Thema «Testing the technology: Playing Games with Video Conferencing» erwähnen. Sie haben in ihrer Studie ebenfalls aufgezeigt, dass Personen, welche das Spiel «Mafia» über eine Videokonferenz spielen, ein ähnliches Mass an Zufriedenheit, Spass und Frustration empfinden wie solche, welche in einer echten Umgebung spielen. Dies deutet auf ein grosses Potenzial hin, um ein Multiplayer-Spiel für die Spieler mit Hilfe eines Videochats reichhaltiger und ansprechender zu gestalten.\\\\
Eine Studie von Veinott et al \cite{veinott_video_1999} hat zudem aufgezeigt, dass Videogespräche eine bessere Basis für die Verständigung zwischen zwei Personen im Geschäftsumfeld schaffen, falls diese etwas untereinander verhandeln müssen. In ähnlicher Weise erfordern Multiplayer-Spiele mit aktiver Kommunikation zwischen den Spielenden oft subtile Hinweise, um Spielentscheidungen zu treffen. Dies deutet darauf hin, dass Videochats einen wesentlichen Einfluss auf das Spielgeschehen haben können.\\\\
Eine Studie von Bos et al \cite{bos_short_2001} hat ausserdem verglichen, wie sich Vertrauen zwischen den Spielern in einem Social Dilemma-Spiel \cite{harrod_social_1983} via Videochat entwickelt. Dabei hat sich gezeigt, dass zwischen Spielenden via Videochat ein vergleichbares Vertrauen entstand, wie in Durchgängen, welche Face-To-Face durchgeführt wurden. Die Studie notiert allerdings auch, dass es verglichen mit den vor Ort durchgeführten Spielen länger dauerte, bis dasselbe Vertrauen aufgebaut wurde.\\\\
Erkenntnisse aus einer Studie von Derboven et al \cite{derboven_designing_2012} deuten ergänzend dazu darauf hin, dass in einem Mehrspieler-Videospiel die zusätzliche Kommunikationsfunktionalität durch einen Videochat oft sowohl von älteren als auch von den jüngeren Personen begrüsst wird. Die Videochat-Funktionalität hatte dem Spiel, welches eine betagte Person jeweils mit einem jüngeren Mitspieler spielte, eine zusätzliche soziale Dimension verleihen. Derboven et al empfehlen daher in ihren abschliessenden Worten sogar, Videochats in intergenerationellen Spielen vermehrt einzusetzen. Allerdings warnen sie auch davor, dass ein Videochat in gewissen Fällen zu unangenehmen Situationen führen kann – besonders wenn die beiden Spieler nicht das Bedürfnis haben, miteinander zu sprechen.\\\\
Es lässt sich also zusammenfassen, dass Videochats in diversen Bereichen einen wesentlichen Einfluss auf die Kommunikation zwischen zwei Parteien haben. Besonders in Videospielen, in welchen der aktive Austausch zwischen den Spielenden Teil des Spielgeschehens ist, sollte dieser Einfluss messbar sein und im Game Design berücksichtigt werden.\\\\
Allerdings bietet die Forschung bisher keine detaillierte Studie mit Kindern und betagten Personen in diesem Zusammenhang.

\chapter{Aufstellung der Forschungsfragen}
\emph{
	\begin{itemize}
		\color{blue}
		\item Lücken der bisherigen Forschung
		\item Aufstellung der Forschungsfragen
	\end{itemize}
}
Aus der in der Einleitung formulierten Ausgangslage stellt sich folgende Game-Design-Frage: Wie lässt sich ein Videospiel für unterschiedliche Altersgruppen ansprechend gestalten? Da diese Frage schon in vielen Studien bezüglich intergenerationellem Spieledesign behandelt wurde \cite{chua_lets_2013}, \cite{de_la_hera_benefits_2017}, \cite{soldati_create_2020}, konzentriert sich das Projekt Palim-Palim in seiner Aufgabenstellung speziell auf den Aspekt der Video-Telefonie in Videospielen. Dies ist besonders interessant, da die Kombination von Videospielen mit Video-Telefonie eine grosse Möglichkeit bietet, Video-Chats für intergenerationelle Altersgruppen attraktiver zu gestalten.\\

\section{Hauptfragestellung}
Mit Palim-Palim soll herausgefunden werden, wie Videospiele und Video-Telefonie kombiniert werden können. Zusätzlich sollen für diese Kombinationen die Auswirkungen auf die Interaktionen zwischen betagten Menschen und Kindern erforscht werden. Deshalb befasst sich das Projekt mit folgender übergeordneter Fragestellung:\\
			\textbf{Wie lässt sich Video-Telefonie mit Video-Spielen kombinieren, damit zwei Personen (Kind und betagte Person) übers Netz miteinander spielen und sich gleichzeitig unterhalten können?}

\section{Einzelfragen im thematischen Zusammenhang}
Um die übergeordnete Aufgabenstellung messbar zu machen, wird sich die begleitende Thesis von PalimPalim mit spezifischen Fragestellungen zu den Themen User Experience (siehe 2.2.1) und Kommunikation (siehe 2.2.2) auseinandersetzen. Die dazu formulierten Hypothesen können dabei durch die Auswertung der Resultate aus den Spieltests verifiziert oder widerlegt werden.

\subsection{Spezifische Fragestellung zur User Experience:}

\paragraph{1. Wie beeinflusst die Einbindung von Video-Telefonie die User Experience in Videospielen?}
	\subparagraph{Fragestellung 1a:} Welche Wirkung hat ein integrierter Video-Chat auf das Spielerlebnis von betagten Menschen?
	\subparagraph{Hypothese 1a:} Ein integrierter Video-Chat hat eine positive Wirkung auf das Spielerlebnis von betagten Menschen.
 
	\subparagraph{Fragestellung 1b:} Welche Wirkung hat ein integrierter Video-Chat auf das Spielerlebnis von Kindern?
	\subparagraph{Hypothese 1b:} Ein integrierter Video-Chat hat eine positive Wirkung auf das Spielerlebnis von Kindern.

	\subparagraph{Fragestellung 1c:} Welche Auswirkungen hat die Art und Weise, wie der Video-Chat in das Spiel integriert ist, auf das Spielerlebnis?
	\subparagraph{Hypothese 1c:} Je stärker der Video-Chat ins Gameplay integriert ist, desto positiver ist das Spielerlebnis.


\subsection{Fragestellung zur Kommunikation:}

\paragraph{2. Wie beeinflusst die Einbindung von Video-Telefonie die Kommunikation in Videospielen?}
	\subparagraph{Fragestellung 2a:} Welche Wirkung hat ein integrierter Video-Chat auf die Kommunikation zwischen den Spielenden? 
	\subparagraph{Hypothese 2a:} Ein im Spiel integrierter Video-Chat fördert die Kommunikation zwischen den Spielenden. 

	\subparagraph{Fragestellung 2b:} Wird ein integrierter Video-Chat aktiv als Kommunikationsmittel zur Bewältigung von Spielaufgaben genutzt? 
	\subparagraph{Hypothese 2b:} Der Video-Chat wird aktiv als Kommunikationsmittel zur Bewältigung der Spielaufgabe verwendet. 

	\subparagraph{Fragestellung 2c:} Welche Auswirkungen hat die Art und Weise, wie der Video-Chat in das Spiel integriert ist, auf die Förderung der Kommunikation? 
	\subparagraph{Hypothese 2c:} Je stärker der Video-Chat ins Spiel integriert ist, desto angeregter ist der Austausch zwischen den Spielenden. 
	
\chapter{Methoden}
\emph{
	\color{blue}
	In Palim-Palim verwendete Methoden
	\begin{itemize}
		\item Methode der Spieletest
		\item Methode der Spieletest-Auswertung
	\end{itemize}}

% -- Praktischer Teil --
\chapter{Spieletests und Resultate}
	\emph{\section{Resultate}
		\begin{itemize}
			\color{blue}
			\item Ergebnisse
			\item Beantwortung der aufgestellten Forschungsfragen
		\end{itemize}}
	\subsection{Erkenntnisse von Seiten der betagten Menschen}
	\subsubsection{Allgemein}
		\begin{itemize}
			\item älteren Menschen fällt es teilweise schwer lange Sätze zu lesen, sind sehr langsam
			\item Wenn das Enkelkind leise spricht, ist der Ton zu leise und kann deswegen nicht gut verstanden werden.
		\end{itemize} 
	\subsubsection{Menu und Spielführung}
		\begin{itemize}
			\item unklar, dass zusammen mit dem Enkelkind bestummen werden kann, was gespielt wird. ("Muss ich ihn fragen, was wir spielen sollen? Muss ich mit ihm sprechen?")
			\item für das Testing wurde nach jedem Druchgang der Raum verlassen, da das Refreshing der Scene noch nicht sauber funktionierte -> dadurch unklar, dass man nach einem Durchgang nicht mehr miteinander sprechen konnte (nicht mehr im Raum war)
			\item Begriff «Raum» ist nicht komplett verständlich («Und Enter oder so?») und wird auch als unschön empfunden («Raum tönt so wie in einer Zelle. Das Wort Raum ist sehr unschön.
Vielleicht Standort? Oder wo du bist? Ich finde das tönt wie eine Einzelzelle. Oder man sagt ich bin gerade im Bad oder in der Küche… Ein bisschen freier machen. Oder ich bin auf dem Balkon, irgend so was. Einfach nicht in einem Raum, das engt so ein.»)
			\item Erwartung ist vorhanden, dass das Spiel weitergeht oder dass noch etwas kommt. Unklar, dass man jetzt das Ziel erreicht hat. «Also ich habe meine Einkäufe getätigt. Und jetzt?»
		\end{itemize}
	\subsubsection{Anleitung}
		\begin{itemize}
			\item bei der Anleitung ist unklar, dass dies noch nicht das Spiel ist. Sinnvoll wäre, wenn das Spiel bereits beginnt und dann die Erklärungen kommen. So ist klar, dass bereits gespielt werden kann. Also dass man bereits nach einem Gegenstand fragen kann, diesen Gegenstand bereits dragen kann und die Einkaufsliste öffnen kann
			\item umstritten, ob die Anleitung jedes Mal wieder erscheinen soll. Ein Proband findet das wichtig und eine Probandin stört sich daran. Antowrt auf die Frage, ob die Anleitung nicht jedes Mal erscheinen soll: «Nein, weil ich will ja spielen, ich will mit dem Enkel Kontakt haben, dann will ich nicht jedes Mal die Gebrauchsanweisung lesen. Weil in diesem kurzen Moment ist es drin (Anmerkung: vergesse ich es nicht).»
			\item Die Anweisung «Frag dein Enkelkind nach Objekten» in der Anleitung ist unklar: Probandin fragt das Kind, was es gerne hätte.
		\end{itemize}
	\subsubsection{Spiel}
		\begin{itemize}
			\item Es ist nicht ganz klar oder auch nicht erwünscht, dass die Gegenstände von der Einkaufslite gekauft werden sollen (es wird einfach mal nach irgendeinem Gegenstand gefragt)
			\item das Sehen der Person ist nicht zwingend notwendig, macht das Ganze aber einfacher. Wenn das Spiel zum ersten Mal gespielt wird (ohne, dass es zuvor mit Video gespielt wurde) ist es sehr schwierig nur mit dem Ton.
			\item Eine Probandin kann mit einem Entchen nichts anfangen.
			\item Hintergrundbild ist unklar: Kontrast müsste höher sein, um die Objekte zu erkennen (Dies würde aber zu noch mehr Unklarheit betreffend Hintergrund führen). Es ist unklar, dass diese Objekte nur als Hintergrund dienen und nicht bestellt werden können oder mit ihnen interagiert werden kann.
			\item Interaktion per Drag war einmal unklar. Probandin hat versucht ein Item anzuklicken und dann auf den Einkaufskorb zu klicken, um das Item im Einkaufskorb abzulegen.
		 \item Objekte die nahe beieinander liegen wurden einmal von einer Probandin als Gruppe wahrgenommen und deswegen wurde erwartet, dass diese miteinander bewegt werden können.
		 \item Die Einkaufsliste wurde versucht mit Draggen statt mit einem Klick zu öffnen
		 \item zu wenig anspruchsvoll / langweilig			
		 \item Hintergrund und Verkaufsobjekte sollten auf die Thematik des Modus abgestimmt sein: «Also ich komme jetzt in einen Gemüseladen und ich sehe ja gar kein Gemüse. Ich sehe Flaschen an der Wand in Pastelltönen und ich habe einen Einkaufszettel, auf dem Sachen stehen, die ich im Laden gar nicht sehe. Und das befremdet eigentlich noch.» --> zusätzliche und vor allem sichtbare Objekte würden für mehr Inspiration sorgen, was noch eingekauft werden könnte.
		 \item «Aktion nur heute» wäre noch spannend, wenn ich noch mehr sehen, wenn man das schon anpreist.
		\end{itemize}
		

\subsubsection{Was funktioniert gut}
	\begin{itemize}
		\item Das Abstreichen des Gegenstands gibt klar zu erkennen, dass dieser gekauft wurde
		\item Das Dragen von Objekten funktioniert sehr flüssig
		\item als Zitat reinnehmen?
		«Also gerade jetzt in der Covid-Zeit, in der man sich nicht gesehen hat, dann wäre das jetzt noch ziemlich gut gewesen, dann hätte man die Oma gesehen oder die Oma hätte den Enkel gesehen und dann hätte man noch ein Spiel machen können. Da würde ich sagen, ja… Es hat Potenzial.» Ursula
	\end{itemize}
	
\subsubsection{Ideen/Wünsche:}
	\begin{itemize}
		\item je nach Spielfortschritt weitere Verkaufsartikel anbieten
		\item nicht nur einen Einkaufsladen anbieten, sondern auch andere Szenarien
		\item Im Regal im Hintergrund hat es Gegenstände, welche man auch kaufen kann, damit man die Gegenstände sieht, die zur Auswahl stehen (wie in einem Tante Emma Laden). Dies inspiriert verschiedene Produkte zu kaufen.
		\item Illustrationen von Verkaufspersonal und Einkaufsperson nicht fix: Rbuen: «Was? Ich war ein Mädchen?»
		\item Einkaufszettel lebendiger gestalten («toter Einkaufszettel»)
		\item selbst bestimmen, was man einkaufen möchte/muss («Also würdest du gerne selbst bestimmen, was du einkaufst?»
«Ja, also wenn ich schon in einen Laden gehe, dann will ich gerne… ich habe mich ja schliesslich zuhause ein wenig vorbereitet, was ich einkaufen möchte, dafür bin ich ja in den Laden gekommen. Wenn man einen Laden sieht, dann lässt man sich inspirieren und sagt sich, ah das ist jetzt toll, wenn sie das haben, möchte ich das unbedingt. Zum Beispiel etwas Saisonales, die Aprikosen. Dann sieht man die und denkt, doch, dann gönne ich mir heute eine Aprikosenwähe. Kannst mir gerne noch Blätterteig dazu verkaufen.»)
	\end{itemize} 
	
	
	
Dieser Absatz beschreibt die Erkenntnisse aus den Spieletests, welche auf der Seite der betagten Menschen erlangt wurden. «Testpersonen» wird synonym für die testenden betagten Personen, verwendet, exklusiv der testenden Kinder.\\\\
Die Spieletests haben gezeigt, dass ältere Personen einiges langsamer lesen als jüngere Menschen, zu dieser Erkenntnis kamen auch (Liu et al. 2017). Von einer Testperson wurde bemängelt, dass es schwierig ist, leise sprechende Kinder über Palim Palim zu verstehen.\\\\
Die Verbindung zwischen den zwei Spielenden findet über eine gemeinsame Raumnummer statt. Die Formulierung «Raum» ist für die Testpersonen undeutlich sowie auch unschön und kalt. Bei der Auswahl des Spielmodus verstand die eine betagte Person nicht, dass diese Entscheidung zusammen mit dem Kind getroffen werden kann – dass während der Auswahl gesprochen werden kann.\\\\
Für beide Testpersonen ist unklar, dass die Abbildung mit den Erklärungen noch nicht zum Spiel gehört. Sie versuchten bereits das Einkaufsobjekt zu verschieben und die Einkaufliste mit tippen zu öffnen. Die Anweisung «Frag dein Enkelkind nach Objekten» war für eine Person unklar, denn sie fragte das Kind, was dieses gerne hätte, obwohl sie nach einem Gegenstand ihrer Einkaufsliste fragen müsste. Die Testpersonen waren sich uneinig, ob es sinnvoll ist, die Anleitung vor jedem Spielstart anzuzeigen. Eine Testperson findet dies wichtig, damit klar ist, was gemacht werden soll, die andere Person ist der Meinung, dass es mühsam sei und sie in dieser kurzen Zeit die simplen Funktionalitäten nicht vergessen habe.\\\\
Zu Beginn des Gameloops ist es für die Testpersonen unklar, dass sie nach einem Gegenstand fragen sollen, welcher auf der Einkaufsliste zu finden ist. Stattdessen fragen sie nach einem beliebigen Produkt.\\\\
Der illustrierte und zu 70 Prozent transparente Hintergrund sorgte für Unklarheiten. Einerseits wurde von den Testpersonen bemängelt, dass dieser thematisch zu wenig auf den Spielmodus abgestimmt ist und andererseits ist es schwierig diesen als Hintergrund zu erkennen. Der niedrige Kontrast, welcher durch die hohe Transparenz entsteht, sorgt dafür, dass es für die betagten Personen schwierig ist, die einzelnen Objekte des Hintergrunds auseinanderzuhalten. Dass ältere Personen Mühe mit dem Kontrastsehen haben, bestätigt die Studie von xxxxxxx aus dem Jahre xxxxx. //Todo Studie suchen!\\\\
Zu den Interaktionen konnten die drei folgenden Beobachtungen gemacht werden. Einmal war es unklar, dass ein Objekt mit Ziehen verschoben werden kann, stattdessen wurde versucht zuerst auf das Objekt zu tippen und danach an den gewünschten Zielort. Als zwei gleiche Objekte sehr nah nebeneinander platziert waren, wurden diese als Gruppe wahrgenommen und versucht, diese zusammen an den gewünschten Zielort zu ziehen, was jedoch nicht funktionierte. Des Weiteren hat eine Testperson die Einkaufsliste zuerst mit ziehen statt mit tippen versucht zu öffnen.\\\\
Das mitspielende Kind zu sehen, ist laut den Aussagen der beiden Testpersonen nicht zwingend notwendig und führt daher auch zu keiner veränderten Wirkung des Spielerlebnisses. Beim zweiten Spieletest, bei welchem zuerst die Variante ohne Video gespielt wurde, konnte beobachtet werden, dass es schwierig ist, mit dieser Version zu starten.\\\\
Nach dem Konfettiregen, welcher das Ende des Spiels simuliert, war für eine Testperson unklar, dass das Spiel nun zu Ende ist, respektive wie es nun weitergeht.\\\\
Im Allgemeinen wurde Palim Palim von einer Testperson als zu wenig anspruchsvoll und langweilig bezeichnet.\\\\

//Todo positive Sachen ausformulieren\\\\
//Todo Ideen/Wünsche ausformulieren\\\\
	
	
	
	\subsection{Erkenntnisse von Seiten der Kinder}
	
	Beim Beobachten der Kinder ist aufgefallen, dass ihr Spielerlebnis schon zu Beginn durch die einseitige Bedienung der Gamelobby beeinträchtigt wurde. Das Kind kann in Palim-Palim nicht mit der Gamelobby interagieren und sieht ausser dem Video auch keine visuellen Indikatoren darüber, was das Gegenüber gerade auswählt. Kombiniert mit dem Fakt, dass sein betagtes Gegenüber teilweise ein bisschen länger braucht, um sich zurechtzufinden, führt dies schon vor dem Spielbeginn zu Interessensverlust. Die Gamelobby sollte in so einem asymmetrischen Spiel wie Palim-Palim also möglichst für beide Spieler interaktiv sein sollte. Sie müsste so gestaltet werden, dass beide Seiten sehen, was die Gegenseite gerade auswählt. So könnte schon in der Gamelobby ein aktiverer Austausch stattfinden.\\\\
Nach dem Start des Spiels wird in Palim-Palim zuerst ein Screenshot des Spiels mit grafischen Elementen als Erklärung der Spielmechanik gezeigt. Die Kinder versuchten ebenfalls, wie die betagten Personen, auf dieser Abbildung bereits die Objekte zu verschieben. Sie waren dementsprechend verwirrt als nichts passierte. Hieraus lässt sich folgern, dass aus bisherigen Erfahrungen mit Games ein interaktives Tutorial erwartet wird. Ebenfalls kommt hier wieder die einseitige Bedienung ins Spiel: die Kinder wussten nicht, dass nur die betagte Person am anderen Ende die Spielanleitung schliessen kann. Dadurch entstand der Eindruck, dass das Spiel nicht responsiv sei. Dies lässt sich wieder auf die asymmetrische Gestaltung des Spieles zurückführen, welche bei Spielen basierend auf einem Video-Chat natürlich inhärent sind. Es sollte also immer bedacht werden, dass schon vor Spielbeginn eine Synchronisation und wenn möglich auch die Visualisierung aller Interaktionen des Remote-Peers auf jedem Client stattfinden sollten.\\\\
Die Spielmechanik von Palim-Palim selbst wurde von den Kindern sehr schnell begriffen, sobald sich die Spieler in der 3D-Szene befanden. Sie hatten keine Probleme die Gegenstände dem Gegenüber anzubieten. Das Dragging der Objekte funktionierte einwandfrei. Jedoch fehlten Reaktionen des Spieles. Es wurde bemängelt, dass nichts passiert. Dies ist jedoch auf den jungen Entwicklungsstand des Spiels zurückzuführen.\\\\
Das fehlende Video in den Modi 1 und 2 wurde von den Kindern bemerkt. Es wurde sogar mehrmals versucht, mit dem Video der betagten Person zu interagieren. Zum Beispiel wurde versucht die Gegenstände dem Gegenüber zu füttern oder damit das Gesicht lustiger zu gestalten. Dies zeigt, dass ein Video-Stream als Teil der Gameplay-Szene für die Kinder kein fremdes Element ist. Vielmehr bietet sich hier eine grosse Möglichkeit, noch mehr lustige Interaktionen einzubauen.\\\\
Aus diesen Beobachtungen sowie dem qualitativen Feedback der Kinder lässt sich folgern, dass ein Video-Chat durchaus positive Wirkungen auf das Spielerlebnis haben kann. Jedoch muss darauf geachtet werden, dass Kinder hohe Erwartungen an ein Spiel haben. Sie brauchen mehr Inhalte und Interaktionsmöglichkeiten, um die Spannung aufrecht zu erhalten. Es sollte also bewusst mehrere Interaktionsmöglichkeiten geschaffen werden. Insbesondere Interaktionsmöglichkeiten mit dem Video fühlen sich für sie natürlich an und sollten unbedingt in das Gameplay integriert werden.\\\\


	\subsection{Allgemeine Erkenntnisse}
	
	\subsection{Notizen zur Beantwortung der Forschungsfragen}
	\subsubsection{1a}
	Die zwei Testings haben dies nicht bewiesen. Für den einen Probanden braucht es das Video nicht unbedingt: «Jetzt hast du den Ruben gar nicht gesehen, sein Bild nicht gesehen. Hast du das jetzt schlechter oder besser gefunden? oder hat dich das gestört?» «Nein, spielt eigentlich keine Rolle.» «Also musst du ihn nicht unbedingt sehen?» «Nicht unbedingt.»
	-> diese beiden Probanden kennen sich, sind jedoch nicht verwandt
	Für die andere Probandin hat der integrierte Videochat keine grosse Auswirkung auf das Spielerlebnis, jedoch findet sie den dritten Modus herzig, da man das Kind hinter der Theke sieht.
	-> diese beiden ProbandInnen kannten sich vorher nicht
	=> vielleicht würde dies bei einem persönlichen Bezug zueinander anders empfunden werden.
	\subsubsection{1b}
	
	\subsubsection{1c}
	Das Testing zeigte keine unterschiedliche Wahrnehmung des Spielerlebnis, ob der Videostream in die Spielszene integriert oder unabhängig davon ist.
	\subsubsection{2a}
	
	\subsubsection{2b}
	Game zu wenig weit, um dies zu testen.
	\subsubsection{2c}
				
	\section{Planung und Durchführung der Spieletests}
		\emph{\begin{itemize}
			\color{blue}
			\item Organisation der Spieletests (Aufbau, Ablauf, Testpersonen, Testszenarien, Testumgebung)
			\item Beobachtungen
		\end{itemize}}
	
Die Überprüfung der im Kapitel 2 aufgestellten Hypothesen wird durch Spieltests mit den unterschiedlichen Spielvarianten geschehen. Das Ziel dabei ist es, mit den Methoden Beobachtung und Befragung die User Experience sowie die Spielenden-Kommunikation messbar zu machen. Das Beobachten dient einerseits dazu, sofortige und auch unterbewusste Reaktionen bei den Testpersonen zu erkennen. Andererseits möchten wir mittels Befragungen qualitatives Feedback zu den einzelnen Spielvarianten einholen. Dazu wird für die Spieltests vorab ein passender Fragenkatalog erarbeitet, bzw. ein entsprechendes Beobachtungs-Protokoll vorbereitet und geführt.\\\\
			In der Projektwoche 12 wird mit einem funktionalen Prototyp ein erstes Mal Feedback zu den Grundmechaniken eingeholt. Diese erste Standortbestimmung wird mit Bekannten durchgeführt.\\\\
			Für die tatsächlichen Spieltests werden sechs Test-Paare benötigt, wobei ein Test-Paar sich aus einer betagten Person sowie einem Kind zusammensetzt. In der Woche 19 und 20 wird mit allen Paaren getestet. Um die Testpersonen zu finden, wird mit der Organisation terzStiftung \cite{noauthor_terzstiftung_nodate} zusammengearbeitet. Mithilfe eines Newsletters werden die Mitglieder der terzStiftung für ein Testing angefragt. Die angefragten betagten Menschen sind mindestens 65 Jahre alt und weisen eine höhere Technik-Affinität auf als ein Abbild der Gesamtheit dieser Altersgruppe.\\\\
			Nach Möglichkeit (Situation SARS-CoV-2) werden die Spieltests bei den Probanden zuhause, an der FHNW in Brugg, an einem öffentlichen Ort oder remote durchgeführt (absteigende Priorität). Bei den drei erstgenannten Varianten würde das Projektteam die Beobachtungen direkt vor Ort vornehmen, sollten die Tests remote durchgeführt werden müssen, würden die Projektmitglieder sich online zuschalten. Die Probanden müssen nicht zwingend ein Tablet besitzen, dies kann Ihnen von uns zur Verfügung gestellt werden. Jedoch muss bereits vor den Tests ein solches benutzt worden sein.	
	
\chapter{Gamedesign}
	\section{Spielkonzept}
		Das dem Videospiel zugrunde liegende Spielkonzept lehnt sich dem Kinder-Kaufladen-Spielprinzip aus dem echten Leben an. Das Kind und die betagte Person schlüpfen dabei in die entsprechenden Rollen von Verkaufspersonal und Kundschaft in einem virtuellen Einkaufsladen. Die Spieler*innen sehen dabei jeweils einen gemeinsamen Tresen aus der Perspektive ihrer Rolle (siehe Abbildung 1). Beide Seiten können Objekte auf dem Tresen platzieren, beziehungsweise Objekte vom Tresen wegziehen. Zusätzlich besteht die Möglichkeit, die Objekte auf den Video-Stream des Gegenübers zu ziehen. Durch all diese Aktionen können je nach Objekt diverse Animationen, Filter-Effekte und Gameplay-Events ausgelöst werden.
		\subsection{Gameplay-Loops}
		Durch die Aufteilung in die beiden Rollen lassen sich für beide Spieler*innen verschiedene Aktionen und somit ein unterschiedliches Gameplay definieren. Diese Art der Unterteilung eignet sich hervorragend, um das Gameplay jeweils auf die spezifischen Anforderungen von Kindern bzw. betagten Personen anpassen zu können.\\\\
		Der Gameplay-Loop ist in der nachfolgenden Abbildung 2 skizziert. Die Haupt-Interaktionen zwischen den beiden Spieler*innen bestehen aus der Kommunikation der Einkaufsliste und dem Hin- und Herreichen von Objekten auf der gemeinsamen Tresen-Fläche.\\\\
		\emph{\color{blue} Hier das Spiel (aktueller Stand beschreiben und nur auf mögliche Weiterentwicklungen hinweisen. Nicht alle möglichen angedachten Funktionen auflisten, das machen wir erst am Schluss}\\\\
		Dieses übergeordnete Gameplay soll als Leitfaden für den Spielverlauf dienen. Allerdings bietet das Spiel durch die gemeinsame Tresen-Fläche und dem interaktiven Video-Stream bewusst Möglichkeiten für sonstige Interaktionen mit dem Gegenüber. Damit soll für die Benutzerinnen und Benutzer ein gewisser Freiraum entstehen, in welchem sie den Verlauf des Spiels auf spielerische Art und Weise beeinflussen können. Dies soll das Spiel besonders für das Kind interessanter machen, um eine aktive und lustige Kommunikation mit der betagten Person zu fördern. 
	\section{Spielvarianten}
	Damit der Einfluss der Video-Telefonie als Teil des Spiels messbar wird, werden unterschiedliche Varianten des Spiels implementiert und getestet: 
\begin{enumerate}
	\item Spiel ohne Video-Chat 
	\item Spiel mit Video-Chat, aber getrennt von der Spiel-Szene 
	\item Spiel mit Video-Chat, in die Spiel-Umgebung integriert 
	\item Spiel mit Video-Chat, in die Spiel-Umgebung integriert und optionale Interaktionsmöglichkeiten mit dem Video-Stream 
	\item Spiel mit Video-Chat, speziell in die Spiel-Umgebung integriert und zum Spielerfolg notwendige Interaktionsmöglichkeiten mit dem Video-Stream 
\end{enumerate}
\emph{\color{blue}Tabelle Spielvarianten, klar zeigen welche implementiert wurden und welche nicht. Screenshots einzelner Varianten}
	\section{Design und Usability}
	\emph{\color{blue} Warum 3D-Welt, Wahl der 3D-Assets, Platzierung der Viedos gegenüber, Asysmetrische Gameplay (ältere Person sieht Liste, Jüngere nicht), Positionierung Tablet/Kamera, Bedienung mit Touch (Sphere um Items zum Draggen) - Gesten, Aufteilung Screenspace}

\chapter{Technologien}
	Palim-Palim kombiniert die Funktionalitäten einer Video-Chat-Applikation mit einem Multiplayer-Spiel. Um diesen Anforderungen gerecht zu werden, wurden entsprechende Technologien zur Umsetzung ausgewählt. Dabei fiel die Wahl auf WebRTC zur Übertragung der Multimedia-Daten, sowie Three.js als 3D(dreidimensional)-Bibliothek für die Gestaltung und Umsetzung der 3D-Gameplay-Szene.
	
			\section{WebRTC}
WebRTC ist ein Open-Source-Projekt, das die Echtzeitkommunikation von Audio, Video und Daten in Web- und nativen Anwendungen ermöglicht \cite{noauthor_webrtc_2011}. Die Technologie ist in allen modernen Browsern sowie auf nativen Clients für alle wichtigen Plattformen verfügbar, ist eine Empfehlung des World Wide Web Consortium (W3C) \cite{noauthor_webrtc_2021} und ein Standard der Internet Engineering Task Force (IETF) \cite{noauthor_support_2021}. Dabei wird zwischen zwei Browsern eine Peer-To-Peer-Verbindung aufgebaut, worüber die Daten gestreamt werden. Auf der Peer-To-Peer-Verbindung lassen sich auch eigene Daten-Kanäle erstellen, die zum Beispiel zur Übertragung von Text-Nachrichten, Positions-Daten von Game-Objekten oder sogar Dokumenten zwischen den Peers verwendet werden können \cite{noauthor_webrtc_nodate}.
			\subsection{Signaling-Server}
			WebRTC verwendet die RTCPeerConnection JavaScript API (Application Programming Interface), um Streaming-Daten direkt zwischen Browsern zu kommunizieren \cite{noauthor_rtcpeerconnection_nodate}. Doch wie wird dieser Prozess initiiert? Die beiden Peers - genannt Alice und Bob - kennen sich zu Beginn ja gar noch nicht. Zusätzlich wird also ein Mechanismus benötigt, welcher die Kommunikation der beiden Peers koordiniert und Kontrollnachrichten sendet. Dieser Prozess wird in WebRTC als \textit{Signaling} bezeichnet. Signaling-Methoden und -protokolle sind von WebRTC nicht spezifiziert und können je nach Anwendungsfall entsprechend gewählt werden.  \\	
			In Palim-Palim wurde die JavaScript-Library Socket.io für die Signalisierung verwendt \cite{arrachequesne_socketio_2021}. Socket.io erlaubt eine einfache und bidirektionale Kommunkation zwischen den Clients und einem Signaling-Server. Mit seinem integrierten Room-Konzept eignet sich Socket.io zudem sehr gut für eine Video-Chat-App. Eine Node.js Server-Applikation fungiert als Signaling-Server für Palim-Palim. Der Server hat dabei folgende zwei Aufgaben: er dient als Message Relay und verwaltet alle Videochat-Räume. \\\\
Die Funktionaliät als Message Relay ist wichtig, da sich Alice und Bob vor dem Verbindungsaufbau noch nicht kennen. Ein Signaling-Server, welcher beiden Clients bekannt ist, wird benötigt. So können initiale Informationen von Alice zu Bob ausgetauscht werden, damit diese untereinander eine WebRTC-Peer-Verbindung aufbauen können:
			\begin{lstlisting}
socket.on('message', function (message) {
  socket.to(room).emit('message', message);
});
\end{lstlisting}
Zusätzlich verwaltet der Signaling-Server alle WebRTC-Videochat-'Räume'. Tritt zum Beispiel Alice einem Raum bei, so überprüft der Server, ob die Raumkapazität schon erreicht wurde und sendet entsprechende Antworten an sie:
\begin{lstlisting}
if (numClients === 0) {
  socket.join(room);
  socket.emit('created', room, socket.id);
} else if (numClients === 1) {
  socket.join(room);
  socket.emit('joined', room, socket.id);
  io.sockets.in(room).emit('ready');
} else { // max two clients
  socket.emit('full', room);
}
\end{lstlisting}
Wie im Code ersichtlich ist, erlaubt Palim-Palim maximal zwei Peers in einem Raum. Ist Alice die erste Spielerin, erhält sie die Antwort 'created' vom Signaling-Server. Ist bereits ein Client in dem Raum, wird ihr ein 'joined' zurückgegeben. Wurde die Raum-Kapazität überschritten, sendet der Palim-Palim Server Alice ein 'full'. Diese Nachrichten werden dann Client-seitig im Browser von Alice entsprechend behandelt.


			\subsection{STUN- und TURN-Server}			

			WebRTC ist grundsätzlich so konzipiert, dass es Peer-to-Peer funktioniert. Alice und Bob können sich also auf dem direktesten Weg verbinden. Die Technologie ist jedoch auch bewusst darauf ausgelegt, mit realen Netzwerken zurechtzukommen: Client-Anwendungen müssen  NAT-Gateways (Network Address Translation) und Firewalls überwinden, und Peer-to-Peer-Netzwerke benötigen Fallbacks, falls die direkte Verbindung ausfällt. Als Teil dieses Prozesses verwendet die WebRTC-API zwei Netzwerkprotokolle als Hilfsmittel: 
			
\begin{itemize}
  \item \textit{STUN} (Session Traversal Utilities for NAT), um die öffentliche IP-Adresse und Port-Nummer für den direkten Kontaktaufbau zu ermitteln  \cite{noauthor_stun_2021}.
  \item \textit{TURN} (Traversal Using Relay NAT), welches die Kommunikation über NAT- oder Firewallgrenzen hinweg ermöglicht und als Fallback-Relay genutzt werden kann, sollte keine Peer-to-Peer-Verbindung möglich sein \cite{noauthor_traversal_2021}.
\end{itemize}

Für beide Protokolle müssen Server bereitgestellt werden, welche den WebRTC-Clients bekannt sein müssen. Die STUN- und TURN-Adressen müssen bei dem Aufbau der Peer-Connection der Clients an den Signaling-Server übermittelt werden \cite{noauthor_turn-server_nodate-1}. Dazu setzt jeder Palim-Palim-Client beim Verbindungsaufbau in den Konfigurations-Werten die Addressen der STUN- und TURN-Server:
			
			\begin{lstlisting}
			this.peerConnectionConfig = {
            'iceServers': [
                {
                  'urls': 'stun:stun.l.google.com:19302'
                },
                {
                  'urls': 'turn:86.119.43.130:3478',
                  'credential': '*****************',
                  'username': 'palimpalim'
                }
            ]
        };
			\end{lstlisting}
			
			 
Nur mit diesen beiden Server-Addressen im entsprechenden Konstruktor kann die Funktionalität des Video- und Audio-Streams über das Internet gewährleistet werden. Palim-Palim verwendet als STUN-Server einen öffentlich verfügbaren Server von Google. Da der STUN-Server nur zur Ermittlung der eigenen öffentlichen IP-Addresse dient, lässt sich hier ohne grosse Auswirkungen auf den Datenschutz oder die Sicherheit ein öffentlich gehosteter Server verwenden. \\	

Als Fallback-Verbindungen wird allerdings im produktiven Betrieb einer WebRTC-Applikation unbedingt ein eigener TURN-Server mit einer öffentlich sichtbaren IP-Adresse benötigt \cite{noauthor_webrtc_2020}. Es gibt keine kostenlosen, öffentlich gehosteten TURN-Server, da der Netzwerkverkehr über so einen Server sehr stark ansteigen kann. Daher lohnt es sich für niemanden, fremden Verkehr über seinen TURN-Server zuzulassen. Es ist also ebenfalls von Vorteil, seinen TURN-Server mit einer ensprechenden Authentifizierung zu versehen. Palim-Palim verwendet einen auf SwitchEngines gehosteten TURN-Server. Dieser Linux-Server verwendet Coturn, eine Open-Source Implementierung des TURN-Protokolls \cite{noauthor_coturncoturn_2021}.  \\	
			
Gemäss Auswertungen des WebRTC Call Quality Dienstleisters callstats.io haben im Durchschnitt etwa 30 Prozent aller WebRTC-Sessions einen Client, der sich über einen TURN-Server verbindet \cite{callstats_why_nodate}. Das heisst in etwa 70 Prozent der Fälle ist die Peer-To-Peer-Verbindung stabil genug, und ein TURN-Server wird überhaupt nicht benötigt. Jedoch müssen sie trotzdem bei jeder Verbindung angegeben werden. Denn gewisse Benutzer*innen wären ohne die Unterstützung eines TURN-Relay-Servers nicht in der Lage zu kommunizieren.


			\begin{figure}[h!]
			\includegraphics[width=\textwidth]{WebRTC_nat_stun_firewall_turn_black}
			\caption[Caption for LOF]{STUN, TURN, und Signalisierung in WebRTC (eigene Darstellung).}
			\end{figure}
			
Mit STUN-, TURN- und Signaling-Server besitzt Palim-Palim ein stabiles Backend für die Implementierung des Video-Streams mit WebRTC. Zusätzlich kann die Peer-To-Peer-Funktionalität von WebRTC auch zur Synchronisation von Gameplay-Daten direkt zwischen Alice und Bob genutzt werden. Die Multiplayer-Funktionalität kann so auch unabhängig von einem zentralen Server als gemeinsame Authorität ermöglicht werden.
			
		\subsection{Sicherheit}
		Der Open-Source-Charakter von WebRTC kann bei potenziellen Anwender*innen der Technologie Sicherheitsbedenken hervorrufen. Diese Bedenken sind berechtigt und wurden bei der Entwicklung von Palim-Palim ebenfalls berücksichtigt. Damit der Datenschutz für die Spieler*innen sowie die Stabilität des Games gewährleistet werden kann, müssen bei der Implementation einige Punkte berücksichtigt werden. \\
		
		Auf der Protokoll-Ebene ist WebRTC als Standard sehr sicher und auch bereits etabliert. Verschlüsselung ist für alle WebRTC-Komponenten obligatorisch, und seine JavaScript-APIs können nur von sicheren Quellen (HTTPS oder localhost) aus verwendet werden \cite{noauthor_webrtc_2020-1}. Das verwendete Verschlüsselungs-Protokoll hängt dabei vom Kanaltyp ab. Daten-Kanäle werden mit Datagram Transport Layer Security (DTLS) und Medien-Kanäle mit Secure Real-time Transport Protocol (SRTP) verschlüsselt. Unverschlüsselte Kommunikation gibt es in WebRTC also nicht. Die Schlüssel zur Entschlüsselung der Medien-Kanäle werden zudem nicht über den Signaling-Server ausgetauscht, sondern nur direkt zwischen den Peers. \\
		
		Ebenfalls werden bei der Umleitung des Datenverkehrs über den TURN-Server die Daten nicht interpretiert oder modifiziert. Dies ist so im TURN-Standard definiert \cite{noauthor_rfc5766_nodate}. Das heisst, die Verwendung des TURN-Servers fügt dem WebRTC-Datenverkehr keine Sicherheitsschwachstellen hinzu. Die Verbindung zum Signaling-Server ist ebenfalls mit HTTPS geschützt, was als sicher genug für Online-Banking und Behörden-Websites gilt \cite{noauthor_study_nodate}. \\
		
		Somit lässt sich WebRTC als eigenständiges Framework für Videotelefonie als sehr sicher bezeichnen. Allerdings hängt diese Sicherheit auch von der umliegenden Web-Applikation und diese wiederum von dem verwendeten Browser ab. Als ein im Internet zugängliches Spiel sollte Palim-Palim deshalb über eine entsprechende User-Authentifizierung erweitert werden, sollte das Spiel dauerhaft produktiv betrieben werden wollen. Ebenfalls empfiehlt es sich, die WebRTC-Bibliothek und andere Abhängigkeiten aktuell zu halten.
	
	\section{Three.js}
	Three.js ist eine 3D-Bibliothek, welche meistens die Web Graphics Library (WebGL) verwendet, um 3D-Inhalt zu zeichnen \cite{noauthor_threejs_nodate}. WebGL ist eine Rasterisierungs-Engine, das bedeutet, dass diese Bibiliothek nur Punkte, Linien und Dreiecke zeichnet und erst auf Basis des Codes entstehen dreidimensionale Objekte \cite{noauthor_webgl_nodate}. In Three.js stehen primitive 3D-Objekte genauso wie beispielsweise Lichter, Schatten, Materialien oder Texturen als vorgefertigte Code-Pakete zur Verfügung \cite{noauthor_threejs_nodate}.
	

	\subsection{Struktur}
Das oberste Element einer Three.js-Struktur ist der Renderer. Dieser nimmt eine Szene (Scene) und eine Kamera (Camera) entgegen und zeichnet den im Blickfeld befindlichen Teil der 3D-Scene auf eine 2D-Leinwand.
In der Scene werden Lichter (Light), 3D-Objekte (Object3D) und Kameras (Camera) platziert. Die Scene ist die Wurzel einer Baumstruktur, in welcher Kinder relativ zu ihren Eltern ausgerichtet werden.
Mesh-Objekte sind gezeichnete Formen (Geometry) mit einem bestimmten Material. Geometry- und Material-Objekte können von verschiedenen Mesh-Objekten verwendet werden. Three.js bietet bereits einige primitive Geometry-Objekte an, wie beispielsweise Würfel (Cube), Zylinder (Cylinder) oder Pyramiden (Cone). Es können aber auch eigene Geometries erstellt oder aus einer Datei importiert werden. Material-Objekte repräsentieren die Oberflächenbeschaffenheit eines Objekts, wie zum Beispiel die Farbe oder wie fest ein Objekt spiegelt \cite{noauthor_threejs_nodate}.


\chapter{Implementation}
In diesem Kapitel werden die einzelnen Komponenten von Palim-Palim und deren Funktionen vorgestellt. Als erstes wird das Server-seitige Backend erklärt und danach die beiden zentralen Client-Komponenten, der PeerConnectionManager sowie der GameManager vorgestellt. Die Struktur von Palim-Palim wurde bewusst so aufgebaut, dass die Video-Chat-Funktionalität möglichst von der Game-Logik entkoppelt ist. Dadurch ist es einfacher, den Video-Chat auch in anderen Spielen zu integrieren. Ebenfalls war die Absicht, dadurch eine möglichst gute Grundlage zur Entwicklung eines entsprechenden Frameworks zu legen.
	
\section{Server}\label{server}
Da Palim-Palim über das Internet zugänglich sein soll, benötigt das Spiel einen Server, welcher den Client Code zur Verfügung stellt. Der Server-Code im server.js File nutzt daher Express.js (Link zu express), womit der mit Webpack gebaute Client-Javascript-Code den Benutzer beim Aufruf der Webapplikation zur Verfügung gestellt wird.\\\\
Der Server verwaltet zusätzlich alle Videochat-Räume und dient als Vermittler für den Verbindungsaufbau des Video-Calls. Dies wird auch Signaling genannt (siehe Kapitel «Signaling») und funktioniert folgendermassen: Sobald zwei Spielende einem Raum betreten, wird dieser Raum genutzt, um Verbindungsinformationen zwischen den Peers auszutauschen. Dies wurde in Palim-Palim gemäss einem Google Codelab Beispiel umgesetzt \cite{noauthor_real_nodate}. Socket.io erlaubt eine einfache und bidirektionale Kommunikation zwischen den Clients und dem Server. Mit seinem Room-Konzept eignet sich Socket.io zudem sehr gut für eine Video-Chat-App. Über den Socket.io-Raum kann die Peer-To-Peer-Verbindung zwischen den Clients dann ausgehandelt und erstellt werden. Steht diese direkte Verbindung einmal, wird der Server nicht mehr benötigt. Ab dann läuft die Kommunikation in den meisten Fällen nur noch direkt zwischen den Clients.\\\\
Um unabhängig zu funktionieren, wurde für Palim-Palim zudem ein eigener TURN-Server aufgesetzt. Der im Projekt verwendet TURN-Server läuft mit Linux Ubuntu und benutzt Coturn, eine Open-Source Implementierung des TURN-Protokolls \cite{noauthor_coturncoturn_2021}.\\\\
Um selbst ein TURN-Sever bereitzustellen, wird lediglich ein Server mit einer öffentlichen IP-Adresse benötigt. Nach der Konfiguration kann die Adresse des TURN-Servers beim Erstellen der PeerConnection im Client-Code von Palim-Palim in der Konfiguration mitgegeben (siehe Codefragment \ref{lst:peerConfig}). Eine Anleitung zum Einrichten und Konfigurieren eines Linux-TURN-Servers inklusive Verweise auf weitere Ressourcen zu dem Thema sind im Anhang dieser Arbeit zu finden.\\\\

	\begin{lstlisting}[caption={Konfiguration der PeerConnection mit der TURN-Server Adresse},label={lst:peerConfig},language=JavaScript]
this.peerConnectionConfig = {
	'iceServers': [
		{
			'urls': 'stun:stun.l.google.com:19302'
		},
		{
			'urls': 'turn:86.119.43.130:3478',
			'credential': '*****************',
			'username': 'palimpalim'
		}
	]
};
	\end{lstlisting}
	
\section{PeerConnectionManager}
Die Video-Chat-Funktionalität wurde in Palim-Palim möglichst unabhängig vom Gameplay als eigene Klasse implementiert. Die PeerconnectionManager-Klasse übernimmt hierbei zentrale Funktionen wie das Betreten eines Raumes, die Etablierung der Video-Streams zwischen den Spielern sowie den Aufbau von dedizierten Datenkanälen (sogenannten DataChannels).\\\\ In der PeerConnectionManager-Klasse selbst findet der Aufbau der Peer-To-Peer-Verbindung statt. Die Klasse instanziiert hierzu ein PeerConnection-Objekt, welches die RTCPeerConnection erweitert. RTCPeerConnection ist die API, welche von WebRTC-Anwendungen verwendet wird, um eine Verbindung zwischen Peers herzustellen und Audio und Video zu übertragen \cite{noauthor_rtcpeerconnection_nodate-1}. Diese PeerConnection stellt eine WebRTC-Verbindung zwischen dem lokalen Computer und einem entfernten Peer dar. Sie bietet Methoden, um eine Verbindung zu einer Gegenseite herzustellen, die Verbindung aufrechtzuerhalten und zu überwachen sowie die Verbindung zu schließen, wenn sie nicht mehr benötigt wird. Der PeerConnectionManager orchestriert über dieses Objekt den Verbindungsaufbau und dient als einzige Schnittstelle des Video-Chats zum Rest der Applikation.\\\\
Um alle diese Funktionen innerhalb dieser Klasse möglichst übersichtlich zu umzusetzen, ist die PeerConnectionManager-Klasse von Palim-Palim selbst in einzelne Unterklassen mit den entsprechenden Teil-Verantwortlichkeiten aufgeteilt. Diese Komponenten und deren Zuständigkeiten werden in den nachfolgenden Kapiteln kurz vorgestellt.

\subsection{RoomManager}
Der RoomManager ist als einzige Schnittstelle mit dem Server für das WebRTC-Signaling des Clients zuständig. Er besitzt die Verantwortung über die Client-seitige Socket.io-Verbindung. Der RoomManager ermöglicht es somit einem Client via Socket.io einem Raum beizutreten. Der PeerConnectionManager verwendet den RoomManager, um Signaling-Messages an den Server zu versenden sowie diese zu empfangen. Auf dem Server werden die Anfragen der einzelnen Clients dann gehandelt, wie in Kapitel \ref{server} genauer beschrieben.

\begin{figure}[!htb]
\includegraphics[width=\textwidth]{RoomManagerSd_noFrame}
\caption[Caption for LOF]{Nachrichtenfluss beim Betreten eines Raumes (eigene Darstellung).}
\end{figure}

\subsection{VideoChatManager}
Sobald der Spieler einen Raum betreten hat, wird via VideoChatManager der lokale Video- und Audiostream initiiert. Im JavaScript-Code geschieht dies via Navigator, wodurch der Browser den User um die Erlaubnis der Verwendung von Mikrofon und Kamera bittet. Erst wenn diese Erlaubnis gegeben wurde, kann auf die Daten von Kamera und Mikrofon zugegriffen werden. Der VideoChatManager fügt dann den Stream dem entsprechenden DOM-Element hinzu, wodurch die Spieler ihr eigenes Video sehen können. Ebenfalls löst dies in Palim-Palim ein «got user media»-Event aus, wodurch der PeerConnectionManager sowie der Signaling-Server informiert werden, dass dieser Client jetzt Zugriff auf Video- und Audio-Daten hat. Danach können der PeerConnection die MediaTracks hinzugefügt werden. \\\\
Das Hinzufügen der Tracks auf der PeerConnection löst in WebRTC auf der Gegenseite automatisch ein «trackAdded»-Event aus \cite{noauthor_webrtc_nodate-1}. Dieses wird ebenfalls durch den VideoChatManager gehandlet. Fügt also das Gegenüber der gemeinsamen PeerConnection seinen Track zur Verbindung hinzu, werden im gegenüberliegenden VideoChatManager aus diesem Event die nötigen Daten für den Remote-Video-Stream ausgelesen. Dadurch entsteht eine Video-Konferenz zwischen den beiden Mitspielern.\\\\

(Grafik Ablauf getMedia, Track added, handleTrack added)

\subsection{DataChannelManager}
Die dritte Teil-Komponente des PeerConnectionManagers ist der DataChannelManager, welcher für das Erstellen von Datenkanälen zuständig ist. Diese sogenannten DataChannels erlauben das Versenden beliebiger Daten über die PeerConnection \cite{noauthor_webrtc_nodate}. In Palim-Palim wird diese Funktion genutzt, um die 3D-Positionen der Gegenstände zu synchronisieren, sowie den Remote-Peer über spezifische Game-Events zu informieren. Dazu werden zwei einzelne DataChannels erstellt, welche unabhängig voneinander, aber auf der gleichen PeerConnection laufen.\\\\
Der darunterliegende RTCDataChannel besitzt einige Properties, um die Übertragungsart genau an die Anforderungen anzupassen. Zum Beispiel kann mittels dem Property «ordered» bestimmt werden, ob die Nachrichten in der gleichen Reihenfolge ankommen wie sie versendet wurden. WebRTC baut dann im Hintergrund je nach Konfiguration des DataChannel-Objekts eine entsprechende TCP oder UDP-Verbindung auf \cite{noauthor_rtcdatachannel_nodate}. Beim Erstellen eines DataChannels muss dabei ebenfalls die Callback-Funktion angegeben werden, welche im Falle einer Message auf diesem Kanal aufgerufen werden soll. Deshalb wird hier eine Referenz auf den GameSyncManager von Palim-Palim gebraucht, welcher in Kapitel x genauer beschrieben ist.

	
	\section{Game}
Neben dem PeerConnectionManager als zentrale Steuereinheit der Videochat-Funktionalität, bildet der GameManager die zweite zentrale Klasse und übernimmt die zentralen Funktionen des Spiels Palim-Palim. Dazu gehört das Starten des Games, die Verwaltung der Game-Lobby, das Handling von Spielgeschehnissen sowie das Handling des Spielendes. Um diese Funktionen und das komplette Handling des Spiels übersichtlich zu gestalten, besitzt der GameManager je eine Instanz von weiteren Managern, welche wiederum Aufgaben übernehmen. Diese einzelnen Klassen werden nachfolgend genauer erläutert.\\\\
Der \textbf{GameLobbyManager} dient zur Verwaltung der Game-Lobby. Er besitzt die Möglichkeit den Eröffnungsscreen, die Einstellungsscreens, die Erfolgsmeldung sowie auch den Screen für das Spielende und den Neustart anzuzeigen und die getätigten Benutzereingaben zu verwalten. Einige Eingaben führen zu weiteren Screens, andere lösen über den GameSyncManager eine Nachricht an den Peer aus.\\\\
Der \textbf{GameSyncManager} hält die beiden Peers synchron. Vom GameLobbyManager oder vom GameManager aus, können über ihn Nachrichten an den anderen Peer gesendet werden. Bei einkommenden Nachrichten entscheidet der GameSyncManager, welche Events dadurch ausgelöst werden. Wie die genaue Synchronisation der Spielobjekte funktioniert, ist im späteren Kapitel «Gameplay Synchronisation» beschrieben.\\\\
Der \textbf{SceneManager} ist die zentrale Steuereinheit der Game-Scene, alle Änderungen an der Scene geschehen darüber. Das Laden der 3D-Objekte, welches ebenfalls vom SceneManager übernommen wird, ist im nachfolgenden Kapitel «3D-Objekte laden» genauer erklärt.\\\\
Der \textbf{AudioManager} ist befähigt Audio-Dateien abzuspielen.\\\\
Über den \textbf{ShoppingListManager} wird die Einkaufsliste zufällig generiert. Dabei werden zwischen drei und fünf (BemSev: Vielleicht wissenschaftlich belegen, dass man sich so viele Objekte gut merken kann?) Verkaufsobjekte aus den möglichen Objekten ausgewählt, in einer Map gespeichert und über den SceneManager dem Benutzer angezeigt.\\\\
Der \textbf{GameStateManager} überprüft, ob das Ziel des Spiels bereits erreicht wurde. Dabei wird die Map mit der Einkaufsliste mit den Objekten im virtuellen Einkaufskorb (Basket) abgeglichen. Dies erfolgt jeweils beim Hinzufügen eines Objekts in den Einkaufskorb. Im Erfolgsfall, wird über den gameSyncManager eine Nachricht an den Peer verschickt und über den EventDispatcher eine lokale Nachricht versendet. \\\\
Der \textbf{InteractionManager} sorgt dafür, dass Benutzerinteraktionen in der 3D-Welt zu einer Aktion führen. Das genaue Handling dieser Interaktionen wird im Kapitel «Interaktionen» beschrieben.\\\\
(BemSev: Sequenzdiagramme aus dem Anhang erwähnen)
	
	\subsection{3D-Objekte laden}
	Nach Möglichkeit sollte bei Webanwendungen die Datenmenge, welche über den Flaschenhals Netzwerk gesendet werden muss, möglichst geringgehalten werden, um lange Ladezeiten zu verhindern. Da 3D-Objekte bereits eher grosse Dateien sind, ist es gerade bei deren Verwendung wichtig, einige Punkte zu beachten. Nachfolgend wurde dokumentiert welche Optimierungen in Palim-Palim implementiert sind.
Palim-Palim lädt 3D-Objekte, welche in allen Spielmodi verwendet werden (Tresen, Einkaufskorb), während sich die Spielenden noch in der Gamelobby befinden, weswegen für diese keine Wartezeit entsteht. Die vom Spielmodi abhängigen 3D-Objekte (Einkaufsgegenstände) werden geladen, während die betagte Person die Anleitung liest und das Kind das Spiel erklärt bekommt.\\\\
Für die 3D-Objekte verwendet Palim-Palim Dateien des Graphics Language Transmission Format (glTF). Dieses Dateiformat kann 3D-Modelle sehr effizient übertragen und laden, weswegen es sich sehr gut für Webapplikationen eignet [https://www.khronos.org/gltf/]. In einer glTF-Datei ist nicht nur die Geometrie eines 3D-Objekts gespeichert, sondern Szenen, Kameras, Materialien, Texturen und auch Animationen (nicht abschliessende Liste) [https://github.com/KhronosGroup/glTF/blob/master/specification/2.0/figures/gltfOverview-2.0.0b.png]. Deswegen muss in Palim-Palim beim Laden der 3D-Objekte zuerst die Szene der glTF Datei ausgelesen und danach in dieser nach Meshes (Geometrie eines 3D-Objekts) gesucht werden.\\\\
\begin{lstlisting}
const loadedData = await loader.loadAsync(this.config.models[i].path);
loadedData.scene.traverse((o) => {
    if (o.isMesh) {
	 ...
    }
});
\end{lstlisting}
Die in Palim-Palim verwendeten 3D-Objekte wurden mit Blender [https://www.blender.org/] auf eine möglichst kleine Dateigrösse gebracht, um die Ladezeit, zusätzlich zum optimalen Dateiformat, zu optimieren. Dazu wurde der «Decimate Modifier» angewendet, welcher es erlaubt, die Polygone einer Geometrie zu reduzieren und dabei den ursprünglichen Körper nur minimal zu verändern [https://docs.blender.org/manual/en/latest/modeling/modifiers/generate/decimate.html]. Für die bessere Verwaltung im Programm wurden in Blender zusätzlich die zwei folgenden Modifikationen an den ursprünglichen glTF-Dateien durchgeführt. Einerseits wurden Objekte, die aus mehreren Meshes bestanden, zu einem Mesh zusammengefügt und anderseits wurde der Mittelpunkt des Meshs auf den Koordinatenursprung gelegt. Die Skalierung des Meshs hingegen kann über einen Parameter in der Konfigurationsdatei von Palim-Palim festgelegt werden, somit können die einzelnen Objekte aufeinander abgestimmt werden.
	
	\subsection{Interaktionen}
Ein Objekt auf einem Screen so zu steuern, dass dieses in einem 3D-Raum bewegt wird, ist nicht intuitiv. Deswegen gibt es verschiedene Lösungsmöglichkeiten, wie dies bewerkstelligt werden kann. Beispielsweise kann das Gyroskop in die Steuerung miteinbezogen werden. Dabei könnte die Dimension in die Tiefe nur angesteuert werden, wenn das Tablet flach liegt (Winkel zur Horizontalen < 20 Grad), die vertikale Dimension nur, wenn das Tablet «steht» (Winkel zur Horizontalen > 20 Grad). Dies ist für Palim-Palim allerdings nicht geeignet, da so die Videoaufnahme sehr unruhig wird. Eine weitere Möglichkeit ist es, die Ansteuerung der dritten Dimension in die Tiefe per Ziehen vom Mittelpunkt zu einer äusseren Ecke zu bewerkstelligen. Das Ziehen von der Mitte zur rechten oberen Ecke würde das Element nach hinten bewegen und das Ziehen zur rechten unteren Ecke würde das Element nach vorne bewegen, dies haben beispielsweise Tseng et al. (EZ Manipulator Designing a mobile fast and ambiqui.pdf // todo severin pdf zotero) in ihrem Forschungsartikel so beschrieben. Die Zielgruppe von Palim-Palim soll nicht mit neuen Gesten überfordert werden, weswegen diese Technologie ebenfalls ungeeignet ist.\\\\
Palim-Palim setzt auf eine simple 2D-Steuerung, welche sich jedoch auf einer gekippten Ebene abspielt. Diese Ebene, genannt Interaktionsebene, ist vom Bildschirm aus gesehen um 45 Grad nach hinten gekippt. Das gelbe Linienkonstrukt auf der Abbildung abc zeigt die Interaktionsebene des Verkaufspersonals. Die Interaktionsebene des Kaufenden ist an der xy-Ebene in der Mitte des Tresens gespiegelt. Wird nun ein Objekt angetippt, wird die Methode onPointerDown des InteractionManager aufgerufen. Anhand eines Raycasts (BemSev: Quelle für Beschreibung?) wird das Element ausgewählt, welches am nächsten an der Kamera des Interagierenden ist. Dieses wird «aufgenommen» und kann nun in der Interaktionsebene bewegt werden und gewinnt oder verliert abhängig von der Höhe an Tiefe.\\\\
Wie im Kapitel «Gamedesign» beschrieben, benötigen die Verkaufsobjekte für eine einfachere Bedienung eine grössere Touch-Fläche. Dies ist mit einer einfachen Hitbox in Form einer Kugel realisiert. Die Kugel ist eine möglichst kleine Begrenzungskugel um den ebenfalls möglichst kleinen Begrenzungsquader um die Geometrie des Objekts. Three.js stellt dafür die Methoden getBoundingSphere und getBoundingBox zur Verfügung. Für den Einkaufskorb ist ebenfalls eine Hitbox vorhanden (roter Linienquader auf Abb. abc).\\\\
Um alle Verkaufsobjekte herum ist ebenfalls eine Box (blauer Linienquader auf Abb. abc) vorhanden. Werden Objekte innerhalb diese Bereichs gezogen, werden diese wieder an ihre Ursprungsposition gesetzt.
	

		
	\newpage
	\subsection{Gampeplay-Synchronisation}
	Um die Interaktionen der beiden Spieler zu synchronisieren, wird normalerweise ein Server verwendet, der die Inputs der Spieler entgegennimmt, und als zentrale Authorität den Zustand des Spiels bestimmt. Bei Palim-Palim wurde dies aber anders gelöst. Das Spiel nutzt seine Peer-To-Peer-Funktionalität aus und kreiert neben dem Video- und Audiostream auch einen spezifischen DataChannel für Game-Updates. Dieser Channel wird genutzt, um die Objekte beider Szenen zu synchronisieren.
	

	
Der DataChannel überträgt JSON-Strings via UDP. UDP als Protokoll ist sehr schnell und deshalb sind die Änderungen auch fast ohne Verzögerung beim Peer sichtbar. Die Wahl des Protokolls kann beim Erstellen des DataChannels gewählt werden [Code Erstllung DataChannel]. Eine weitere Eigenschaft von UDP ist, dass die Reihenfolge der Übertragung nicht garantiert ist - im Gegensatz zu TCP z.B. Da die Game-Updates sehr oft geschickt werden, ist es in diesem Fall egal, in welcher Reihenfolge die Nachrichten ankommen. (Man muss sich das für jeden Fall überlegen...)
\begin{lstlisting}
const dataChannel = peerConnection.createDataChannel('gameUpdates', {
  ordered: false,
  id: room
  });
dataChannel.onmessage = handleReceiveMessage;
dataChannel.onerror = handleError;
dataChannel.onopen = handleDataChannelStatusChange;
dataChannel.onclose = handleDataChannelStatusChange;
\end{lstlisting}

Diese Art der Gameplay-Synchronisation erlaubt es den Clients, total unabhängig von einem Server zu spielen. Da der Server nur fürs Signaling benutzt wird, ist auch eine grosse Skalierung der Spieleranzahl denkbar. Der Server muss die Spieler nur inital vermitteln, was keine grosse Sache ist. Alle anderen Berechnungen sowie der Abgleich der Spielwelt werden auf den Clients direkt vorgenommen. Da kein Umweg über einen Server genommen werden muss, ist auch die Latenz niedrig. (Einziges Manko: es gibt keine zentrale Authorität. Das heisst Spieler könnten den Client-Code manipulieren und sich unfaire Vorteile verschaffen. Da Palim-Palim und sein Publikum aber nicht kompetitiv sind, ist diese Gefahr des Cheatings vernachlässigbar.)
		
			
			
\chapter{Fazit}
	\begin{itemize}
		\item Zusammenfassung des Erreichten / Zielerreichung
		\item Zentrale Erkenntnisse
		\item Reflektion
		\item Mögliche Weiterentwicklungen (bezogen auf die Software)
		\item Weiterführende Forschung
	\end{itemize}

	\subsection{Physik als mögliche Weiterentwicklung}
	\emph{
		\color{blue}
	\begin{itemize}
		\item Versuche mit ammojs und headless Möglichkeiten auf Server 
	\end{itemize}}

\chapter{Literaturverzeichnis}
\printbibliography[heading=none]

\appendix

\chapter{Ehrlichkeitserklärung}

\chapter{Testprotokolle, weitere Spielkonzepte/Ideenfindung, Readme des Repos... }


\end{document}