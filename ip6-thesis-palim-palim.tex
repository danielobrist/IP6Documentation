\documentclass[12pt,a4paper]{report}

\begin{document}


\begin{titlepage}
\paragraph{Titelblatt}
\end{titlepage}

\begin{abstract}	
	\begin{itemize}
 		\item Ergebnisse aus den Spieletests, der Entwicklung und der Literaturrechereche zusammengefasst erläutern.
 		\item Erst am Schluss schreiben
 		\item Wichtig für den ersten Eindruck
	\end{itemize}
\end{abstract}


\tableofcontents

\break

\chapter{Einleitung}
\paragraph{Teil 1}
\begin{itemize}
 \item Beschreibung des Videospiels Palim-Palim (inkl. Screenshot)
 \item Aufstellung der Forschungsfragen und den Erkenntnissen
\end{itemize}

\paragraph{Teil 2}
\begin{itemize}
 \item Ausgangslage inkl. Forschungsstand
 \item Relevanz der Problemstellung
\end{itemize}

\paragraph{Teil 3}
\begin{itemize}
 \item Grobe Beschreibung der angewendeten Methodik
 \begin{itemize}
 	\item Spielentwicklung (Architektur, Technologien)
 	\item Spieletests (Methoden)
 \end{itemize}
\end{itemize}
 
\paragraph{Teil 4}
\begin{itemize}
	\item Aufbau des Dokuments und Überleitung in den theoretischen Teil
\end{itemize}

\chapter{Theoretischer Teil}
\begin{itemize}
	\item Beschreibung des Umfelds / Anwendungsdomäne (betagte Personen und Kinder)
	\item Intergenerationelles Spielen
	\begin{itemize}
		\item Forschungsstand (Welche Methoden/Ansätze werden angewendet?)
		\item Bisherige Erkenntnisse zu intergenerationellem Spielen
	\end{itemize}
	\item Videochats in Videospielen
	\begin{itemize}
		\item Forschungsstand (Welche Methoden/Ansätze werden angewendet?)
		\item Bisherige Erkenntnisse zu Videochats in Videospielen
	\end{itemize}
	\item Lücken der bisherigen Forschung
	\item Aufstellung der Forschungsfragen
	\item In Palim-Palim verwendete Methoden
	\begin{itemize}
		\item Methode der Spieletest
		\item Methode der Spieletest-Auswertung
	\end{itemize}
\end{itemize}

\chapter{Praktischer Teil}
	\section{Videospiel Palim-Palim}
		\subsection{Gamedesign}
			\begin{itemize}
				\item Auswahl Spielkonzept
				\item Spielkonzept
				\item Gameplay-Loops
				\item Spielvarianten
				\item Design und Usability
			\end{itemize}
		\subsection{Implementation}
			\begin{itemize}
				\item Technologien
				\item Architektur
				\item Sicherheit
				\item Testing
				\item Deployment
				\item Betrieb
			\end{itemize}
		\section{Durchführung Spieletests}
			\begin{itemize}
				\item Organisation der Spieletests (Aufbau, Ablauf, Testpersonen, Testszenarien, Testumgebung)
				\item Beobachtungen
				\item Ergebnisse
				\item Beantwortung der aufgestellten Forschungsfragen
			\end{itemize}
			
			
\chapter{Schlussfolgerung}
	\begin{itemize}
		\item Zusammenfassung des Erreichten / Zielerreichung
		\item Zentrale Erkenntnisse
		\item Reflektion
		\item Mögliche Weiterentwicklungen (bezogen auf die Software)
		\item Weiterführende Forschung
	\end{itemize}


\chapter{Literaturverzeichnis}

\appendix

\chapter{Ehrlichkeitserklärung}

\chapter{Testprotokolle, weitere Spielkonzepte... }


\end{document}