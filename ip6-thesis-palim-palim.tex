\documentclass[12pt,a4paper]{report}

% -- Imports biblatex and defines bib file --
\usepackage[backend=bibtex,style=numeric,language=german,sorting=none]{biblatex}
\addbibresource{references.bib}
% http://tug.ctan.org/info/biblatex-cheatsheet/biblatex-cheatsheet.pdf

% -- Language --
\usepackage[utf8]{inputenc}
\usepackage[ngerman]{babel}

% -- Images --
\usepackage{graphicx}
\graphicspath{ {./images/} }

% -- Continuous figure numbering --
\usepackage{chngcntr}
\counterwithout{figure}{chapter}

% -- Code blocks --
\usepackage{listings}
\usepackage{color}
\definecolor{lightgray}{rgb}{.9,.9,.9}	
\definecolor{darkgray}{rgb}{.4,.4,.4}
\definecolor{purple}{rgb}{0.65, 0.12, 0.82}

% -- Code blocks Stlye --
\renewcommand\lstlistingname{Codefragment}
\renewcommand\lstlistlistingname{Codefragmente}

\lstdefinelanguage{JavaScript}{
  keywords={typeof, new, true, false, catch, function, return, null, catch, switch, var, if, in, while, do, else, case, break},
  keywordstyle=\color{blue}\bfseries,
  ndkeywords={class, export, boolean, throw, implements, import, this},
  ndkeywordstyle=\color{darkgray}\bfseries,
  identifierstyle=\color{black},
  sensitive=false,
  comment=[l]{//},
  morecomment=[s]{/*}{*/},
  commentstyle=\color{purple}\ttfamily,
  stringstyle=\color{red}\ttfamily,
  morestring=[b]',
  morestring=[b]"
}
\lstset{
   language=JavaScript,
   backgroundcolor=\color{lightgray},
   extendedchars=true,
   basicstyle=\footnotesize\ttfamily,
   showstringspaces=false,
   showspaces=false,
   numbers=left,
   numberstyle=\footnotesize,
   numbersep=9pt,
   tabsize=2,
   breaklines=true,
   showtabs=false,
   captionpos=b
}

\newcommand{\paragraphwithnewline}[1]{\paragraph{#1}\mbox{}\\}

\begin{document}

\begin{titlepage}
    \begin{center}
        \vspace*{1cm}
            
        \Huge
        \textbf{Palim-Palim}
            
        \vspace{0.5cm}
        \Large
        Ein Videospiel mit integriertem Videochat für Kinder und betagte Menschen 
            
        \vspace{1.5cm}
            
        \textbf{Daniel Obrist}\\
        \textbf{Severin Peyer}

        \vfill
            
        Bachelorthesis
        \vspace{0.8cm}

        \includegraphics[width=0.65\textwidth]{fhnw_ht_10mm}

        \vspace{0.8cm}
            
            
        \Large
        Studiengang Informatik\\
        Profilierung iCompetence\\

        \vspace{0.8cm}

		Betreuende: Marco Soldati, Tabea Iseli\\
		Im Auftrag der Fachhochschule Nordwestschweiz
		
		\vspace{0.8cm}
        
        Brugg, 20.08.2021\\
        
        \vspace{0.8cm}

    \end{center}
\end{titlepage}

\chapter*{Projektinformationen}
Titel: Palim-Palim\\
Projektnummer: 21FS\_I4DS08

\paragraphwithnewline{Projekt-Team}
Daniel Obrist, 8iCbb\\
daniel.obrist@students.fhnw.ch\\\\
Severin Peyer, 8iCbb\\
severin.peyer@students.fhnw.ch

\paragraphwithnewline{Auftraggeber und Betreuung FHNW}
Marco Soldati\\
Fachhochschule Nordwestschweiz FHNW\\
Hochschule für Technik\\
Bahnhofstrasse 6\\
CH-5210 Windisch\\
+41 56 202 77 31\\
marco.soldati@fhnw.ch\\\\
Tabea Iseli\\
Fachhochschule Nordwestschweiz FHNW\\
Hochschule für Technik\\
Bahnhofstrasse 6\\
CH-5210 Windisch\\
+41 56 202 86 53\\
tabea.iseli@fhnw.ch\\

\paragraphwithnewline{Experte}
Jonas Weibel\\
joweibel@microsoft.com

\paragraphwithnewline{Zeitbudget}
Das Projekt wird im Rahmen des Frühlingssemesters 2021 durchgeführt. Nominell sind für das Projekt 360~Arbeitsstunden pro Teammitglied veranschlagt. 

\paragraphwithnewline{Wichtigste Daten}
Beginn:	22. Februar 2021\\ 
Ende:	20. August 2021 \\

\begin{abstract}	
	\begin{itemize}
 		\item Ergebnisse aus den Spieletests, der Entwicklung und der Literaturrechereche zusammengefasst erläutern.
 		\item Erst am Schluss schreiben!
 		\item Wichtig für den ersten Eindruck
	\end{itemize}
\end{abstract}


\tableofcontents

\break

\chapter{Einleitung}
\paragraph{Teil 1 / Was wurde erreicht?}
\emph{\begin{itemize}
\color{blue}
 \item Beschreibung des Videospiels Palim-Palim (inkl. Screenshot)
 \item Aufstellung der Forschungsfragen und den Erkenntnissen
\end{itemize}}

\paragraph{Teil 2 / Warum wurde es gemacht?}
\emph{\begin{itemize}
\color{blue}
 \item Ausgangslage inkl. Forschungsstand
 \item Relevanz der Problemstellung
 \item Was ist das Umfeld?
\end{itemize}}

\paragraph{Teil 3 / Wie wurde es gemacht?}
\emph{\begin{itemize}
\color{blue}
 \item Grobe Beschreibung der angewendeten Methodik
 \begin{itemize}
 	\item Spielentwicklung (Architektur, Technologien)
 	\item Spieletests (Methoden)
 \end{itemize}
\end{itemize}}
 
\paragraph{Teil 4}
\emph{\begin{itemize}
\color{blue}
	\item Aufbau des Dokuments und Überleitung in den theoretischen Teil
\end{itemize}}
Palim-Palim ist ein interaktives Multiplayer-Videospiel für Kinder und betagte Menschen. Zwei Spielende können damit Kinder-Kaufladen in einer virtuellen Umgebung spielen. Es verfügt über einen Video-Chat als inhärente Game-Mechanik, um die Kommunikation zwischen den Spielerinnen und Spielern zu fördern.\\\\
\begin{figure}
\includegraphics[width=\textwidth]{PalimPalim_screens}
\caption[Caption for LOF]{Das Videospiel Palim-Palim. Perspektive der betagten Person (links) und des Kindes (rechts) (eigene Darstellung).}
\end{figure}
Mit dem Spiel wird untersucht, welchen Einfluss ein Video-Chat in Videospielen auf die User Experience und die Kommunikation zwischen den Spielenden hat. Im Speziellen werden die folgenden Fragestellungen untersucht:
\begin{itemize}
	\item Welche Wirkung hat ein integrierter Video-Chat auf das Spielerlebnis von betagten Menschen?
	\item Welche Wirkung hat ein integrierter Video-Chat auf das Spielerlebnis von Kindern? 
	\item Welche Auswirkungen hat die Art und Weise, wie der Video-Chat in das Spiel integriert ist, auf das Spielerlebnis? 
	\item Welche Wirkung hat ein integrierter Video-Chat auf die Kommunikation zwischen den Spielenden?
	\item Wird ein integrierter Video-Chat aktiv als Kommunikationsmittel zur Bewältigung von Spielaufgaben genutzt?
	\item Welche Auswirkungen hat die Art und Weise, wie der Video-Chat in das Spiel integriert ist, auf die Förderung der Kommunikation?
\end{itemize}
Erste Spieletests und ergänzende Literaturrecherchen haben gezeigt, dass Spiele mit Video-Chats ein grosses Potenzial haben, Beziehungen zwischen Generationen auch über weite Distanzen zu fördern. Durch die spielerische Komponente verlieren Kinder weniger schnell das Interesse an der Unterhaltung. Ältere Leute schätzen dabei die sozialen Aspekte des Spielens sehr. Für Kinder hingegen sollte der Fokus eher auf ein spannendes Gameplay gelegt werden, wobei lustige Interaktionen mit dem Video des Gegenübers sich als spannende Weiterentwicklung anbieten.\\\\

Während der Entwicklung von Palim-Palim haben sich zudem diverse Eigenheiten herauskristallisiert, welche bei der Integration von Video-Chats in ein Videospiel eine Rolle spielen. Diese Erkenntnisse wurde festgehalten, um die Entwicklung eines zukünftigen Frameworks für Video-Chat-Spiele zu erleichtern.\\\\

Video-Chats werden heutzutage schon von vielen Familien benutzt, um mit Ihren Verwandten zu kommunizieren. Auch in der Kommunikation mit den Grosseltern wird dabei immer mehr auf Video-Telefonie gesetzt. Dies gilt besonders für Zeiten, in denen Besuche im Alters- oder Pflegeheim auf Grund kursierenden Viren wie Corona schwierig oder unmöglich werden. Betagte Menschen schätzen und nutzen diese moderne Art der Kommunikation mit der Familie auch immer mehr – besonders weil es öfters auf ältere Benutzergruppen angepasste Angebote gibt \cite{glaab_silver_2015}.\\\\
			Allerdings hat die Video-Telefonie immer noch Grenzen, welche die Interaktionen unnatürlich und teilweise entfremdend wirken lassen. Vor allem für Kinder ist es schwierig, Gesprächsthemen und Kommunikationswege zu finden, die sich so lustig und verbindend anfühlen, wie die Zeit mit der Grossmutter oder dem Grossvater im echten Leben. Oft sind Kinder vom Gespräch schnell gelangweilt – sie würden lieber etwas spielen \cite{tulloch_7_2020}. Spielen kann ein Mittel sein, um Kinder besser einzubeziehen und die Interaktion mit ihnen zu unterstützen, wie bisherige Studien zu dem Thema Video-Calls mit Eltern und Kindern zeigen \cite{follmer_video_2010}.\\\\
			Ergänzend zum Video-Telefonie-Aspekt gibt es bereits viel Literatur bezüglich generationenübergreifender Computerspiele \cite{chua_lets_2013}, \cite{de_la_hera_benefits_2017}, \cite{soldati_create_2020}. Erkenntnisse aus einer Studie von Derboven et al \cite{derboven_designing_2012} deuten ausserdem drauf hin, dass in einem Multiplayer-Videospiel die zusätzliche Kommunikationsfunktionalität durch einen Video-Chat oft sowohl von älteren als auch von jüngeren Personen begrüsst wird. Allerdings bietet die Forschung bisher keine detaillierte Studie mit Kindern und betagten Personen in diesem Zusammenhang.\\\\
			Am Institut für Data Science (I4DS) der Fachhochschule Nordwestschweiz wird im Rahmen des Projekts Myosotis schon seit einigen Jahren daran gearbeitet, Video-Spiele zu entwickeln, welche in unterhaltsamer Weise die soziale Interaktion zwischen betagten Menschen und ihren Angehörigen unterstützen \cite{soldati_fhnw_2015}. Dabei wurden schon etliche Spiele umgesetzt und getestet \cite{soldati_create_2020}. Mit einer Integration eines Video-Chats in ein Spiel hat sich jedoch bisher noch kein Team explizit auseinandergesetzt. Palim-Palim schliesst diese Lücke und zeigt wertvolle Erkenntnisse über die Kombination von Video-Chats und Video-Spielen mit Kindern und betagten Personen.\\\\
			Um die formulierten Fragen zu beantworten, wurden mehrere Varianten eines Videospiels implementiert, in welchen ein Video-Stream zu einem Teil des Spiels wird. Anschliessend wurden mit allen Varianten Spieltests durchgeführt, um herauszufinden, ob die Integration eines Video-Chats in einem Video-Spiel einen positiven oder negativen Einfluss auf die User Experience sowie die Kommunikation zwischen den Spielenden hat. Die Arbeit fokussiert sich speziell auch auf die Art und Weise, wie ein Video-Stream in ein Spiel integriert werden kann.

\paragraph{\emph{\color{blue} Grobe Systemarchitektur, verwendete Methoden und Konzepte}}

 

% -- Theoretischer Teil --
\chapter{Umfeldanalyse und Zielgruppe}
\emph{
	\begin{itemize}
		\color{blue}
		\item Beschreibung des Umfelds / Anwendungsdomäne (betagte Personen und Kinder)
		\item Persona?
	\end{itemize}
}
Die verwendeten Begriffe Kind und betagte Person werden dabei im Rahmen des Projekts wie folgt definiert:\\
\subparagraph{Kind} Person zwischen fünf und acht Jahren. 
\subparagraph{Betagte Person} Person ab einem Alter von 65 Jahren ohne grössere mentale Beeinträchtigung. 
Diese beiden Personengruppen stellen zugleich die Zielgruppen für die durchzuführenden Spieltests dar.


Palim-Palim wird im Rahmen des Forschungsprojekts «Myosotis» entwickelt. Die Videospiele von Myosotis sollen Angehörigen, insbesondere auch Kindern, dabei helfen, mit ihren betagten Angehörigen Zeit zu verbringen [https://myosotis.i4ds.net/computer-spiele-fuer-menschen-in-altersheimen/]. Die Spiele sollen auch helfen, durch positive soziale Interaktionen das Wohlbefinden betagter Menschen zu verbessern \cite{soldati_create_2020}.\\\\
Das zusammen Zeit verbringen ist in Zeiten von Pandemien, wie beispielsweise der Covid19 Pandemie, oder auch für weit entfernt lebende Verwandte gar nicht so leicht. Immer öfters wird deswegen auf die Videotelefonie zurückgegriffen. In einem Seniorenzentrum in Bielefeld-Senne beispielsweise wird schon seit Juli 2013 auf innovative Technologien gesetzt. Die Videotelefonie wurde dabei von den Bewohnenden sehr schnell rege genutzt, um ihre sozialen Kontakte aufrecht zu erhalten [Michels-Rieß, B., Johnigk, U. Altenhilfe integriert smarte Technik. Heilberufe 69, 44–45 (2017)].\\\\

\chapter{Intergenerationelles Spielen}
Um zwischen zwei Generationen eine Brücke zu schlagen, bieten sich Videospiele als eine vielversprechende Möglichkeit an. Denn besonders im familiären Umfeld fördert gemeinsames Spielen die Beziehungen untereinander und ruft positive Emotionen bei älteren sowohl als auch jüngeren Generationen hervor \cite{osmanovic_beyond_2016}. Es hat sich gezeigt, dass Personen eine grössere Zuneigung zu ihrem Spielpartner aus einer anderen Altersgruppe entwickeln und sich im Umgang mit der anderen Altersgruppe wohler fühlen, wenn sie regelmässig zusammen spielen \cite{chua_lets_2013}. Eine Literaturstudie von de la Hera et al \cite{de_la_hera_benefits_2017} fasst diverse Erkenntnisse zu dem Thema sehr gut zusammen. Sie identifiziert folgende vier Vorzüge des intergenerationellen Spielens: (1) eine Stärkung der familiären Bindung, (2) eine Förderung des wechselseitigen Lernens, (3) ein grösseres Verständnis für die andere Generation und (4) eine Verringerung sozialer Ängste. Unter Berücksichtigung dieser Faktoren lässt sich argumentieren, dass Spielen eine wirkungsvolle Methode ist, um jüngere und ältere Menschen auf einem emotionalen, aber auch sozialen Level zu verbinden.\\\\
Durch den gegebenen Altersunterschied zwischen jungen und älteren Spielenden ergibt sich eine besondere Herausforderung. In der Gestaltung von intergenerationellen Spielen müssen stets zwei sehr unterschiedliche Zielgruppen berücksichtigt werden. Denn nicht nur die Werte und Normen sind teilweise zwischen Generationen grundlegend anders. Auch biologische Einflüsse des Älterwerdens sind bei jüngeren Spielenden noch weniger ausgeprägt. Dazu kommen ein unterschiedliches technisches Verständnis sowie divergierende Vorkenntnisse. Personen ab 50 Jahren bevorzugen zum Beispiel Gelegenheitsspiele gegenüber komplexeren und ausdauernden Spielen \cite{schultheiss_entertainment_2012}. Auch Gajadhar et al. \cite{gajadhar_out_2010} kamen zum Schluss, dass Spiele für Senioren eher wenig Fokus auf den sozialen Wettbewerb und einen besonderen Fokus auf das kooperative Spiel legen sollten. Diese Ergebnisse wurden von De Schutter \cite{de_schutter_meaningful_2008} bestätigt, der feststellte, dass soziale Interaktion der wichtigste Prädiktor für die Spieldauer bei älteren Erwachsenen ist. Senioren sind also besonders an Spielerfahrungen interessiert, die eine starke soziale Komponente haben. Diese deckt sich auch mit Empfehlungen aktueller Literatur zur Spieleentwicklung \cite{schell_art_2008}.\\\\
Kinder hingegen haben differenziertere Anforderungen an Videospiele. Es gibt in der Videospiel-Industrie grundlegende Empfehlungen für gewisse Altersgruppen, welche stark auf der mentalen Entwicklung beruhen. Ganz kleine Kinder bis zu 3 Jahren sind sehr an Spielzeug interessiert, aber die Komplexität und Problemlösung, die mit Spielen verbunden ist, ist im Allgemeinen noch zu hoch für sie \cite{schell_art_2008}. Ab dem vierten Lebensjahr beginnen Kinder dann ihr erstes Interesse an Spielen zu zeigen. Spiele für 4- bis 6-Jährige sind oft sehr einfach und werden häufiger mit den Eltern gespielt als mit Gleichaltrigen. Denn die Eltern wissen, wie sie die Regeln so gestalten können, dass das damit die Spiele Spass machen und interessant sind \cite{schell_art_2008}. Ab 7 Jahren können Kinder dann in der Regel lesen, Dinge selbst herausfinden und auch schwierigere Probleme lösen. Sie beginnen selbst zu entscheiden, welche Art von Spielzeug und Spielen sie mögen und welche nicht \cite{schell_art_2008}. Im späteren Verlauf des Lebens kristallisieren sich dann weitere Vorlieben heraus, welche für die Zielgruppen von Palim-Palim allerdings nicht von grosser Bedeutung sind. Es lässt sich allerdings ein klarer Unterschied zu den Vorlieben und Bedürfnissen älterer Leute feststellen.\\\\
Diese beiden abweichenden Anforderungen von betagten Personen und Kindern an das Game Design stellen eine besondere Herausforderung dar. Um dem entgegenzuwirken, berücksichtigt Palim-Palim beide Benutzergruppen durch eine asymmetrische Gestaltung der Gameplay-Loops. Das heisst, beide Spielenden erhalten unterschiedliche Rollen und Aufgaben im Spiel. So begegnen sich Kinder und betagte Personen als Verkäufer und Kunde in einem Kaufladen, mit zwei unterschiedlichen Sichten auf eine gemeinsame Spielwelt. Dies wurde bei der Erarbeitung des Spielkonzepts bewusst so angedacht, um das intergenerationelle Spielen für beide Seiten möglichst flexibel gestaltbar zu halten. So wird beispielsweise aus der Perspektive des Kindes bewusst auf Texte verzichtet, da es wahrscheinlich noch nicht lesen kann. Die konkrete Gestaltung dieses asymmetrischen Gameplays wird im Kapitel «Design und Usability» noch genauer ausformuliert.

\chapter{Videochats in Videospielen}
Videochats sind ein wichtiges Kommunikationsmittel um zwei oder mehr Menschen über weite Distanzen miteinander zu verbinden. Auch betagte Menschen schätzen und nutzen diese moderne Art der Kommunikation mit der Familie auch immer mehr – besonders weil es öfters auf ältere Benutzergruppen angepasste Angebote gibt \cite{glaab_silver_2015}.\\\\
Während Videochats im geschäftlichen Umfeld bereits recht gut erforscht sind, gibt es im privaten Bereich noch Lücken, wie auch Batcheller et al \cite{batcheller_testing_2007} in ihrer Arbeit zum Thema «Testing the technology: Playing Games with Video Conferencing» erwähnen. Sie haben in ihrer Studie ebenfalls aufgezeigt, dass Personen, welche das Spiel «Mafia» über eine Videokonferenz spielen, ein ähnliches Mass an Zufriedenheit, Spass und Frustration empfinden wie solche, welche in einer echten Umgebung spielen. Dies deutet auf ein grosses Potenzial hin, um ein Multiplayer-Spiel für die Spieler mit Hilfe eines Videochats reichhaltiger und ansprechender zu gestalten.\\\\
Eine Studie von Veinott et al \cite{veinott_video_1999} hat zudem aufgezeigt, dass Videogespräche eine bessere Basis für die Verständigung zwischen zwei Personen im Geschäftsumfeld schaffen, falls diese etwas untereinander verhandeln müssen. In ähnlicher Weise erfordern Multiplayer-Spiele mit aktiver Kommunikation zwischen den Spielenden oft subtile Hinweise, um Spielentscheidungen zu treffen. Dies deutet darauf hin, dass Videochats einen wesentlichen Einfluss auf das Spielgeschehen haben können.\\\\
Eine Studie von Bos et al \cite{bos_short_2001} hat ausserdem verglichen, wie sich Vertrauen zwischen den Spielern in einem Social Dilemma-Spiel \cite{harrod_social_1983} via Videochat entwickelt. Dabei hat sich gezeigt, dass zwischen Spielenden via Videochat ein vergleichbares Vertrauen entstand, wie in Durchgängen, welche Face-To-Face durchgeführt wurden. Die Studie notiert allerdings auch, dass es verglichen mit den vor Ort durchgeführten Spielen länger dauerte, bis dasselbe Vertrauen aufgebaut wurde.\\\\
Erkenntnisse aus einer Studie von Derboven et al \cite{derboven_designing_2012} deuten ergänzend dazu darauf hin, dass in einem Mehrspieler-Videospiel die zusätzliche Kommunikationsfunktionalität durch einen Videochat oft sowohl von älteren als auch von den jüngeren Personen begrüsst wird. Die Videochat-Funktionalität hatte dem Spiel, welches eine betagte Person jeweils mit einem jüngeren Mitspieler spielte, eine zusätzliche soziale Dimension verleihen. Derboven et al empfehlen daher in ihren abschliessenden Worten sogar, Videochats in intergenerationellen Spielen vermehrt einzusetzen. Allerdings warnen sie auch davor, dass ein Videochat in gewissen Fällen zu unangenehmen Situationen führen kann – besonders wenn die beiden Spieler nicht das Bedürfnis haben, miteinander zu sprechen.\\\\
Es lässt sich also zusammenfassen, dass Videochats in diversen Bereichen einen wesentlichen Einfluss auf die Kommunikation zwischen zwei Parteien haben. Besonders in Videospielen, in welchen der aktive Austausch zwischen den Spielenden Teil des Spielgeschehens ist, sollte dieser Einfluss messbar sein und im Game Design berücksichtigt werden.\\\\

\chapter{Aufstellung der Forschungsfragen}
Wie in den vorhergehenden Kapiteln theoretisch dargelegt wurde, bietet die Kombination von Videospielen mit Video-Telefonie eine grosse Möglichkeit für intergenerationelle Spiele. Allerdings gibt es bisher in der Forschung noch keine detaillierte Studie mit Kindern und betagten Personen in diesem Zusammenhang. Die zentralen Forschungsfragen von Palim-Palim fokussieren deshalb speziell auf den Aspekt der Video-Telefonie in Videospielen. Dieses Kapitel fasst hierzu alle Fragestellungen zusammen, welche mit Palim-Palim untersucht wurden.\\

\section{Hauptfragestellung}\label{hauptfragestellung}
Mit Palim-Palim wurde untersucht, wie Videospiele und Video-Telefonie kombiniert werden können. Zusätzlich wurden für diese Kombinationen die Auswirkungen auf die Interaktionen zwischen betagten Menschen und Kindern erforscht. Deshalb wurde für das Projekt folgende übergeordnete Fragestellung formuliert:
\begin{quote}
\textbf{Wie lässt sich Video-Telefonie mit Video-Spielen kombinieren, damit zwei Personen (Kind und betagte Person) übers Netz miteinander spielen und sich gleichzeitig unterhalten können?}
\end{quote}
			

\section{Einzelfragestellungen im thematischen Zusammenhang}
Um die übergeordnete Aufgabenstellung messbar zu machen, setzt sich Palim-Palim mit spezifischen Fragestellungen zu den Themen User Experience (siehe \ref{fragestellungenUserExperience}) und Kommunikation (siehe \ref{fragestellungenKommunikation}) auseinander. Die dazu formulierten Hypothesen werden dabei durch die Resultate aus den Spieltests in Kapitel \ref{spieletestsUndResultate} verifiziert oder widerlegt.

\subsection{Fragestellungen zur User Experience}\label{fragestellungenUserExperience}
Folgende Einezlfragestellungen befassten sich mit den Auswirkungen des Videochats auf das Spielerlebnis.

\paragraph{1. Wie beeinflusst die Einbindung von Video-Telefonie die User Experience in Videospielen?}
	\subparagraph{Fragestellung 1a:} Welche Wirkung hat ein integrierter Video-Chat auf das Spielerlebnis von betagten Menschen?
	\subparagraph{Hypothese 1a:} Ein integrierter Video-Chat hat eine positive Wirkung auf das Spielerlebnis von betagten Menschen.

	\subparagraph{Fragestellung 1b:} Welche Wirkung hat ein integrierter Video-Chat auf das Spielerlebnis von Kindern?
	\subparagraph{Hypothese 1b:} Ein integrierter Video-Chat hat eine positive Wirkung auf das Spielerlebnis von Kindern.

	\subparagraph{Fragestellung 1c:} Welche Auswirkungen hat die Art und Weise, wie der Video-Chat in das Spiel integriert ist, auf das Spielerlebnis?
	\subparagraph{Hypothese 1c:} Je stärker der Video-Chat ins Gameplay integriert ist, desto positiver ist das Spielerlebnis.


\subsection{Fragestellungen zur Kommunikation}\label{fragestellungenKommunikation}
Folgende Einezlfragestellungen befassten sich mit den Auswirkungen des Videochats auf das Kommunikationsverhalten der Spielenden.

\paragraph{2. Wie beeinflusst die Einbindung von Video-Telefonie die Kommunikation in Videospielen?}
	\subparagraph{Fragestellung 2a:} Welche Wirkung hat ein integrierter Video-Chat auf die Kommunikation zwischen den Spielenden? 
	\subparagraph{Hypothese 2a:} Ein im Spiel integrierter Video-Chat fördert die Kommunikation zwischen den Spielenden. 

	\subparagraph{Fragestellung 2b:} Wird ein integrierter Video-Chat aktiv als Kommunikationsmittel zur Bewältigung von Spielaufgaben genutzt? 
	\subparagraph{Hypothese 2b:} Der Video-Chat wird aktiv als Kommunikationsmittel zur Bewältigung der Spielaufgabe verwendet. 

	\subparagraph{Fragestellung 2c:} Welche Auswirkungen hat die Art und Weise, wie der Video-Chat in das Spiel integriert ist, auf die Förderung der Kommunikation? 
	\subparagraph{Hypothese 2c:} Je stärker der Video-Chat ins Spiel integriert ist, desto angeregter ist der Austausch zwischen den Spielenden. 
	
\chapter{Methoden}
Wie Jesse Schell in «The Art of Game Design» beschreibt, ist die Selbstbeobachtung eine wichtige Methode, um sich schnell einen Überblick zu verschaffen, was im entwickelten Spiel funktioniert und was nicht. Dabei sollte jedoch die Gefahr der Subjektivität nicht unterschätzt werden, denn «was nach eigener Erfahrung wahr ist, mag für andere nicht wahr sein» [direktes Zitat]. Aus diesem Grund ist es nützlich und notwendig, anderen, im Speziellen der Zielgruppe des Spiels, sehr gut zuzuhören. Dafür eignen sich Spieletests. Dabei werden Menschen der Zielgruppe eingeladen, das Spiel zu spielen, während sie beobachtet werden und so festgestellt werden kann, ob das beabsichtigte Spielerlebnis vermittelt wird \cite{schell_art_2008}.\\\\
Palim-Palim setzt auf einen qualitativen statt auf einen quantitativen Testansatz. Einerseits weil die Projektzeit kein vergleichendes Testing mit mindestens 10 Personen \cite{noauthor_how_2009} zuliess, andererseits weil Palim-Palim als ein erster funktionsfähiger Prototyp und nicht als verkaufsfertiges Videospiel verstanden wird. Um Palim-Palim zu testen und die in Kapitel \ref{hauptfragestellung} aufgestellten Hypothesen zu überprüfen wurde ein kontrollierter Feldversuch mit zwei Testpaaren durchgeführt (betagte Menschen: 1 Frau, 1 Mann, Alter: 80+, Kinder: 2 Jungen, 9 und 10 Jahre alt). Die teilnehmenden betagten Menschen wiesen keine grösseren gesundheitlichen Probleme auf. Die Spieletests wurden in drei Phasen gegliedert, in welchen die Videomodi 1 bis 3 (siehe Kapitel \ref{spielvarianten}) in einer zufälligen Reihenfolge gespielt wurden.\\\\
Die Tests wurden an einem für die Teilnehmenden vertrauten Ort durchgeführt, dadurch kann das Spiel in einer ihm zugedachten natürlichen Umgebung stattfinden, was gemäss Jesse Schell ein grosser Vorteil sein kann \cite{schell_art_2008}. Die eine Spielsequenz dauerte rund 19 Minuten, die andere rund 28 Minuten. Die Teilnehmenden erhielten vor den Spieletests eine kurze Erklärung, was Palim-Palim ist und eine Erklärung, wie die Spieletests ablaufen werden [Verweis auf «Information zum Spieletest von Palim-Palim» im Anhang]. Ebenfalls wurden diese über die Freiwilligkeit der Tests und die Möglichkeit, die Tests jederzeit und ohne Konsequenzen abzubrechen, informiert. Alle Teilnehmenden, respektive ihre Erziehungsberechtigten, unterzeichneten auch eine Einverständniserklärung für die Verwendung ihrer Daten im Rahmen von Palim-Palim [Verweis auf die Einverständniserklärungen im Anhang].\\\\
Die Spieletests wurden auf Video aufgezeichnet, um die Möglichkeit der Reproduzierbarkeit von Notizen gewisser Aussagen zu haben und diese in einen Kontext stellen zu können. Die Teilnehmenden wurden nach jedem Videomodus gebeten ihr Spielerlebnis zu beurteilen und einige Fragen zu beantworten. Dabei ging es darum Erkenntnisse zu positiven und negativen Gameplay-Elementen zu gewinnen und herauszufinden, ob Palim-Palim der Testperson zusagt oder nicht. Nach den drei gespielten Runden wurde die Testperson zusätzlich nach möglichen Ideen und Wünschen gefragt. Die Fragen wurden sehr offen gehalten, um diese auf den Teilnehmenden anpassen zu können sowie auf das Gespräch einzugehen. Die Fragebögen und die Transkription der Spieletests sind im Anhang zu finden.\\\\
Für die Auswertung wurden die aufgenommenen Videos zur Hilfe beigezogen. Zentrale Beobachtungen durch die Moderierenden sowie wichtige Aussagen, Ideen und Verbesserungswünsche der Testpersonen sind im Kapitel «Resultate» dokumentiert.

% -- Praktischer Teil --
\chapter{Spieletests und Resultate}\label{spieletestsUndResultate}

In diesem Kapitel werden die Ergebnisse der Spieletests von Palim-Palim präsentiert. Dabei ist sehr wichtig, dass diese Resultate mit besonderer Vorsicht genossen werden. Die sehr kleine Anzahl Testpersonen lässt es leider nicht zu, aussagekräftige Schlüsse aus den Resultaten zu ziehen. Es wurden daher die zentralsten Aussagen und Beobachtungen zusammengefasst und mit Erkenntnisse aus der Literatur ergänzt.\\\\
Als erstes werden die Resultate aus den Spieletests mit den betagen Menschen beleuchtet. Anschliessend wird die Seite der Kinder betrachtet. Nach einigen allgmeienen Erkenntnissen werden dann die gestellten Forschungsfragen beantwortet.

\section{Resultate der Spieletests mit betagten Menschen}
Dieser Absatz beschreibt die Erkenntnisse aus den Spieletests, welche auf der Seite der betagten Menschen erlangt wurden. «Testpersonen» wird synonym für die testenden betagten Personen, verwendet, exklusiv der testenden Kinder.\\\\
Die Spieletests haben gezeigt, dass ältere Personen einiges langsamer lesen als jüngere Menschen, zu dieser Erkenntnis kamen auch Liu et al. \cite{liu_age-related_2017}. Von einer Testperson wurde bemängelt, dass es schwierig ist, leise sprechende Kinder über Palim-Palim zu verstehen.\\\\
Die Verbindung zwischen den zwei Spielenden findet über eine gemeinsame Raumnummer statt. Die Formulierung «Raum» ist für die Testpersonen undeutlich sowie auch unschön und kalt. Bei der Auswahl des Spielmodus verstand die eine betagte Person nicht, dass diese Entscheidung zusammen mit dem Kind getroffen werden kann – dass während der Auswahl gesprochen werden kann.\\\\
Für beide Testpersonen ist unklar, dass die Abbildung mit den Erklärungen noch nicht zum Spiel gehört. Sie versuchten bereits das Einkaufsobjekt zu verschieben und die Einkaufliste mit tippen zu öffnen. Die Anweisung «Frag dein Enkelkind nach Objekten» war für eine Person unklar, denn sie fragte das Kind, was dieses gerne hätte, obwohl sie nach einem Gegenstand ihrer Einkaufsliste fragen müsste. Die Testpersonen waren sich uneinig, ob es sinnvoll ist, die Anleitung vor jedem Spielstart anzuzeigen. Eine Testperson findet dies wichtig, damit klar ist, was gemacht werden soll, die andere Person ist der Meinung, dass es mühsam sei und sie in dieser kurzen Zeit die simplen Funktionalitäten nicht vergessen habe.\\\\
Zu Beginn des Gameloops ist es für die Testpersonen unklar, dass sie nach einem Gegenstand fragen sollen, welcher auf der Einkaufsliste zu finden ist. Stattdessen fragen sie nach einem beliebigen Produkt.\\\\
Der illustrierte und zu 70 Prozent transparente Hintergrund sorgte für Unklarheiten. Einerseits wurde von den Testpersonen bemängelt, dass dieser thematisch zu wenig auf den Spielmodus abgestimmt ist und andererseits ist es schwierig diesen als Hintergrund zu erkennen. Der niedrige Kontrast, welcher durch die hohe Transparenz entsteht, sorgt dafür, dass es für die betagten Personen schwierig ist, die einzelnen Objekte des Hintergrunds auseinanderzuhalten. Dass ältere Personen Mühe mit dem Kontrastsehen haben, bestätigen auch Ijsselsteijn et al. \cite{ijsselsteijn_digital_2007}.\\\\
Zu den Interaktionen konnten die drei folgenden Beobachtungen gemacht werden. Einmal war es unklar, dass ein Objekt mit Ziehen verschoben werden kann, stattdessen wurde versucht zuerst auf das Objekt zu tippen und danach an den gewünschten Zielort. Als zwei gleiche Objekte sehr nah nebeneinander platziert waren, wurden diese als Gruppe wahrgenommen und versucht, diese zusammen an den gewünschten Zielort zu ziehen, was jedoch nicht funktionierte. Des Weiteren hat eine Testperson die Einkaufsliste zuerst mit ziehen statt mit tippen versucht zu öffnen.\\\\
Das mitspielende Kind zu sehen, ist laut den Aussagen der beiden Testpersonen nicht zwingend notwendig und führt daher auch zu keiner veränderten Wirkung des Spielerlebnisses. Beim zweiten Spieletest, bei welchem zuerst die Variante ohne Video gespielt wurde, konnte beobachtet werden, dass es schwierig ist, mit dieser Version zu starten.\\\\
Nach dem Konfettiregen, welcher das Ende des Spiels simuliert, war für eine Testperson unklar, dass das Spiel nun zu Ende ist, respektive wie es nun weitergeht.\\\\
Positive Rückemeldungen erhielt Palim-Palim bezüglich dem Abstreichen des gekauften Gegenstands auf der Liste. Ebenfalls wurde das Verschieben der Objekte als sehr flüssig wahrgenommen. \\\\
Vebesserungsvorschläge der Testpersonen bezogen sich in erster Linie auf mehr Inhalt. Der junge Entwicklungsstand des Spiels wurde dadurch sehr klar. braucht mehr Verkaufsartikel, verschiedene Aufgaben je nach Spielfortschritt und vielleicht sogar andere Szenarien als nur einen Einkaufsladen. Die Testpersonen wünschten sich, selbst entscheiden zu können, was sie einkaufen möchten.\\\\
Es lässt sich zusammenfassen, dass Palim-Palim aus Sicht der Benutzer noch sehr viel Verbesserungspotenzial hat. Allerdings wurden auch die Möglichkeiten eines solchen Spiels erkannt. Um eine Testperson zu zitieren: 
\begin{quote}
\textit{``Also gerade jetzt in der Covid-Zeit, in der man sich nicht gesehen hat, dann wäre das jetzt noch ziemlich gut gewesen, dann hätte man die Oma gesehen oder die Oma hätte den Enkel gesehen und dann hätte man noch ein Spiel machen können. Da würde ich sagen, ja… es hat Potenzial.``}
\end{quote}
	
\section{Resultate der Spieletests mit Kinder}
Beim Beobachten der Kinder ist aufgefallen, dass ihr Spielerlebnis schon zu Beginn durch die einseitige Bedienung der Gamelobby beeinträchtigt wurde. Das Kind kann in Palim-Palim nicht mit der Gamelobby interagieren und sieht ausser dem Video auch keine visuellen Indikatoren darüber, was das Gegenüber gerade auswählt. Kombiniert mit dem Fakt, dass sein betagtes Gegenüber teilweise ein bisschen länger braucht, um sich zurechtzufinden, führt dies schon vor dem Spielbeginn zu Interessensverlust. Die Gamelobby sollte in so einem asymmetrischen Spiel wie Palim-Palim also möglichst für beide Spieler interaktiv sein sollte. Sie müsste so gestaltet werden, dass beide Seiten sehen, was die Gegenseite gerade auswählt. So könnte schon in der Gamelobby ein aktiverer Austausch stattfinden.\\\\
Nach dem Start des Spiels wird in Palim-Palim zuerst ein Screenshot des Spiels mit grafischen Elementen als Erklärung der Spielmechanik gezeigt. Die Kinder versuchten ebenfalls, wie die betagten Personen, auf dieser Abbildung bereits die Objekte zu verschieben. Sie waren dementsprechend verwirrt als nichts passierte. Hieraus lässt sich folgern, dass aus bisherigen Erfahrungen mit Games ein interaktives Tutorial erwartet wird. Ebenfalls kommt hier wieder die einseitige Bedienung ins Spiel: die Kinder wussten nicht, dass nur die betagte Person am anderen Ende die Spielanleitung schliessen kann. Dadurch entstand der Eindruck, dass das Spiel nicht responsiv sei. Dies lässt sich wieder auf die asymmetrische Gestaltung des Spieles zurückführen, welche bei Spielen basierend auf einem Video-Chat natürlich inhärent sind. Es sollte also immer bedacht werden, dass schon vor Spielbeginn eine Synchronisation und wenn möglich auch die Visualisierung aller Interaktionen des Remote-Peers auf jedem Client stattfinden sollten.\\\\
Die Spielmechanik von Palim-Palim selbst wurde von den Kindern sehr schnell begriffen, sobald sich die Spieler in der 3D-Szene befanden. Sie hatten keine Probleme die Gegenstände dem Gegenüber anzubieten. Das Dragging der Objekte funktionierte einwandfrei. Jedoch fehlten Reaktionen des Spieles. Es wurde bemängelt, dass nichts passiert. Dies ist jedoch auf den jungen Entwicklungsstand des Spiels zurückzuführen.\\\\
Das fehlende Video in den Modi 1 und 2 wurde von den Kindern bemerkt. Es wurde sogar mehrmals versucht, mit dem Video der betagten Person zu interagieren. Zum Beispiel wurde versucht die Gegenstände dem Gegenüber zu füttern oder damit das Gesicht lustiger zu gestalten. Dies zeigt, dass ein Video-Stream als Teil der Gameplay-Szene für die Kinder kein fremdes Element ist. Vielmehr bietet sich hier eine grosse Möglichkeit, noch mehr lustige Interaktionen einzubauen.\\\\
Aus diesen Beobachtungen sowie dem qualitativen Feedback der Kinder lässt sich folgern, dass ein Video-Chat durchaus positive Wirkungen auf das Spielerlebnis haben kann. Jedoch muss darauf geachtet werden, dass Kinder hohe Erwartungen an ein Spiel haben. Sie brauchen mehr Inhalte und Interaktionsmöglichkeiten, um die Spannung aufrecht zu erhalten. Es sollte also bewusst mehrere Interaktionsmöglichkeiten geschaffen werden. Insbesondere Interaktionsmöglichkeiten mit dem Video fühlen sich für sie natürlich an und sollten unbedingt in das Gameplay integriert werden.\\\\

	
\section{Beantwortung der Forschungsfragen}
\subsubsection{Fragestellung 1a: Welche Wirkung hat ein integrierter Videochat auf das Spielerlebnis von betagten Menschen?}
\subsubsection{Hypothese 1a: Ein integrierter Videochat hat eine positive Wirkung auf das Spielerlebnis von betagten Menschen.}
Die zwei Testings konnten diese Hypothese nicht belegen. Aber auch zum Widerlegen der Hypothese fehlen entsprechende Daten. Für den einen Probanden braucht es das Video nicht unbedingt:
\begin{quote}
\textit{«Jetzt hast du den Ruben gar nicht gesehen, sein Bild nicht gesehen. Hast du das jetzt schlechter oder besser gefunden? Oder hat dich das gestört?» - «Nein, spielt eigentlich keine Rolle.» - «Also musst du ihn nicht unbedingt sehen?» - «Nicht unbedingt.»}
\end{quote}
Gemäss der anderen Probandin hat der integrierte Videochat ebenfalls keine grosse Auswirkung auf ihr Spielerlebnis. Jedoch findet sie den dritten Modus niedlich, da man das Kind hinter der Theke sieht.\\
Um die Fragestellung 1a jedoch eindeutig beantworten zu können, müssten eine grössere Anzahl User-Tests durchgeführt werden. Ebenfalls sollten dabei unterschiedliche Spiele berücksichtigt werden, um den Einfluss des Spiels selbst auf das Erlebnis auszuschliessen.

\subsubsection{Fragestellung 1b: Welche Wirkung hat ein integrierter Videochat auf das Spielerlebnis von Kindern?}
\subsubsection{Hypothese 1b: Ein integrierter Videochat hat eine positive Wirkung auf das Spielerlebnis von Kindern.}
Die Beobachtungen der Kinder deuten darauf hin, dass der Videochat sich positiv auf Ihr Spielerlebnis ausgewirkt hat. Besonders aufgefallen sind die Interaktionsversuche mit dem Video, als ein Kind die Banane auf dem Video als Nase der betagten Person platziert hat. Dies hat beim Kind zu einigen Lachern geführt.\\
Ebenfalls bewerteten beide Kinder die Durchgänge mit Video höher als diejenigen ohne. Allerdings waren die Bewertungen der Kinder allgemein erstaunlich hoch. Es wird vermutet, dass sie sich nicht getrauten Kritik auszusprechen. Und wegen der geringen Anzahl Tests kann leider auch hier die Hypothese nicht abschliessend bestätigt werden. Es zeichnen sich aber Tendenzen zu Erkenntnissen aus den Studien von Follmer et al. \cite{follmer_video_2010} ab, welche besagt, dass Spielen das Engagement der Kinder in einem Videochat erhöht. Hierzu müssten aber weitere Tests durchgeführt werden, welche den Fokus auf die Messung des Engagements legen.

\subsubsection{Fragestellung 1c: Welche Auswirkungen hat die Art und Weise, wie der Videochat in das Spiel integriert ist, auf das Spielerlebnis?}
\subsubsection{Hypothese 1c: Je stärker der Videochat ins Gameplay integriert ist, desto positiver ist das Spielerlebnis.}
Das Testing zeigte keine unterschiedliche Wahrnehmung der Spielerlebnisse, ob der Videostream in die Spielszene integriert oder unabhängig davon ist. Da Palim-Palim zum Testzeitpunkt über keinen interaktiven Videochat verfügte unterschieden sich gespielten Videomodi aber auch nur geringfügig. Dass Videomodus 2 das Video als simples Overlay präsentierte und Videomodus 3 das Video auf eine Ebene in der 3D-Szene hinter dem virtuellen Tresen projizierte, wurde von den Probanden nicht gross bemerkt oder angesprochen. Es wurde lediglich bemerkt, dass die Person hinter dem Tresen zu stehen scheint. Dies wurde als «niedlich» bezeichnet.\\
Dies lässt darauf schliessen, dass man das Video noch stärker als Teil der Gameplay-Szene gestalten solle, um eine solche Frage beantworten zu können. Ein mögliches Freistellen des Videos, oder sogar das Erweitern eines virtuellen Avatars mit einem Ausschnitt des Videos bieten sich als Möglichkeiten an. Da Palim-Palim diese Funktionalitäten noch nicht besitzt, kann die Fragestellung 1c also auch nicht abschliessend beantwortet werden. 


\subsubsection{Fragestellung 2a: Welche Wirkung hat ein integrierter Videochat auf die Kommunikation zwischen den Spielenden?}
\subsubsection{Hypothese 2a: Ein im Spiel integrierter Videochat fördert die Kommunikation zwischen den Spielenden.}
Beobachtungen der Testpersonen haben gezeigt, dass beim Betreten des Videochats jedes Mal eine initiale Kommunikation in Form von «Hallo hörst du mich?» oder «Bist du bereit?» verwendet wurde. In den Testfällen mit Video haben sich die Probanden zudem gegenseitig zugewinkt. Diese Art der non-verbalen Begrüssung wurde auch schon von Derboven et al \cite{derboven_designing_2012} in ihrer Studie festgestellt. Dieses Verhalten deutet darauf hin, dass ein Videochat bei der Kontaktaufnahme zum Gegenüber eine persönlicheren Eindruck vermitteln kann.\\
Im Spiel selbst konnte aber auch bei dieser Fragestellung kein signifikanter Unterschied zwischen den beiden Modis mit und ohne Video festgestellt werden. Der Video-Chat in Palim-Palim hat die Kommunikation zwischen den Spielern also nicht gefördert. Es muss allerdings erwähnt werden, dass Palim-Palim sehr wenige Spielelemente besitzt, welche eine Kommunikation zwischen den Spielern erfordert. Es wäre angebracht, mehr Kommunikations-fördernde Teile in das Gameplay zu integrieren, bevor man hierzu eine repräsentative Aussage für Videospiele im Allgemeinen machen kann.

\subsubsection{Fragestellung 2b: Wird ein integrierter Videochat aktiv als Kommunikationsmittel zur Bewältigung von Spielaufgaben genutzt?}
\subsubsection{Hypothese 2b: Der Videochat wird aktiv als Kommunikationsmittel zur Bewältigung der Spielaufgabe verwendet.}
Im Hinblick auf die Spielaufgaben wurde der Videochat nicht aktiv verwendet. Allerdings lässt sich dies auch auf die mangelnde Vielfalt an Spielaufgaben zurückführen. Da Palim-Palim zum Testzeitpunk im Hinblick auf das Gameplay nur das Übergeben der Einkaufsgegenstände implementierte, kann diese Frage nicht abschliessend beantwortet werden. Es müssten weitere Spielaufgaben, welche auf verschiedene Arten gelöst werden können, untersucht werden.
\subsubsection{Fragestellung 2c: Welche Auswirkungen hat die Art und Weise, wie der Videochat in das Spiel integriert ist, auf die Förderung der Kommunikation?}
\subsubsection{Hypothese 2c: Je stärker der Videochat ins Spiel integriert ist, desto angeregter ist der Austausch zwischen den Spielenden.}
Zwischen den Videomodis wurde kein Anstieg in dem Austausch zwischen den Spielenden festgestellt. Wie schon bei Forschungsfrage 1c beschrieben, lässt sich hierzu keine Aussage machen, ohne weitere Arten der Video-Integration zu testen.
			
	
\chapter{Gamedesign}

\section{Ideenfindung}
Parallel zur Entstehung erster Ideen für Palim-Palim wurden unzählige weitere Ansätze für intergenerationelle Spiele mit einem Videochat ausgearbeitet. Diese sind im Anhang inklusive einer kurzen Beschreibung aufgelistet. Bei der Entwicklung dieser Spielkonzepte galt es dabei mehrheitlich immer wieder die gleichen Faktoren zu berücksichtigen. Einerseits sollte der Videochat ein zentrales Element des Spiels sein. Die Spielenden sollen mit dem Video-Element interagieren und es als Teil der Spielumgebung wahrnehmen. Andererseits musste bei der Erarbeitung der Ideen berücksichtigt werden, dass es sich um zwei sehr unterschiedliche Zielgruppen handelt, welche zusammen etwas spielen sollen.\\\\
Nebst dem umgesetzten Konzept «Kinder-Einkaufsladen», auf welchem Palim-Palim beruht, wurde beispielsweise die Idee für ein «virtuelles Bilderbuch» in Erwägung gezogen. Die Grosseltern und ihr Enkelkind sollten sich dabei als virtuelle Avatare durch ein Bilderbuch hindurchbewegen. Mit Hilfe des Videochats sollen die Avatare die Gesichter der Grosseltern und des Kindes erhalten. Der betagten Person würde der Text der Geschichte zum Vorlesen angezeigt werden, während das Kind nur die Bilder sieht. Über die Videochat-Verbindung könnten die beiden Spielenden so die Geschichte gemeinsam entdecken.\\\\
Im Falle von Palim-Palim, aber auch im Beispiel des virtuellen Bilderbuches war dabei schnell klar, dass die ältere Person die Kontrolle über das Spiel übernehmen muss. Mit einer asymmetrischen Gestaltung der Spielmechanik, wie zum Beispiel dem Verteilen spezifischer Rollen sollte hier Abhilfe geschafft werden. Dass damit gleichzeitig die unterschiedlichen Ansprüche der beiden Zielgruppen an das Game Design angegangen werden konnten, war ebenfalls ein Grund, warum sich ein Rollen-basiertes Spiel wie Palim-Palim durchgesetzt hat.\\\\
Der Entscheid für Palim-Palim begründet weiterhin darin, dass das Kaufladen-Prinzip beiden Spielenden bereits aus dem echten Leben bekannt ist. Damit sollte eine tiefere Lernkurve für die Spielmechanik sowie ein gewisser Wiedererkennungswert geschaffen werden. Ausserdem verspricht dieses Konzept sehr viel Ausbaumöglichkeiten. Palim-Palim bietet mit seiner Sicht über den Tresen die Möglichkeit, den Videochat in unterschiedlichsten Ausprägungen in die Szene zu integrieren, ohne die Mechanik grundlegend zu verändern.

	\section{Spielkonzept}
		Das dem Videospiel zugrunde liegende Spielkonzept lehnt sich dem Kinder-Kaufladen-Spielprinzip aus dem echten Leben an. Das Kind und die betagte Person schlüpfen dabei in die entsprechenden Rollen von Verkaufspersonal und Kundschaft in einem virtuellen Einkaufsladen. Die Spieler*innen sehen dabei jeweils einen gemeinsamen Tresen aus der Perspektive ihrer Rolle (siehe Abbildung 1). Beide Seiten können Objekte auf dem Tresen platzieren, beziehungsweise Objekte vom Tresen wegziehen. Zusätzlich besteht die Möglichkeit, die Objekte auf den Video-Stream des Gegenübers zu ziehen. Durch all diese Aktionen können je nach Objekt diverse Animationen, Filter-Effekte und Gameplay-Events ausgelöst werden.
		\subsection{Gameplay-Loops}
		Durch die Aufteilung in die beiden Rollen lassen sich für beide Spieler*innen verschiedene Aktionen und somit ein unterschiedliches Gameplay definieren. Diese Art der Unterteilung eignet sich hervorragend, um das Gameplay jeweils auf die spezifischen Anforderungen von Kindern bzw. betagten Personen anpassen zu können.\\\\
		Der Gameplay-Loop ist in der nachfolgenden Abbildung 2 skizziert. Die Haupt-Interaktionen zwischen den beiden Spieler*innen bestehen aus der Kommunikation der Einkaufsliste und dem Hin- und Herreichen von Objekten auf der gemeinsamen Tresen-Fläche.\\\\
		\emph{\color{blue} Hier das Spiel (aktueller Stand beschreiben und nur auf mögliche Weiterentwicklungen hinweisen. Nicht alle möglichen angedachten Funktionen auflisten, das machen wir erst am Schluss}\\\\
		Dieses übergeordnete Gameplay soll als Leitfaden für den Spielverlauf dienen. Allerdings bietet das Spiel durch die gemeinsame Tresen-Fläche und dem interaktiven Video-Stream bewusst Möglichkeiten für sonstige Interaktionen mit dem Gegenüber. Damit soll für die Benutzerinnen und Benutzer ein gewisser Freiraum entstehen, in welchem sie den Verlauf des Spiels auf spielerische Art und Weise beeinflussen können. Dies soll das Spiel besonders für das Kind interessanter machen, um eine aktive und lustige Kommunikation mit der betagten Person zu fördern. 
	\section{Spielvarianten/Videomodi}\label{spielvarianten}
Damit der Einfluss der Videotelefonie als Teil des Spiels messbar wird, implementiert Palim-Palim unterschiedliche Videomodi. Dabei sind die ersten drei der folgenden Modi im aktuellen Stand bereits umgesetzt, die letzten beiden nur angedacht: 
\begin{enumerate}
	\item Spiel ohne Videochat 
	\item Spiel mit Videochat, aber getrennt von der Spielszene 
	\item Spiel mit Videochat, in die Spielumgebung integriert 
	\item Spiel mit Videochat, in die Spielumgebung integriert und optionale Interaktionsmöglichkeiten mit dem Videostream 
	\item Spiel mit Videochat, speziell in die Spiel-Umgebung integriert und zum Spielerfolg notwendige Interaktionsmöglichkeiten mit dem Videostream 
\end{enumerate}

In der ersten Variante sehen sich die Spielenden nicht. Sie können aber per Audio miteinander kommunizieren. Wie in Abb. xy ersichtlich ist, werden die Verkaufsperson sowie die einkaufende Person durch Illustrationen ersetzt.\\\\
Screenshot Videomodus 1 (Perspektive betagte Person, also Einkäufer*in sichtbar)\\\\
Der Screenshot des zweiten Videomodus (Abb. xy) zeigt, dass in diesem Modus die Videos der Spielenden unabhängig von der Gameszene in einer Ecke des Bildschirms dargestellt werden. Identisch zum ersten Modus werden die Personen durch Illustrationen ersetzt.\\\\
Screenshot Videomodus 2 (Perspektive Kind, also Käufer*in sichtbar)\\\\
Der Videomodus 3 rendert die Videos der Spielenden an die Position der ein- und verkaufenden Person. Dadurch soll der Eindruck entstehen, das Kind stünde hinter dem Tresen und verkauft Objekte und die Grossmutter oder der Grossvater des Kindes stünde auf der anderen Seite des Tresens und ist die Kundschaft. Diese Szenerie wird auf Abb. xy aus Sicht der betagten Person gezeigt.\\\\
Screenshot Videomodus 3 (Perspektive betagte Person, also Kind sichtbar)\\\\

	\section{Design und Usability}
	
Dieses Kapitel soll die zentralsten Entscheide im Bereich des Designs und der Usability von Palim-Palim aufzeigen und erklären.\\\\
Wie im Kapitel zum intergenerationellen Spielen erwähnt, ist ein asymmetrisches Gameplay für intergenerationelle Spiele sehr geeignet. Der gesamte Prozess vom Aufruf der Website bis zum Starten des Spiels ist ebenfalls asynchron gestaltet. Die betagte Person übernimmt dabei die zentrale Rolle, betritt als erstes den Spielraum, startet das Spiel und wählt den Game- und Video-Modus aus. Dies wurde aufgrund des Alters der Zielgruppe so umgesetzt, da das Kind vielleicht noch nicht lesen kann. Wie im Kapitel «Erkenntnisse von Seiten der Kinder» beschrieben, haben die Spieletests jedoch gezeigt, dass diese Umsetzung nicht optimal ist. Wie eine synchrone Umsetzung der Gamelobby aussehen könnte, wird im Kapitel «Mögliche Weiterentwicklungen» beschrieben.\\\\
Palim-Palim ist im Querformat umgesetzt und aktuell nicht für Hochformat optimiert. Der Hauptgrund dafür ist die bessere Stabilität, wenn das Tablet mit einer Halterung aufgestellt wird. Dies ist für ein Videospiel mit einer zentralen Funktionalität der Kamera wichtig, denn wenn das Tablet auf den Tisch gelegt wird, ist der Bildausschnitt suboptimal.\\\\
Aus verschiedenen Gründen ist Palim-Palim in einer 3D-Welt gestaltet. Der wichtigste Grund ist, dass dadurch ein Raumgefühl entsteht und sich die Spielenden mittendrin befinden sollen, beispielsweise bekommt der Raum zwischen den Spielenden, respektive der Tresen, so eine Tiefe, wodurch ein Abstand zwischen den Verkaufsobjekten und dem Mitspielenden entsteht. Des Weiteren ist die Möglichkeit vorhanden, eine Physik-Engine zu implementieren. Durch die 3D-Welt können Objekte dann nicht nur herunterfallen und nach rechts oder links rollen, sondern auch in die Tiefe rollen oder geworfen werden. Eine genauere Erklärung einer solchen Erweiterung ist im Kapitel «Weiterentwicklungen» zu finden.\\\\
Der Hintergrund der Szene soll zeigen, dass es sich um einen Kaufladen handelt, die Erkennbarkeit der Details liegt dabei nicht im Vordergrund, da diese für das Spiel nicht entscheidend sind. Wie im Kapitel «Erkenntnisse von Seiten der betagten Menschen» beschrieben, wurde bei den nicht repräsentativen Spieletests der Hintergrund als suboptimal bezeichnet.\\\\
Für die 3D-Objekte wurden möglichst realitätsgetreue Modelle verwendet, unter der Annahme, dass diese von den betagten Menschen einfacher erkannt werden können, im Gegensatz zu abstrakteren Illustrationen.\\\\
Das Verschieben von Objekten mit dem Finger erfordert eine gewisse Präzision. Um eine höhere Trefferwahrscheinlichkeit zu gewährleisten und damit die Benutzerfreundlichkeit zu erhöhen, implementiert Palim-Palim um jedes verschiebbare Objekt eine etwas grössere Hitbox, welche zusätzliche Fläche zur Auswahl ermöglicht. Die genaue Beschreibung der Interaktion und die Implementation der Hitbox ist im Kapitel «Interaktionen» dokumentiert.


\chapter{Technologien}
	Palim-Palim kombiniert die Funktionalitäten einer Videochat-Applikation mit einem Multiplayer-Spiel. Um diesen Anforderungen gerecht zu werden, wurden entsprechende Technologien zur Umsetzung evaluiert. Dabei fiel die Wahl auf WebRTC (Web Real-Time Communication) \cite{noauthor_webrtc_2011} zur Übertragung der Multimedia-Daten, sowie Three.js \cite{noauthor_threejs_nodate} als 3D-Bibliothek für die Gestaltung und Umsetzung der 3D-Gameplay-Szene. In diesem Kapitel wird als erstes die Technologiewahl begründet. Anschliessend werden die verwendeten Frameworks und deren Funktionen vorgestellt.
	
\section{Technologiewahl}
Zu den zentralen Anforderungen des Spielkonzepts von Palim-Palim gehören die Mehrspieler-Funktionalität sowie der Videochat. Um diesen Bedürfnissen gerecht zu werden, wurden zu Projektbeginn nach passenden Technologien gesucht. Zur Umsetzung des Videochats hat sich sehr schnell WebRTC als Framework für Video-Anrufe anerboten. Die Auswahl von WebRTC begründet sich hauptsächlich in der hohen Etablierung des Standards im Web. Zudem wird es von allen wichtigen Browsern unterstützt \cite{noauthor_webrtc_nodate-3}.\\\\
Für die Umsetzung der Spiellogik wurde vorerst Unity als Game-Engine in Betracht gezogen. Mit dem «WebRTC for Unity»-Package \cite{noauthor_webrtc_2021-1} würde auch hier die Möglichkeit bestehen, einen Videochat in eine Unity-Szene zu integrieren. Es wurde bei der Evaluierung der Technologien aber schnell klar, dass eine möglichst hohe Kontrolle über den Video-Stream benötigt wird, falls dieser interaktiv und Teil des Gameplays werden sollte. Funktionen wie Gesichtserkennung für eventuelle Filter-Effekte, sowie Hintergrund-Segmentierung zum Freistellen der Person im Video sind zentrale Ansprüche eines Spieles, in dem der Videochat so eine zentrale Rolle spielt. Hierzu bietet Unity selbst wenig Unterstützung. Es müssten kostenpflichtige Plugins wie das Face Tracking Plugin von Banuba \cite{noauthor_introducing_nodate} herbeigezogen werden, welches einige praktische Funktionen mit sich bringt. Allerdings war nicht klar, ob mit solchen Plugin-Lösungen für Unity wirklich alle Bedürfnisse von Palim-Palim damit erfüllt können. Zusätzlich möchte man keine Abhängigkeiten zu den Autoren dieser Plugins aufbauen, welche man bei der Weiterentwicklung dann bereut.\\\\
Es hat sich deshalb anerboten, Palim-Palim als Web-Applikation mit JavaScript zu entwickeln. Dies ermöglicht die Nutzung von unzähligen Open-Source-Bibliotheken und bietet maximale Flexibilität in der Umsetzung. Bestehende Libraries von Tensorflow.js können beispielsweise für Gesichts-Filter-Effekte \cite{noauthor_tfjs-modelsface-landmarks-detection_nodate} oder zur Hintergrund-Segmentierung \cite{noauthor_tfjs-modelsbody-pix_nodate} importiert werden.\\\\
Für die Gestaltung der Spielewelt wurde Three.js als Framework gewählt. Mit dem ursprünglichen Ziel, eine realistisch wirkende Spielumgebung für Jung und Alt zu schaffen, schien Three.js eine zeitgemässe Wahl zur Erschaffung einer immersiven Spielwelt.
	
\section{WebRTC}
WebRTC ist ein Open-Source-Projekt, das die Echtzeitkommunikation von Audio, Video und Daten in Web- und nativen Anwendungen ermöglicht \cite{noauthor_webrtc_2011}. Die Technologie ist in allen modernen Browsern sowie auf nativen Clients für alle wichtigen Plattformen verfügbar, ist eine Empfehlung des World Wide Web Consortium (W3C) \cite{noauthor_webrtc_2021} und ein Standard der Internet Engineering Task Force (IETF) \cite{noauthor_support_2021}. Dabei wird zwischen zwei Browsern eine Peer-to-Peer-Verbindung aufgebaut, worüber die Daten gestreamt werden. Auf der Peer-to-Peer-Verbindung lassen sich auch eigene Daten-Kanäle erstellen, die zum Beispiel zur Übertragung von Text-Nachrichten, Positions-Daten von Game-Objekten oder sogar Dokumenten zwischen den Peers verwendet werden können \cite{noauthor_webrtc_nodate}. Damit der Verbindungsaufbau reibungslos funktioniert, bedient sich WebRTC einiger Server-Verbindungen, welche in den folgenden beiden Kaptiel erläutert werden.

\subsection{Signaling-Server}\label{signaling-server}
WebRTC verwendet die RTCPeerConnection JavaScript API (Application Programming Interface), um Streaming-Daten direkt zwischen Browsern zu kommunizieren \cite{noauthor_rtcpeerconnection_nodate}. Um eine direkte Verbindung zwischen zwei sich unbekannten Clients herzustellen, wird ein Prozess benötigt, welcher die Kommunikation der beiden Peers koordiniert und Kontrollnachrichten sendet. Dieser Prozess wird in WebRTC als \textit{Signaling} bezeichnet. Signaling-Methoden und -protokolle sind von WebRTC nicht spezifiziert und können je nach Anwendungsfall entsprechend gewählt werden. WebRTC-Anwendungen, welche skalierbar sein sollen, benötigen Signalisierungsserver, welche in der Lage sind eine erhebliche Last zu bewältigen. Solche Lösungen wurden im Rahmen von Palim-Palim nicht benötigt, weshalb an dieser Stelle lediglich auf entsprechende Ressoucen verwiesen wird \cite{article_build_nodate} \cite{venema_how_nodate}.\\
Für einfachere Anwendungen ohne Anforderungen an die Skalierung der Benutzeranzahl bietet sich die JavaScript-Library Socket.io für die Signalisierung an \cite{arrachequesne_socketio_2021}. Socket.io erlaubt eine einfache und bidirektionale Kommunkation zwischen den Clients und einem Signaling-Server. Mit seinem integrierten Room-Konzept eignet sich Socket.io zudem sehr gut für eine Video-Chat-App. Der Server hat dabei folgende zwei Aufgaben: er dient als Message Relay und verwaltet alle Videochat-Räume. \\\\
Die Funktionaliät als Message Relay ist wichtig, da sich die beiden Clients vor dem Verbindungsaufbau noch nicht kennen. Der Signaling-Server als Message Relay ist dabei für beide Clients der erste gemeinsame Kontaktpunkt. Nur so können initiale Informationen der Clients ausgetauscht werden, damit diese untereinander eine WebRTC-Peer-Verbindung aufbauen können.
		

\subsection{STUN- und TURN-Server}			
WebRTC ist grundsätzlich so konzipiert, dass es Peer-to-Peer funktioniert. Zwei Clients können sich also auf dem direktesten Weg verbinden. Die Technologie ist jedoch auch bewusst darauf ausgelegt, mit realen Netzwerken zurechtzukommen: Client-Anwendungen müssen NAT-Gateways (Network Address Translation) und Firewalls überwinden, und Peer-to-Peer-Netzwerke benötigen eine Absicherung, falls die direkte Verbindung ausfällt. Als Teil dieses Prozesses verwendet die WebRTC-API zwei Netzwerkprotokolle als Hilfsmittel: 
			
\begin{itemize}
  \item \textit{STUN} (Session Traversal Utilities for NAT), um die öffentliche IP-Adresse und Port-Nummer für den direkten Kontaktaufbau zu ermitteln  \cite{noauthor_stun_2021}.
  \item \textit{TURN} (Traversal Using Relay NAT), welches die Kommunikation über NAT- oder Firewallgrenzen hinweg ermöglicht und als Fallback-Relay genutzt werden kann, sollte keine Peer-to-Peer-Verbindung möglich sein \cite{noauthor_traversal_2021}.
\end{itemize}

Für beide Protokolle müssen Server bereitgestellt werden, welche den WebRTC-Clients bekannt sein müssen. Die STUN- und TURN-Adressen müssen bei dem Aufbau der Peer-Connection der Clients an den Signaling-Server übermittelt werden \cite{noauthor_turn-server_nodate-1}. Wie dies in Palim-Palim umgesetzt wurde, ist im Kapitel \ref{server} genauer beschrieben.\\\\
Als Fallback-Verbindungen wird im produktiven Betrieb einer WebRTC-Applikation unbedingt ein eigener TURN-Server mit einer öffentlich sichtbaren IP-Adresse benötigt \cite{noauthor_webrtc_2020}. Es gibt keine kostenlosen, öffentlich gehosteten TURN-Server, da der Netzwerkverkehr über so einen Server sehr stark ansteigen kann. Daher lohnt es sich für niemanden, fremden Verkehr über seinen TURN-Server zuzulassen.\\\\
Gemäss Auswertungen des WebRTC Call Quality Dienstleisters callstats.io haben im Durchschnitt etwa 30 Prozent aller WebRTC-Sessions einen Client, der sich über einen TURN-Server verbindet \cite{callstats_why_nodate}. Das heisst in etwa 70 Prozent der Fälle ist die Peer-to-Peer-Verbindung stabil genug, und ein TURN-Server wird überhaupt nicht benötigt. Jedoch müssen sie trotzdem bei jeder Verbindung angegeben werden. Denn gewisse Benutzer*innen wären ohne die Unterstützung eines TURN-Relay-Servers nicht in der Lage zu kommunizieren.\\\\

\begin{figure}[h!]
\includegraphics[width=\textwidth]{WebRTC_nat_stun_firewall_turn_black}
\caption[Caption for LOF]{STUN, TURN, und Signalisierung in WebRTC (eigene Darstellung).}
\end{figure}
			
Mit STUN-, TURN- und Signaling-Server besitzt Palim-Palim ein stabiles Backend für die Implementierung des Video-Streams mit WebRTC. Zusätzlich kann die Peer-to-Peer-Funktionalität von WebRTC auch zur Synchronisation von Gameplay-Daten direkt zwischen Alice und Bob genutzt werden. Die Multiplayer-Funktionalität kann so auch unabhängig von einem zentralen Server als gemeinsame Authorität ermöglicht werden.\\\\
			
\subsection{Sicherheit}
Der Open-Source-Charakter von WebRTC kann bei potenziellen Anwender*innen der Technologie Sicherheitsbedenken hervorrufen. Diese Bedenken sind berechtigt und wurden bei der Entwicklung von Palim-Palim ebenfalls berücksichtigt. Damit der persönliche Datenschutz für die Spielenden sowie die Stabilität des Games gewährleistet werden kann, müssen bei der Implementation einige Punkte berücksichtigt werden. \\\\
Auf der Protokoll-Ebene ist WebRTC als Standard sehr sicher und auch bereits etabliert. Verschlüsselung ist für alle WebRTC-Komponenten obligatorisch, und seine JavaScript-APIs können nur von sicheren Quellen (HTTPS oder localhost) aus verwendet werden \cite{noauthor_webrtc_2020-1}. Das verwendete Verschlüsselungs-Protokoll hängt dabei vom Kanaltyp ab. Daten-Kanäle werden mit Datagram Transport Layer Security (DTLS) und Medien-Kanäle mit Secure Real-time Transport Protocol (SRTP) verschlüsselt. Unverschlüsselte Kommunikation gibt es in WebRTC also nicht. Die Schlüssel zur Entschlüsselung der Medien-Kanäle werden zudem nicht über den Signaling-Server ausgetauscht, sondern nur direkt zwischen den Peers. \\\\
Ebenfalls werden bei der Umleitung des Datenverkehrs über den TURN-Server die Daten nicht interpretiert oder modifiziert. Dies ist so im TURN-Standard definiert \cite{noauthor_rfc5766_nodate}. Das heisst, die Verwendung des TURN-Servers fügt dem WebRTC-Datenverkehr keine Sicherheitsschwachstellen hinzu. Die Verbindung zum Signaling-Server ist ebenfalls mit HTTPS geschützt, was als sicher genug für Online-Banking und Behörden-Websites gilt \cite{noauthor_study_nodate}. \\\\
Somit lässt sich WebRTC als eigenständiges Framework für Videotelefonie als sehr sicher bezeichnen. Allerdings hängt diese Sicherheit auch von der umliegenden Web-Applikation und diese wiederum von dem verwendeten Browser ab. Als ein im Internet zugängliches Spiel sollte eine solche Anwendung deshalb über eine entsprechende User-Authentifizierung erweitert werden, sollte das Spiel dauerhaft produktiv betrieben werden wollen. Diese Authentifizierung darf dabei nicht über den Signaling-Server laufen, denn der Datenverkehr zu ihm könnte abgefangen werden. Jede Client-Anwendung muss in der Lage sein, die Authentifizierung der potenziellen Gesprächspartner unabhängig vom Signaling-Server durchzuführen. Dies kann durch die Verwendung eines webbasierten Identitätsanbieters (IdP) erreicht werden, wie zum Beispiel Facebook Connect \cite{noauthor_facebook_nodate} oder OAuth (von Twitter) \cite{noauthor_oauth_nodate}. Ebenfalls empfiehlt es sich, die WebRTC-Bibliothek und andere Abhängigkeiten aktuell zu halten \cite{noauthor_study_nodate}.
	
\section{Three.js}
Three.js ist eine 3D-Bibliothek, welche meistens die Web Graphics Library (WebGL) verwendet, um 3D-Inhalt darzustellen \cite{noauthor_threejs_nodate}. WebGL ist eine Rasterisierungs-Engine, das bedeutet, dass diese Bibiliothek nur Punkte, Linien und Dreiecke zeichnet und erst auf Basis des Codes entstehen dreidimensionale Objekte \cite{noauthor_webgl_nodate}. In Three.js stehen primitive 3D-Objekte genauso wie beispielsweise Lichter, Schatten, Materialien oder Texturen als vorgefertigte Code-Pakete zur Verfügung \cite{noauthor_threejs_nodate}.
	

\subsection{Struktur}
Das oberste Element einer Three.js-Struktur ist der Renderer. Dieser nimmt eine Szene (Scene) und eine Kamera (Camera) entgegen und zeichnet den im Blickfeld befindlichen Teil der 3D-Scene auf eine 2D-Leinwand.
In der Scene werden Lichter (Light), 3D-Objekte (Object3D) und Kameras (Camera) platziert. Die Scene ist die Wurzel einer Baumstruktur, in welcher Kinder relativ zu ihren Eltern ausgerichtet werden.
Mesh-Objekte sind gezeichnete Formen (Geometry) mit einem bestimmten Material. Geometry- und Material-Objekte können von verschiedenen Mesh-Objekten verwendet werden. Three.js bietet bereits einige primitive Geometry-Objekte an, wie beispielsweise Würfel (Cube), Zylinder (Cylinder) oder Pyramiden (Cone). Es können aber auch eigene Geometries erstellt oder aus einer Datei importiert werden. Material-Objekte repräsentieren die Oberflächenbeschaffenheit eines Objekts, wie zum Beispiel die Farbe oder wie fest ein Objekt spiegelt \cite{noauthor_threejs_nodate}.


\chapter{Implementation}
In diesem Kapitel werden die einzelnen Komponenten von Palim-Palim und deren Funktionen vorgestellt. Als erstes wird das Server-seitige Backend erklärt und danach die beiden zentralen Client-Komponenten, der PeerConnectionManager sowie der GameManager vorgestellt. Die Struktur von Palim-Palim wurde bewusst so aufgebaut, dass die Video-Chat-Funktionalität möglichst von der Game-Logik entkoppelt ist. Dadurch ist es einfacher, den Video-Chat auch in anderen Spielen zu integrieren. Ebenfalls war die Absicht, dadurch eine möglichst gute Grundlage zur Entwicklung eines entsprechenden Frameworks zu legen.
	
\section{Server}\label{server}
Da Palim-Palim über das Internet zugänglich sein soll, benötigt das Spiel einen Server, welcher den Client Code zur Verfügung stellt. Der Server-Code im server.js File nutzt daher Express.js \cite{noauthor_express_nodate}, womit der mit Webpack \cite{noauthor_webpack_nodate} gebaute Client-Javascript-Code den Benutzer*innen beim Aufruf der Webapplikation zur Verfügung gestellt wird.\\\\
Der Server verwaltet zusätzlich alle Videochat-Räume und dient als Vermittler für den Verbindungsaufbau des Video-Calls. Dies wird auch Signaling genannt, wie bereits im Kapitel \ref{signaling-server} erläutert wurde, und funktioniert folgendermassen: Sobald zwei Spielende einem Raum betreten, wird dieser Raum genutzt, um Verbindungsinformationen zwischen den Peers auszutauschen. Dies wurde in Palim-Palim gemäss einem Google Codelab Beispiel mit Socket.io umgesetzt \cite{arrachequesne_socketio_2021} \cite{noauthor_real_nodate}. Socket.io erlaubt eine einfache und bidirektionale Kommunikation zwischen den Clients und dem Server. Mit seinem Room-Konzept eignet sich Socket.io zudem sehr gut für eine Video-Chat-App. Über den Socket.io-Raum kann die Peer-to-Peer-Verbindung zwischen den Clients dann ausgehandelt und erstellt werden. Steht diese direkte Verbindung einmal, wird der Server nicht mehr benötigt. Ab dann läuft die Kommunikation in den meisten Fällen nur noch direkt zwischen den Clients.\\\\
Um unabhängig zu funktionieren, wurde für Palim-Palim zudem ein eigener TURN-Server aufgesetzt. Der im Projekt Palim-Palim verwendete TURN-Server läuft mit Linux Ubuntu und benutzt Coturn, eine Open-Source Implementierung des TURN-Protokolls \cite{noauthor_coturncoturn_2021}.\\\\

Um selbst ein TURN-Sever bereitzustellen, wird lediglich ein Server mit einer öffentlichen IP-Adresse benötigt. Diesen versieht man idealerweise mit einer ensprechenden Authentifizierung, um unerwünschten Datenverkehr zu unterbinden. Nach der Konfiguration können die Adresse und die Zugangsdaten des TURN-Servers beim Erstellen der PeerConnection im Client-Code von Palim-Palim in der Konfiguration mitgegeben (siehe Codefragment \ref{lst:peerConfig}). Eine Anleitung zum Einrichten und Konfigurieren eines Linux-TURN-Servers inklusive Verweise auf weitere Ressourcen zu dem Thema sind im Anhang dieser Arbeit zu finden.\\\\

	\begin{lstlisting}[caption={Konfiguration der PeerConnection mit der TURN-Server Adresse},label={lst:peerConfig},language=JavaScript]
this.peerConnectionConfig = {
	'iceServers': [
		{
			'urls': 'stun:stun.l.google.com:19302'
		},
		{
			'urls': 'turn:86.119.43.130:3478',
			'credential': '*****************',
			'username': 'palimpalim'
		}
	]
};
	\end{lstlisting}
	

\newpage
\section{PeerConnectionManager}
Die Video-Chat-Funktionalität wurde in Palim-Palim möglichst unabhängig vom Gameplay als eigene Klasse implementiert. Die PeerconnectionManager-Klasse übernimmt hierbei zentrale Funktionen wie das Betreten eines Raumes, die Etablierung der Video-Streams zwischen den Spielern sowie den Aufbau von dedizierten Datenkanälen (sogenannten DataChannels).\\\\ 
In der PeerConnectionManager-Klasse selbst findet der Aufbau der Peer-to-Peer-Verbindung statt. Die Klasse instanziiert hierzu ein PeerConnection-Objekt, welches die RTCPeerConnection erweitert. RTCPeerConnection ist die API, welche von WebRTC-Anwendungen verwendet wird, um eine Verbindung zwischen Peers herzustellen und Audio und Video zu übertragen \cite{noauthor_rtcpeerconnection_nodate-1}. Diese PeerConnection stellt eine WebRTC-Verbindung zwischen dem lokalen Computer und einem entfernten Peer dar. Sie bietet Methoden, um eine Verbindung zu einer Gegenseite herzustellen, die Verbindung aufrechtzuerhalten und zu überwachen sowie die Verbindung zu schließen, wenn sie nicht mehr benötigt wird. Der PeerConnectionManager orchestriert über dieses Objekt den Verbindungsaufbau und dient als einzige Schnittstelle des Video-Chats zum Rest der Applikation.\\\\
Um alle diese Funktionen innerhalb dieser Klasse möglichst übersichtlich zu umzusetzen, ist die PeerConnectionManager-Klasse von Palim-Palim selbst in einzelne Unterklassen mit den entsprechenden Teil-Verantwortlichkeiten aufgeteilt. Diese Komponenten sind in der Abbildung \ref{fig:PeerConnectionManagerDiagram} dargestellt. Die einzelnen Zuständigkeiten der Klassen werden in den nachfolgenden Kapiteln vorgestellt.
\begin{figure}[!htb]
\includegraphics[width=\textwidth]{PeerConnectionManagerDiagram}
\caption[Caption for LOF]{Klassendiagramm der PeerConnectionManager-Klasse und deren Subklassen (eigene Darstellung).}
\label{fig:PeerConnectionManagerDiagram}
\end{figure}

\newpage
\subsection{RoomManager}
Der RoomManager ist als einzige Schnittstelle mit dem Server für das WebRTC-Signaling des Clients zuständig. Er besitzt die Verantwortung über die Client-seitige Socket.io-Verbindung. Der RoomManager ermöglicht es somit einem Client via Socket.io einem Raum beizutreten. Der PeerConnectionManager verwendet den RoomManager, um Signaling-Messages an den Server zu versenden sowie diese zu empfangen. Auf dem Server werden die Anfragen der einzelnen Clients dann gehandelt, wie in Kapitel \ref{server} genauer beschrieben.

\begin{figure}[!htb]
\includegraphics[width=\textwidth]{RoomManagerSd_noFrame}
\caption[Caption for LOF]{Nachrichtenfluss beim Betreten eines Raumes (eigene Darstellung).}
\end{figure}

\newpage 
\subsection{VideoChatManager}
Sobald der Spieler einen Raum betreten hat, wird via VideoChatManager der lokale Video- und Audiostream initiiert. Im JavaScript-Code geschieht dies via Navigator, wodurch der Browser den User um die Erlaubnis der Verwendung von Mikrofon und Kamera bittet. Erst wenn diese Erlaubnis gegeben wurde, kann auf die Daten von Kamera und Mikrofon zugegriffen werden. Der VideoChatManager fügt dann den Stream dem entsprechenden DOM-Element hinzu, wodurch die Spieler ihr eigenes Video sehen können. Ebenfalls löst dies in Palim-Palim ein «got user media»-Event aus, wodurch der PeerConnectionManager sowie der Signaling-Server informiert werden, dass dieser Client jetzt Zugriff auf Video- und Audio-Daten hat. Danach können der PeerConnection die MediaTracks hinzugefügt werden. \\\\
Das Hinzufügen der Tracks auf der PeerConnection löst in WebRTC auf der Gegenseite automatisch ein «trackAdded»-Event aus \cite{noauthor_webrtc_nodate-1}. Dieses wird ebenfalls durch den VideoChatManager gehandlet. Fügt also das Gegenüber der gemeinsamen PeerConnection seinen Track zur Verbindung hinzu, werden im gegenüberliegenden VideoChatManager aus diesem Event die nötigen Daten für den Remote-Video-Stream ausgelesen. Dadurch entsteht eine Video-Konferenz zwischen den beiden Mitspielern.
\begin{figure}[!h]
\includegraphics[width=\textwidth]{VideoChatManager}
\caption[Caption for LOF]{Sequenzdiagramm zur Erstellung der Video-Verbindung (eigene Darstellung).}
\end{figure}

\newpage
\subsection{DataChannelManager}
Die dritte Teil-Komponente des PeerConnectionManagers ist der DataChannelManager, welcher für das Erstellen von Datenkanälen zuständig ist. Diese sogenannten DataChannels erlauben das Versenden beliebiger Daten über die PeerConnection \cite{noauthor_webrtc_nodate}. In Palim-Palim wird diese Funktion genutzt, um die 3D-Positionen der Gegenstände zu synchronisieren, sowie den Remote-Peer über spezifische Game-Events zu informieren. Dazu werden zwei einzelne DataChannels erstellt, welche unabhängig voneinander, aber auf der gleichen PeerConnection laufen.\\\\
Der darunterliegende RTCDataChannel besitzt einige Properties, um die Übertragungsart genau an die Anforderungen anzupassen. Zum Beispiel kann mittels dem Property «ordered» bestimmt werden, ob die Nachrichten in der gleichen Reihenfolge ankommen wie sie versendet wurden. WebRTC baut dann im Hintergrund je nach Konfiguration des DataChannel-Objekts eine entsprechende TCP oder UDP-Verbindung auf \cite{noauthor_rtcdatachannel_nodate}. Beim Erstellen eines DataChannels muss dabei ebenfalls die Callback-Funktion angegeben werden, welche im Falle einer Message auf diesem Kanal aufgerufen werden soll. Deshalb wird hier eine Referenz auf den GameSyncManager von Palim-Palim gebraucht, welcher in Kapitel x genauer beschrieben ist.

	
	\section{Game}
Neben dem PeerConnectionManager als zentrale Steuereinheit der Videochat-Funktionalität, bildet der GameManager die zweite zentrale Klasse und übernimmt die zentralen Funktionen des Spiels Palim-Palim. Dazu gehört das Starten des Games, die Verwaltung der Game-Lobby, das Handling von Spielgeschehnissen sowie das Handling des Spielendes. Um diese Funktionen und das komplette Handling des Spiels übersichtlich zu gestalten, besitzt der GameManager je eine Instanz von weiteren Managern, welche wiederum Aufgaben übernehmen. Diese einzelnen Klassen werden nachfolgend genauer erläutert.\\\\
Der \textbf{GameLobbyManager} dient zur Verwaltung der Game-Lobby. Er besitzt die Möglichkeit den Eröffnungsscreen, die Einstellungsscreens, die Erfolgsmeldung sowie auch den Screen für das Spielende und den Neustart anzuzeigen und die getätigten Benutzereingaben zu verwalten. Einige Eingaben führen zu weiteren Screens, andere lösen über den GameSyncManager eine Nachricht an den Peer aus.\\\\
Der \textbf{GameSyncManager} hält die beiden Peers synchron. Vom GameLobbyManager oder vom GameManager aus, können über ihn Nachrichten an den anderen Peer gesendet werden. Bei einkommenden Nachrichten entscheidet der GameSyncManager, welche Events dadurch ausgelöst werden. Wie die genaue Synchronisation der Spielobjekte funktioniert, ist im späteren Kapitel «Gameplay Synchronisation» beschrieben.\\\\
Der \textbf{SceneManager} ist die zentrale Steuereinheit der Game-Scene, alle Änderungen an der Scene geschehen darüber. Das Laden der 3D-Objekte, welches ebenfalls vom SceneManager übernommen wird, ist im nachfolgenden Kapitel «3D-Objekte laden» genauer erklärt.\\\\
Der \textbf{AudioManager} ist befähigt Audio-Dateien abzuspielen.\\\\
Über den \textbf{ShoppingListManager} wird die Einkaufsliste zufällig generiert. Dabei werden zwischen drei und fünf (BemSev: Vielleicht wissenschaftlich belegen, dass man sich so viele Objekte gut merken kann?) Verkaufsobjekte aus den möglichen Objekten ausgewählt, in einer Map gespeichert und über den SceneManager dem Benutzer angezeigt.\\\\
Der \textbf{GameStateManager} überprüft, ob das Ziel des Spiels bereits erreicht wurde. Dabei wird die Map mit der Einkaufsliste mit den Objekten im virtuellen Einkaufskorb (Basket) abgeglichen. Dies erfolgt jeweils beim Hinzufügen eines Objekts in den Einkaufskorb. Im Erfolgsfall, wird über den gameSyncManager eine Nachricht an den Peer verschickt und über den EventDispatcher eine lokale Nachricht versendet. \\\\
Der \textbf{InteractionManager} sorgt dafür, dass Benutzerinteraktionen in der 3D-Welt zu einer Aktion führen. Das genaue Handling dieser Interaktionen wird im Kapitel «Interaktionen» beschrieben.\\\\
(BemSev: Sequenzdiagramme aus dem Anhang erwähnen)
	
	\subsection{3D-Objekte laden}
	Nach Möglichkeit sollte bei Webanwendungen die Datenmenge, welche über den Flaschenhals Netzwerk gesendet werden muss, möglichst geringgehalten werden, um lange Ladezeiten zu verhindern. Da 3D-Objekte bereits eher grosse Dateien sind, ist es gerade bei deren Verwendung wichtig, einige Punkte zu beachten. Nachfolgend wurde dokumentiert welche Optimierungen in Palim-Palim implementiert sind.
Palim-Palim lädt 3D-Objekte, welche in allen Spielmodi verwendet werden (Tresen, Einkaufskorb), während sich die Spielenden noch in der Gamelobby befinden, weswegen für diese keine Wartezeit entsteht. Die vom Spielmodi abhängigen 3D-Objekte (Einkaufsgegenstände) werden geladen, während die betagte Person die Anleitung liest und das Kind das Spiel erklärt bekommt.\\\\
Für die 3D-Objekte verwendet Palim-Palim Dateien des Graphics Language Transmission Format (glTF). Dieses Dateiformat kann 3D-Modelle sehr effizient übertragen und laden, weswegen es sich sehr gut für Webapplikationen eignet [https://www.khronos.org/gltf/]. In einer glTF-Datei ist nicht nur die Geometrie eines 3D-Objekts gespeichert, sondern Szenen, Kameras, Materialien, Texturen und auch Animationen (nicht abschliessende Liste) [https://github.com/KhronosGroup/glTF/blob/master/specification/2.0/figures/gltfOverview-2.0.0b.png]. Deswegen muss in Palim-Palim beim Laden der 3D-Objekte zuerst die Szene der glTF Datei ausgelesen und danach in dieser nach Meshes (Geometrie eines 3D-Objekts) gesucht werden.\\\\
\begin{lstlisting}
const loadedData = await loader.loadAsync(this.config.models[i].path);
loadedData.scene.traverse((o) => {
    if (o.isMesh) {
	 ...
    }
});
\end{lstlisting}
Die in Palim-Palim verwendeten 3D-Objekte wurden mit Blender [https://www.blender.org/] auf eine möglichst kleine Dateigrösse gebracht, um die Ladezeit, zusätzlich zum optimalen Dateiformat, zu optimieren. Dazu wurde der «Decimate Modifier» angewendet, welcher es erlaubt, die Polygone einer Geometrie zu reduzieren und dabei den ursprünglichen Körper nur minimal zu verändern [https://docs.blender.org/manual/en/latest/modeling/modifiers/generate/decimate.html]. Für die bessere Verwaltung im Programm wurden in Blender zusätzlich die zwei folgenden Modifikationen an den ursprünglichen glTF-Dateien durchgeführt. Einerseits wurden Objekte, die aus mehreren Meshes bestanden, zu einem Mesh zusammengefügt und anderseits wurde der Mittelpunkt des Meshs auf den Koordinatenursprung gelegt. Die Skalierung des Meshs hingegen kann über einen Parameter in der Konfigurationsdatei von Palim-Palim festgelegt werden, somit können die einzelnen Objekte aufeinander abgestimmt werden.
	
	\subsection{Interaktionen}
Ein Objekt auf einem Screen so zu steuern, dass dieses in einem 3D-Raum bewegt wird, ist nicht intuitiv. Deswegen gibt es verschiedene Lösungsmöglichkeiten, wie dies bewerkstelligt werden kann. Beispielsweise kann das Gyroskop in die Steuerung miteinbezogen werden. Dabei könnte die Dimension in die Tiefe nur angesteuert werden, wenn das Tablet flach liegt (Winkel zur Horizontalen < 20 Grad), die vertikale Dimension nur, wenn das Tablet «steht» (Winkel zur Horizontalen > 20 Grad). Dies ist für Palim-Palim allerdings nicht geeignet, da so die Videoaufnahme sehr unruhig wird. Eine weitere Möglichkeit ist es, die Ansteuerung der dritten Dimension in die Tiefe per Ziehen vom Mittelpunkt zu einer äusseren Ecke zu bewerkstelligen. Das Ziehen von der Mitte zur rechten oberen Ecke würde das Element nach hinten bewegen und das Ziehen zur rechten unteren Ecke würde das Element nach vorne bewegen, dies haben beispielsweise Tseng et al. (EZ Manipulator Designing a mobile fast and ambiqui.pdf // todo severin pdf zotero) in ihrem Forschungsartikel so beschrieben. Die Zielgruppe von Palim-Palim soll nicht mit neuen Gesten überfordert werden, weswegen diese Technologie ebenfalls ungeeignet ist.\\\\
Palim-Palim setzt auf eine simple 2D-Steuerung, welche sich jedoch auf einer gekippten Ebene abspielt. Diese Ebene, genannt Interaktionsebene, ist vom Bildschirm aus gesehen um 45 Grad nach hinten gekippt. Das gelbe Linienkonstrukt auf der Abbildung abc zeigt die Interaktionsebene des Verkaufspersonals. Die Interaktionsebene des Kaufenden ist an der xy-Ebene in der Mitte des Tresens gespiegelt. Wird nun ein Objekt angetippt, wird die Methode onPointerDown des InteractionManager aufgerufen. Anhand eines Raycasts (BemSev: Quelle für Beschreibung?) wird das Element ausgewählt, welches am nächsten an der Kamera des Interagierenden ist. Dieses wird «aufgenommen» und kann nun in der Interaktionsebene bewegt werden und gewinnt oder verliert abhängig von der Höhe an Tiefe.\\\\
Wie im Kapitel «Gamedesign» beschrieben, benötigen die Verkaufsobjekte für eine einfachere Bedienung eine grössere Touch-Fläche. Dies ist mit einer einfachen Hitbox in Form einer Kugel realisiert. Die Kugel ist eine möglichst kleine Begrenzungskugel um den ebenfalls möglichst kleinen Begrenzungsquader um die Geometrie des Objekts. Three.js stellt dafür die Methoden getBoundingSphere und getBoundingBox zur Verfügung. Für den Einkaufskorb ist ebenfalls eine Hitbox vorhanden (roter Linienquader auf Abb. abc).\\\\
Um alle Verkaufsobjekte herum ist ebenfalls eine Box (blauer Linienquader auf Abb. abc) vorhanden. Werden Objekte innerhalb diese Bereichs gezogen, werden diese wieder an ihre Ursprungsposition gesetzt.
	

		
	\newpage
	\subsection{Gampeplay-Synchronisation}
	Um die Interaktionen der beiden Spieler zu synchronisieren, wird normalerweise ein Server verwendet, der die Inputs der Spieler entgegennimmt, und als zentrale Authorität den Zustand des Spiels bestimmt. Bei Palim-Palim wurde dies aber anders gelöst. Das Spiel nutzt seine Peer-to-Peer-Funktionalität aus und kreiert neben dem Video- und Audiostream auch einen spezifischen DataChannel für Game-Updates. Dieser Channel wird genutzt, um die Objekte beider Szenen zu synchronisieren.
	

	
Der DataChannel überträgt JSON-Strings via UDP. UDP als Protokoll ist sehr schnell und deshalb sind die Änderungen auch fast ohne Verzögerung beim Peer sichtbar. Die Wahl des Protokolls kann beim Erstellen des DataChannels gewählt werden [Code Erstllung DataChannel]. Eine weitere Eigenschaft von UDP ist, dass die Reihenfolge der Übertragung nicht garantiert ist - im Gegensatz zu TCP z.B. Da die Game-Updates sehr oft geschickt werden, ist es in diesem Fall egal, in welcher Reihenfolge die Nachrichten ankommen. (Man muss sich das für jeden Fall überlegen...)
\begin{lstlisting}
const dataChannel = peerConnection.createDataChannel('gameUpdates', {
  ordered: false,
  id: room
  });
dataChannel.onmessage = handleReceiveMessage;
dataChannel.onerror = handleError;
dataChannel.onopen = handleDataChannelStatusChange;
dataChannel.onclose = handleDataChannelStatusChange;
\end{lstlisting}

Diese Art der Gameplay-Synchronisation erlaubt es den Clients, total unabhängig von einem Server zu spielen. Da der Server nur fürs Signaling benutzt wird, ist auch eine grosse Skalierung der Spieleranzahl denkbar. Der Server muss die Spieler nur inital vermitteln, was keine grosse Sache ist. Alle anderen Berechnungen sowie der Abgleich der Spielwelt werden auf den Clients direkt vorgenommen. Da kein Umweg über einen Server genommen werden muss, ist auch die Latenz niedrig. (Einziges Manko: es gibt keine zentrale Authorität. Das heisst Spieler könnten den Client-Code manipulieren und sich unfaire Vorteile verschaffen. Da Palim-Palim und sein Publikum aber nicht kompetitiv sind, ist diese Gefahr des Cheatings vernachlässigbar.)
		
			
			
\chapter{Fazit}
	\begin{itemize}
		\item Zusammenfassung des Erreichten / Zielerreichung
		\item Zentrale Erkenntnisse
		\item Reflektion
		\item Mögliche Weiterentwicklungen (bezogen auf die Software
		\item Weiterführende Forschung
	\end{itemize}
	
Mit Palim-Palim wurde ein Videospiel entwickelt, welches eine Videochat-Applikation mit einem Spiel für Kinder und betagte Personen kombiniert. Es ist ein funktionaler Prototyp entstanden, der weiteren Videochat-Spielen als Inspiration oder sogar als Vorlage für ein entsprechendes Framework dienen kann. Erste Spieletests haben Möglichkeiten offengelegt, die spielerischen Elemente weiter zu verbessern. Die mit der Arbeit verbundene Literaturrecherche zeigte ausserdem auf, dass Spiele mit integriertem Videochat ein hohes Potenzial haben, um Beziehungen zwischen Generationen auch über weite Distanzen zu pflegen. Palim-Palim legt dabei den Grundstein für weitere solche Spiele.\\\\
Der intergenerationelle Aspekt des Spielens rückte dabei schon in der Konzipierungsphase des Projektes in den Mittelpunkt. Mit Palim-Palim wurde festgestellt, dass ein asymmetrisches Game Design hierzu eine Möglichkeit bietet, den Anforderungen zwei unterschiedlicher Zielgruppen gerecht zu werden.\\\\
Da die Ergebnisse aus den Spieletests nicht aussagekräftig sind, lässt sich über mögliche Verbesserungsmöglichkeiten bezüglich dem Gameplay von Palim-Palim nur mutmassen. Bezogen auf das Game Design steht Palim-Palim noch am Anfang seines möglichen Weges. Eine Fortführung des Projekts sollte unbedingt regelmässige Tests in den Entwicklungsprozess integrieren. So kann fortlaufendes Feedback der Usergruppen eingeholt und in der Weiterentwicklung berücksichtigt werden.\\\\
Während der Implementation von Palim-Palim wurden diverse Eigenheiten und Möglichkeiten der verwendeten Technologien exploriert. Es zeigte sich, dass trotz bestehender Frameworks die Integration eines Videochats in ein webbasiertes Spiel nicht trivial ist. WebRTC empfiehlt sich nach wie vor als etablierter Standard. Es bietet sich aber an, ein entsprechendes Framework zu entwickeln, um ein WebRTC-Videochat möglichst einfach in bestehende oder auch neu entwickelte Mehrspieler-Games zu implementieren. Die Server-Infrastruktur für ein Videochat sollte dabei nicht vergessen werden. Sie ist zwar nicht enorm komplex, aber essenziel für die Funktionsweise des Videochats über das Web. Eine solche Infrastruktur könnte aber auch für mehrere Spiele gleichzeitig und direkt in Kombination mit einem Framework genutzt werden.

	\subsection{Mögliche Weiterentwicklungen}
	Das Videospiel Palim-Palim bietet noch viel Potenzial für Weiterentwicklungen. Mögliche Verbesserungen, Erweiterungen und Ideen sind in diesem Kapitel dokumentiert.\\\\
Um zwei Spielende in einem virtuellen «Raum» zu versammeln, wurde die Technologie socket.io verwendet, dies wird im Kapitel Technologien genauer erklärt. Dieses Konzept des virtuellen Raumes wird in Palim-Palim auch im Frontend verwendet. Für die Spielenden bietet dies einige Nachteile. Die betagten Testpersonen empfinden das Wort «Raum» und das Wählen einer Nummer als kalt und emotionslos. Sie wünschen sich einen Prozess, welcher näher an die Realität, statt an die Technik angelehnt ist. Durch die Moderierenden wurde während der Spieletests festgestellt, dass auch der Austausch dieser Raumnummer ein Problem darstellt. Die betagte Person, welche die Raumnummer wählen kann, muss diese auf einem zweiten Kommunikationskanal dem Enkelkind übermitteln. Eine mögliche alternative Umsetzung besteht darin, dass die Spielenden sich mit ihrer E-Mail-Adresse einloggen und sich auch durch diese verknüpfen. Im Hintergrund würde dann vom Server ein «Raum» mit einer zufälligen Raumnummer erstellt werden. Zusätzlich kann so auch das Problem umgangen werden, dass man aus Versehen mit einer fremden Person ein Spiel startet (wenn zwei betagte Personen kurz nacheinander die gleiche Raumnummer auswählen). Dieses Problem könnte jedoch auch mit einer zusätzlichen Station im Verbindungsaufbau gelöst werden, indem der Raumerstellende den Raumbeitretenden zuerst akzeptieren muss. Dafür muss sich der Raumbeitretende aber auch zuerst identifizieren, beispielsweise mit einem Login.\\\\
Die Abbildung mit den Erklärungen war für alle vier Testpersonen verwirrend, weil diese nicht interaktiv ist. Die Spielenden erwarten, dass die beschriebenen Aktionen bereits ausgeführt werden können. Deshalb bietet es sich an, die Erklärungen direkt in die erste Runde des Spiels einzubauen. Dabei würde beispielsweise beim Kind die Anweisung, einen Apfel auf den Tresen zu legen, als Pop up erscheinen und sobald die Aktion ausgeführt wurde, verschwindet das Pop-up beim Kind und es erschiene eine Bestätigungsmeldung, dass die Aktion korrekt ausgeführt wurde. Beim betagten Menschen würde anschliessend die Meldung erscheinen, den Apfel in den Einkaufskorb zu ziehen. Es besteht die Vermutung, dass durch diese Art der Spielerklärung den Spielenden der Start einfacher fällt. Denn diese können die von ihnen verlangten Aktionen ausprobieren und erhalten zusätzlich auch ein Feedback, dass die Aktion korrekt oder nicht ausgeführt wurde.\\\\
Die Spieletests haben gezeigt, dass die einseitige Implementation der Menüführung dafür sorgt, dass dem Kind während der Auswahl des Spiel- und Videomodus langweilig wird. Um dieses Problem zu beheben, könnte die Auswahl gemeinsam getroffen werden. Dafür müssten die Buttons, welche in der aktuellen Version von Palim-Palim aus Text bestehen, zusätzlich ein Bild beinhalten. Beide Spielende könnten ihren gewünschten Modus auswählen, würden visuell auch sehen, was der Mitspielende ausgewählt hat und sobald die Eingaben übereinstimmen, würde bei der betagten Person ein Bestätigungsbutton erscheinen. Falls die Auswahl nicht übereinstimmt, würde eine Info angezeigt, dass sich abgesprochen werden soll.\\\\
Betreffend Design und Usability gibt es ebenfalls noch einige Verbesserungsideen. Beispielsweise sollten die Verkaufsobjekte auf einem Regalboden platziert werden, um sie besser ins Spiel zu integrieren und sie nicht in der Luft schweben zu lassen. Der Hintergrund sollte gemäss den Erkenntnissen in den Spieletests überarbeitet werden. Entweder würde sich ein interaktiver Hintergrund anbieten oder es müssten genauere Recherchearbeiten betrieben werden, wie ein Hintergrund als solcher und nicht als Teil des Gameplays erkannt wird. Um im Videomodus 3 den Spielenden noch besser in die Spielszene zu integrieren, ist es angedacht, die Person auf dem Video freizustellen. Dafür wurde bereits Recherchearbeit betrieben und eine Implementation von Zhu und Oved des Frameworks TensorFlow [https://github.com/tensorflow/tfjs-models/tree/master/body-pix] als geeignet erachtet.\\\\
//todo weitere usability/design verbesserungen dokumentieren\\\\
Die Analyse des momentanen Standes von Palim-Palim wirft einige Fragen auf: Was soll passieren, wenn zu viele nicht auf der Einkaufsliste stehenden Elemente eingekauft wurden und dadurch kein Platz mehr im Einkaufskorb ist? Momentan werden die Objekte zur virtuellen Liste weiter hinzugefügt, aber nach sechs Objekten nicht mehr visuell im Einkaufskorb angezeigt. So kann das Spiel noch erfolgreich beendet werden. Eine mögliche Lösung ist es, dass das Spiel dann nicht bestanden werden kann und abbricht. Eine andere Idee ist es, dass diese Objekte dann wieder zurückgelegt werden können. Eine weitere Frage ist es, wie sich die Spielenden gegenseitig helfen können, beispielsweise wenn das Kind einen gewünschten Gegenstand nicht findet. Hier ist es denkbar, dass zuerst die betagte Person animiert wird, dem Kind den Gegenstand zu beschreiben und wenn das auch nicht erfolgreich ist, könnte die betagte Person ein Pop-up beim Kind auslösen, auf welchem das gewünschte Objekt ersichtlich ist.\\\\
Um die Story des Einkaufserlebnisses abzuschliessen ist eine Erweiterung des Gameplays, respektive das Hinzufügen eines zusätzlichen sekundären Gameplayloops, angedacht. Wie in Abb. xy beschrieben, soll dieser soll den Zahlungsprozess in Palim-Palim integrieren. Dieser würde nach dem primären Gameplayloop folgen, das heisst nachdem alle benötigten Objekte und eventuell noch zusätzliche sich im Einkaufskorb befinden, würde der Zahlungsprozess gestartet.\\\\
//todo Abbildung Gameplay-Loop Zahlung hinzufügen\\\\
Nicht nur der Zahlungsprozess ist eine mögliche Erweiterung des Spielerlebnisses. Die Frage tritt auf, wie die Motivation der Spielenden, das Spiel erneut zu spielen, gefördert werden kann. Welche Anreize können werden hierfür geschaffen? Drei Möglichkeiten dafür werden kurz beschrieben, viele weitere aber sind denkbar. Zunächst einmal ist bereits die komplette Vorarbeit für verschiedene Spielmodi bewältigt. Die offene Arbeit, um weitere Spielszenarien (zum Beispiel ein Gemüseladen oder eine Bäckerei) zu implementieren, besteht hierbei aus der Suche nach zusätzlichen Assets für die Verkaufsobjekte und der Gestaltung von unterschiedlichen Hintergründen. Diese Spielmodi bieten wiederum weitere Möglichkeiten, wie beispielsweise die Integration einer Waage für das Gemüse oder einer Tüte für die verschiedenen Brötchen. Zweitens könnten verschiedene Ziele zu einem interessanteren Spielerlebnis beitragen. Die betagte Person müsste statt konkreten Gegenständen Zutaten für einen Fruchtsalat oder Festartikel für eine Geburtstagsfeier einkaufen. Dadurch kann auch dem Wunsch der Testpersonen nach Selbstbestimmung der Einkaufsgegenstände Folge geleistet werden. Als dritte Möglichkeit wird die Implementation der Videomodi 4 und 5 beschrieben. Diese Modi ermöglichen die Interaktion mit dem Video des Gegenübers, wodurch im Speziellen das Spielerlebnis des Kindes attraktiver wird. Zwei Vorschläge für mögliche Szenarien sind das Aufsetzen von Verkleidungsgegenständen oder das Beschmieren mit Eis. Diese Ideen sind im Anhang als Designentwurf unter xyz zu finden.\\\\
//todo einige punkte evtl. mit quellen untermauern («Die Kunst des Game Designs»?)


technisch\\
\begin{itemize}
		\item Extrahierung der Video-/Room-Komponente in ein Framework
		\item Gamesync: wenn beide user items draggen können sie sich duplizieren
		\item Gameover ist noch nicht sauber implementiert
		\item Feedback, dass der gewählte Raum bereits besetzt ist
		\item Verlassen des Raums und Zurückkehren funktioniert noch nicht -> wenn Verbindung abbricht, muss ein neuer Raum eröffnet werden
	\end{itemize}

Eine weitere Möglichkeit das Spiel aufregender zu gestalten ist die Implementierung einer Physik Engine.\\\\
//todo Dani: Vorarbeit dokumentieren. Versuche mit ammojs und headless Möglichkeiten auf Server. \\\\


\chapter{Literaturverzeichnis}
\printbibliography[heading=none]

\appendix

\chapter{Ehrlichkeitserklärung}

Hiermit erklären wir, die vorliegende Bachelorthesis selbständig, ohne Hilfe Dritter und nur unter Benutzung der angegebenen Quellen verfasst zu haben.\\\\
\begin{center}
\begin{tabular}{p{5cm}p{5cm}}
\centering
\includegraphics[width=4cm]{signature_dani}
Daniel Obrist\\
Brugg, 20.08.2021	&
\centering
\includegraphics[width=4cm]{signature_severin}
Severin Peyer\\
Brugg, 20.08.2021
\end{tabular}
\end{center}


\chapter{Testprotokolle, weitere Spielkonzepte/Ideenfindung, Readme des Repos... }


\section{Weitere Ideen für intergenerationelle Videospiele mit integriertem Videochat}

\paragraph{Bombe entschärfen} Die betagte Person sieht nur die Anleitung und nicht die Bombe, das Kind sieht nur die Bombe und kann mit dieser interagieren. Zusammen soll die Bombe dann entschärft werden. Das Spielprinzip ist auch für andere Use Cases denkbar.

\paragraph{«Brain Out»} Viele kleine Levels mit kniffligen Denksportaufgaben. Diese könnten zusätzlich mit Sensoren des Smartphones (Magnetometer, Rotationssensor...) und nur gemeinsam von beiden Mitspielenden gelöst werden.

\paragraph{Dr. Bibber mit Filter} Der eine Spielende muss virtuelle Objekte aus dem Gesicht/Video des anderen herausoperieren, indem das Objekt beispielsweise einer Linie entlang in einen Zielbereich gezogen wird. Die Person, welche den Patient spielt, muss dabei möglichst ruhig halten.

\paragraph{«Drecksau»} Simples Kartenspiel mit der «Realität entsprechenden» Kartenhandlungen. Beispiel für so ein Spiel ist «Drecksau». Beispielsweise wäscht eine Regenkarte das dreckige Schweinchen wieder sauber. [https://www.spiel-des-jahres.de/spiele/drecksau/]

\paragraph{Escape Room} Lorem ipsum dolor sit amet.

\paragraph{FaceAPI} Lorem ipsum dolor sit amet.

\paragraph{Filter-Mania} Lorem ipsum dolor sit amet.

\paragraph{Filter-Tabu} Lorem ipsum dolor sit amet.

\paragraph{Gesicht zusammensetzen} Lorem ipsum dolor sit amet.

\paragraph{Gesichtsberührungen} Lorem ipsum dolor sit amet.

\paragraph{Grössen-/Gewichtsverhältnisse} Lorem ipsum dolor sit amet.

\paragraph{Ich sehe was, was du nicht siehst} Lorem ipsum dolor sit amet.

\paragraph{Karikatur (Körperknickbilder)} Lorem ipsum dolor sit amet.

\paragraph{Kooperatives Tetris} Lorem ipsum dolor sit amet.

\paragraph{Labyrinth} Lorem ipsum dolor sit amet.

\paragraph{Live Puzzle} Lorem ipsum dolor sit amet.

\paragraph{Montagsmaler} Lorem ipsum dolor sit amet.

\paragraph{Moorhuhn} Lorem ipsum dolor sit amet.

\paragraph{Pokerface} Lorem ipsum dolor sit amet.

\paragraph{Schere - Stein - Papier} Lorem ipsum dolor sit amet.

\paragraph{Schweizer Reise} Lorem ipsum dolor sit amet.

\paragraph{Turm bauen} Lorem ipsum dolor sit amet.

\paragraph{Virtuelles Bilderbuch} Lorem ipsum dolor sit amet.

\paragraph{Wimmelbild} Lorem ipsum dolor sit amet.

\paragraph{Zaubersprüche} Lorem ipsum dolor sit amet.

\end{document}