\documentclass[12pt,a4paper]{report}

% -- Imports biblatex and defines bib file --
\usepackage[backend=bibtex,style=numeric,language=german,sorting=none]{biblatex}
\addbibresource{references.bib}
% http://tug.ctan.org/info/biblatex-cheatsheet/biblatex-cheatsheet.pdf

% -- Language --
\usepackage[utf8]{inputenc}
\usepackage[german]{babel}

% -- Images --
\usepackage{graphicx}
\graphicspath{ {./images/} }

% -- Continuous figure numbering --
\usepackage{chngcntr}
\counterwithout{figure}{chapter}

% -- Code blocks --
\usepackage{listings}
\usepackage{color}
\definecolor{lightgray}{rgb}{.9,.9,.9}
\definecolor{darkgray}{rgb}{.4,.4,.4}
\definecolor{purple}{rgb}{0.65, 0.12, 0.82}

% -- Code blocks Stlye --
\lstdefinelanguage{JavaScript}{
  keywords={typeof, new, true, false, catch, function, return, null, catch, switch, var, if, in, while, do, else, case, break},
  keywordstyle=\color{blue}\bfseries,
  ndkeywords={class, export, boolean, throw, implements, import, this},
  ndkeywordstyle=\color{darkgray}\bfseries,
  identifierstyle=\color{black},
  sensitive=false,
  comment=[l]{//},
  morecomment=[s]{/*}{*/},
  commentstyle=\color{purple}\ttfamily,
  stringstyle=\color{red}\ttfamily,
  morestring=[b]',
  morestring=[b]"
}
\lstset{
   language=JavaScript,
   backgroundcolor=\color{lightgray},
   extendedchars=true,
   basicstyle=\footnotesize\ttfamily,
   showstringspaces=false,
   showspaces=false,
   numbers=left,
   numberstyle=\footnotesize,
   numbersep=9pt,
   tabsize=2,
   breaklines=true,
   showtabs=false,
   captionpos=b
}

\newcommand{\paragraphwithnewline}[1]{\paragraph{#1}\mbox{}\\}

\begin{document}


\begin{titlepage}
\paragraph{Titelblatt}
\end{titlepage}

\chapter*{Projektinformationen}
Titel: Palim-Palim\\
Projektnummer: 21FS\_I4DS08

\paragraphwithnewline{Projekt-Team}
Daniel Obrist, 8iCbb\\
daniel.obrist@students.fhnw.ch\\\\
Severin Peyer, 8iCbb\\
severin.peyer@students.fhnw.ch

\paragraphwithnewline{Auftraggeber und Betreuung FHNW}
Marco Soldati\\
Fachhochschule Nordwestschweiz FHNW\\
Hochschule für Technik\\
Bahnhofstrasse 6\\
CH-5210 Windisch\\
+41 56 202 77 31\\
marco.soldati@fhnw.ch\\\\
Tabea Iseli\\
Fachhochschule Nordwestschweiz FHNW\\
Hochschule für Technik\\
Bahnhofstrasse 6\\
CH-5210 Windisch\\
+41 56 202 86 53\\
tabea.iseli@fhnw.ch\\

\paragraphwithnewline{Zeitbudget}
Das Projekt wird im Rahmen des Frühlingssemesters 2021 durchgeführt. Nominell sind für das Projekt 360~Arbeitsstunden pro Teammitglied veranschlagt. 

\paragraphwithnewline{Wichtigste Daten}
Beginn:	22. Februar 2021\\ 
Ende:	20. August 2021 \\

\begin{abstract}	
	\begin{itemize}
 		\item Ergebnisse aus den Spieletests, der Entwicklung und der Literaturrechereche zusammengefasst erläutern.
 		\item Erst am Schluss schreiben!aa
 		\item Wichtig für den ersten Eindruck
	\end{itemize}
\end{abstract}


\tableofcontents

\break

\chapter{Einleitung}
\paragraph{Teil 1 / Was wurde erreicht?}
\emph{\begin{itemize}
\color{blue}
 \item Beschreibung des Videospiels Palim-Palim (inkl. Screenshot)
 \item Aufstellung der Forschungsfragen und den Erkenntnissen
\end{itemize}}

\paragraph{Teil 2 / Warum wurde es gemacht?}
\emph{\begin{itemize}
\color{blue}
 \item Ausgangslage inkl. Forschungsstand
 \item Relevanz der Problemstellung
 \item Was ist das Umfeld?
\end{itemize}}

\paragraph{Teil 3 / Wie wurde es gemacht?}
\emph{\begin{itemize}
\color{blue}
 \item Grobe Beschreibung der angewendeten Methodik
 \begin{itemize}
 	\item Spielentwicklung (Architektur, Technologien)
 	\item Spieletests (Methoden)
 \end{itemize}
\end{itemize}}
 
\paragraph{Teil 4}
\emph{\begin{itemize}
\color{blue}
	\item Aufbau des Dokuments und Überleitung in den theoretischen Teil
\end{itemize}}

Palim-Palim ist ein interaktives Multiplayer-Videospiel für Kinder und betagte Menschen. Zwei Spielende können damit Kinder-Kaufladen in einer virtuellen Umgebung spielen. Es verfügt über einen Video-Chat als inhärente Game-Mechanik, um die Kommunikation zwischen den Spielerinnen und Spielern zu fördern.\\\\
Mit dem Spiel wird untersucht, welchen Einfluss ein Video-Chat in Videospielen auf die User Experience und die Kommunikation zwischen den Spielenden hat. Im Speziellen werden die folgenden Fragestellungen untersucht:
\begin{itemize}
	\item Welche Wirkung hat ein integrierter Video-Chat auf das Spielerlebnis von betagten Menschen?
	\item Welche Wirkung hat ein integrierter Video-Chat auf das Spielerlebnis von Kindern? 
	\item Welche Auswirkungen hat die Art und Weise, wie der Video-Chat in das Spiel integriert ist, auf das Spielerlebnis? 
	\item Welche Wirkung hat ein integrierter Video-Chat auf die Kommunikation zwischen den Spielenden?
	\item Wird ein integrierter Video-Chat aktiv als Kommunikationsmittel zur Bewältigung von Spielaufgaben genutzt?
	\item Welche Auswirkungen hat die Art und Weise, wie der Video-Chat in das Spiel integriert ist, auf die Förderung der Kommunikation?
\end{itemize}

\paragraph{\emph{\color{blue} Erkenntnisse}}

Video-Chats werden heutzutage schon von vielen Familien benutzt, um mit Ihren Verwandten zu kommunizieren. Auch in der Kommunikation mit den Grosseltern wird dabei immer mehr auf Video-Telefonie gesetzt. Dies gilt besonders für Zeiten, in denen Besuche im Alters- oder Pflegeheim auf Grund kursierenden Viren wie Corona schwierig oder unmöglich werden. Betagte Menschen schätzen und nutzen diese moderne Art der Kommunikation mit der Familie auch immer mehr – besonders weil es öfters auf ältere Benutzergruppen angepasste Angebote gibt \cite{glaab_silver_2015}.\\\\
			Allerdings hat die Video-Telefonie immer noch Grenzen, welche die Interaktionen unnatürlich und teilweise entfremdend wirken lassen. Vor allem für Kinder ist es schwierig, Gesprächsthemen und Kommunikationswege zu finden, die sich so lustig und verbindend anfühlen, wie die Zeit mit der Grossmutter oder dem Grossvater im echten Leben. Oft sind Kinder vom Gespräch schnell gelangweilt – sie würden lieber etwas spielen \cite{tulloch_7_2020}. Spielen kann ein Mittel sein, um Kinder besser einzubeziehen und die Interaktion mit ihnen zu unterstützen, wie bisherige Studien zu dem Thema Video-Calls mit Eltern und Kindern zeigen \cite{follmer_video_2010}.\\\\
			Ergänzend zum Video-Telefonie-Aspekt gibt es bereits viel Literatur bezüglich generationenübergreifender Computerspiele \cite{chua_lets_2013}, \cite{de_la_hera_benefits_2017}, \cite{soldati_create_2020}. Erkenntnisse aus einer Studie von Derboven et al \cite{derboven_designing_2012} deuten ausserdem drauf hin, dass in einem Multiplayer-Videospiel die zusätzliche Kommunikationsfunktionalität durch einen Video-Chat oft sowohl von älteren als auch von jüngeren Personen begrüsst wird. Allerdings bietet die Forschung bisher keine detaillierte Studie mit Kindern und betagten Personen in diesem Zusammenhang.\\\\
			Am Institut für Data Science (I4DS) der Fachhochschule Nordwestschweiz wird im Rahmen des Projekts Myosotis schon seit einigen Jahren daran gearbeitet, Video-Spiele zu entwickeln, welche in unterhaltsamer Weise die soziale Interaktion zwischen betagten Menschen und ihren Angehörigen unterstützen \cite{soldati_fhnw_2015}. Dabei wurden schon etliche Spiele umgesetzt und getestet \cite{soldati_create_2020}. Mit einer Integration eines Video-Chats in ein Spiel hat sich jedoch bisher noch kein Team explizit auseinandergesetzt. Palim-Palim schliesst diese Lücke und zeigt wertvolle Erkenntnisse über die Kombination von Video-Chats und Video-Spielen mit Kindern und betagten Personen.\\\\
			Um die formulierten Fragen zu beantworten, wurden mehrere Varianten eines Videospiels implementiert, in welchen ein Video-Stream zu einem Teil des Spiels wird. Anschliessend wurden mit allen Varianten Spieltests durchgeführt, um herauszufinden, ob die Integration eines Video-Chats in einem Video-Spiel einen positiven oder negativen Einfluss auf die User Experience sowie die Kommunikation zwischen den Spielenden hat. Die Arbeit fokussiert sich speziell auch auf die Art und Weise, wie ein Video-Stream in ein Spiel integriert werden kann.

\paragraph{\emph{\color{blue} Grobe Systemarchitektur, verwendete Methoden und Konzepte}}

 

% -- Theoretischer Teil --
\chapter{Umfeldanalyse und Zielgruppe}
\emph{
	\begin{itemize}
		\color{blue}
		\item Beschreibung des Umfelds / Anwendungsdomäne (betagte Personen und Kinder)
	\end{itemize}
}
Die verwendeten Begriffe Kind und betagte Person werden dabei im Rahmen des Projekts wie folgt definiert:\\
\subparagraph{Kind} Person zwischen fünf und acht Jahren. 
\subparagraph{Betagte Person} Person ab einem Alter von 65 Jahren ohne grössere mentale Beeinträchtigung. 
Diese beiden Personengruppen stellen zugleich die Zielgruppen für die durchzuführenden Spieltests dar.
\chapter{Intergenerationelles Spielen}
\emph{
	\begin{itemize}
		\color{blue}
		\item Forschungsstand (Welche Methoden/Ansätze werden angewendet?)
		\item Bisherige Erkenntnisse zu intergenerationellem Spielen
	\end{itemize}
}
\chapter{Videochats in Videospielen}
\emph{
	\begin{itemize}
		\color{blue}
		\item Forschungsstand (Welche Methoden/Ansätze werden angewendet?)
		\item Bisherige Erkenntnisse zu Videochats in Videospielen
	\end{itemize}
}
\chapter{Aufstellung der Forschungsfragen}
\emph{
	\begin{itemize}
		\color{blue}
		\item Lücken der bisherigen Forschung
		\item Aufstellung der Forschungsfragen
	\end{itemize}
}
Aus der in der Einleitung formulierten Ausgangslage stellt sich folgende Game-Design-Frage: Wie lässt sich ein Videospiel für unterschiedliche Altersgruppen ansprechend gestalten? Da diese Frage schon in vielen Studien bezüglich intergenerationellem Spieledesign behandelt wurde \cite{chua_lets_2013}, \cite{de_la_hera_benefits_2017}, \cite{soldati_create_2020}, konzentriert sich das Projekt Palim-Palim in seiner Aufgabenstellung speziell auf den Aspekt der Video-Telefonie in Videospielen. Dies ist besonders interessant, da die Kombination von Videospielen mit Video-Telefonie eine grosse Möglichkeit bietet, Video-Chats für intergenerationelle Altersgruppen attraktiver zu gestalten.\\

\section{Hauptfragestellung}
Mit Palim-Palim soll herausgefunden werden, wie Videospiele und Video-Telefonie kombiniert werden können. Zusätzlich sollen für diese Kombinationen die Auswirkungen auf die Interaktionen zwischen betagten Menschen und Kindern erforscht werden. Deshalb befasst sich das Projekt mit folgender übergeordneter Fragestellung:\\
			\textbf{Wie lässt sich Video-Telefonie mit Video-Spielen kombinieren, damit zwei Personen (Kind und betagte Person) übers Netz miteinander spielen und sich gleichzeitig unterhalten können?}

\section{Einzelfragen im thematischen Zusammenhang}
Um die übergeordnete Aufgabenstellung messbar zu machen, wird sich die begleitende Thesis von PalimPalim mit spezifischen Fragestellungen zu den Themen User Experience (siehe 2.2.1) und Kommunikation (siehe 2.2.2) auseinandersetzen. Die dazu formulierten Hypothesen können dabei durch die Auswertung der Resultate aus den Spieltests verifiziert oder widerlegt werden.

\subsection{Spezifische Fragestellung zur User Experience:}

\paragraph{1. Wie beeinflusst die Einbindung von Video-Telefonie die User Experience in Videospielen?}
	\subparagraph{Fragestellung 1a:} Welche Wirkung hat ein integrierter Video-Chat auf das Spielerlebnis von betagten Menschen?
	\subparagraph{Hypothese 1a:} Ein integrierter Video-Chat hat eine positive Wirkung auf das Spielerlebnis von betagten Menschen.
 
	\subparagraph{Fragestellung 1b:} Welche Wirkung hat ein integrierter Video-Chat auf das Spielerlebnis von Kindern?
	\subparagraph{Hypothese 1b:} Ein integrierter Video-Chat hat eine positive Wirkung auf das Spielerlebnis von Kindern.

	\subparagraph{Fragestellung 1c:} Welche Auswirkungen hat die Art und Weise, wie der Video-Chat in das Spiel integriert ist, auf das Spielerlebnis?
	\subparagraph{Hypothese 1c:} Je stärker der Video-Chat ins Gameplay integriert ist, desto positiver ist das Spielerlebnis.


\subsection{Fragestellung zur Kommunikation:}

\paragraph{2. Wie beeinflusst die Einbindung von Video-Telefonie die Kommunikation in Videospielen?}
	\subparagraph{Fragestellung 2a:} Welche Wirkung hat ein integrierter Video-Chat auf die Kommunikation zwischen den Spielenden? 
	\subparagraph{Hypothese 2a:} Ein im Spiel integrierter Video-Chat fördert die Kommunikation zwischen den Spielenden. 

	\subparagraph{Fragestellung 2b:} Wird ein integrierter Video-Chat aktiv als Kommunikationsmittel zur Bewältigung von Spielaufgaben genutzt? 
	\subparagraph{Hypothese 2b:} Der Video-Chat wird aktiv als Kommunikationsmittel zur Bewältigung der Spielaufgabe verwendet. 

	\subparagraph{Fragestellung 2c:} Welche Auswirkungen hat die Art und Weise, wie der Video-Chat in das Spiel integriert ist, auf die Förderung der Kommunikation? 
	\subparagraph{Hypothese 2c:} Je stärker der Video-Chat ins Spiel integriert ist, desto angeregter ist der Austausch zwischen den Spielenden. 
	
\chapter{Methoden}
\emph{
	\color{blue}
	In Palim-Palim verwendete Methoden
	\begin{itemize}
		\item Methode der Spieletest
		\item Methode der Spieletest-Auswertung
	\end{itemize}}

% -- Praktischer Teil --
\chapter{Spieletests und Resultate}
	\emph{\section{Resultate}
		\begin{itemize}
			\color{blue}
			\item Ergebnisse
			\item Beantwortung der aufgestellten Forschungsfragen
		\end{itemize}}
	\section{Planung und Durchführung der Spieletests}
		\emph{\begin{itemize}
			\color{blue}
			\item Organisation der Spieletests (Aufbau, Ablauf, Testpersonen, Testszenarien, Testumgebung)
			\item Beobachtungen
		\end{itemize}}
	
Die Überprüfung der im Kapitel 2 aufgestellten Hypothesen wird durch Spieltests mit den unterschiedlichen Spielvarianten geschehen. Das Ziel dabei ist es, mit den Methoden Beobachtung und Befragung die User Experience sowie die Spielenden-Kommunikation messbar zu machen. Das Beobachten dient einerseits dazu, sofortige und auch unterbewusste Reaktionen bei den Testpersonen zu erkennen. Andererseits möchten wir mittels Befragungen qualitatives Feedback zu den einzelnen Spielvarianten einholen. Dazu wird für die Spieltests vorab ein passender Fragenkatalog erarbeitet, bzw. ein entsprechendes Beobachtungs-Protokoll vorbereitet und geführt.\\\\
			In der Projektwoche 12 wird mit einem funktionalen Prototyp ein erstes Mal Feedback zu den Grundmechaniken eingeholt. Diese erste Standortbestimmung wird mit Bekannten durchgeführt.\\\\
			Für die tatsächlichen Spieltests werden sechs Test-Paare benötigt, wobei ein Test-Paar sich aus einer betagten Person sowie einem Kind zusammensetzt. In der Woche 19 und 20 wird mit allen Paaren getestet. Um die Testpersonen zu finden, wird mit der Organisation terzStiftung \cite{noauthor_terzstiftung_nodate} zusammengearbeitet. Mithilfe eines Newsletters werden die Mitglieder der terzStiftung für ein Testing angefragt. Die angefragten betagten Menschen sind mindestens 65 Jahre alt und weisen eine höhere Technik-Affinität auf als ein Abbild der Gesamtheit dieser Altersgruppe.\\\\
			Nach Möglichkeit (Situation SARS-CoV-2) werden die Spieltests bei den Probanden zuhause, an der FHNW in Brugg, an einem öffentlichen Ort oder remote durchgeführt (absteigende Priorität). Bei den drei erstgenannten Varianten würde das Projektteam die Beobachtungen direkt vor Ort vornehmen, sollten die Tests remote durchgeführt werden müssen, würden die Projektmitglieder sich online zuschalten. Die Probanden müssen nicht zwingend ein Tablet besitzen, dies kann Ihnen von uns zur Verfügung gestellt werden. Jedoch muss bereits vor den Tests ein solches benutzt worden sein.	
	
\chapter{Gamedesign}
	\section{Spielkonzept}
		Das dem Videospiel zugrunde liegende Spielkonzept lehnt sich dem Kinder-Kaufladen-Spielprinzip aus dem echten Leben an. Das Kind und die betagte Person schlüpfen dabei in die entsprechenden Rollen von Verkaufspersonal und Kundschaft in einem virtuellen Einkaufsladen. Die Spieler*innen sehen dabei jeweils einen gemeinsamen Tresen aus der Perspektive ihrer Rolle (siehe Abbildung 1). Beide Seiten können Objekte auf dem Tresen platzieren, beziehungsweise Objekte vom Tresen wegziehen. Zusätzlich besteht die Möglichkeit, die Objekte auf den Video-Stream des Gegenübers zu ziehen. Durch all diese Aktionen können je nach Objekt diverse Animationen, Filter-Effekte und Gameplay-Events ausgelöst werden.
		\subsection{Gameplay-Loops}
		Durch die Aufteilung in die beiden Rollen lassen sich für beide Spieler*innen verschiedene Aktionen und somit ein unterschiedliches Gameplay definieren. Diese Art der Unterteilung eignet sich hervorragend, um das Gameplay jeweils auf die spezifischen Anforderungen von Kindern bzw. betagten Personen anpassen zu können.\\\\
		Der Gameplay-Loop ist in der nachfolgenden Abbildung 2 skizziert. Die Haupt-Interaktionen zwischen den beiden Spieler*innen bestehen aus der Kommunikation der Einkaufsliste und dem Hin- und Herreichen von Objekten auf der gemeinsamen Tresen-Fläche.\\\\
		\emph{\color{blue} Gameplay-Loops}\\\\
		Dieses übergeordnete Gameplay soll als Leitfaden für den Spielverlauf dienen. Allerdings bietet das Spiel durch die gemeinsame Tresen-Fläche und dem interaktiven Video-Stream bewusst Möglichkeiten für sonstige Interaktionen mit dem Gegenüber. Damit soll für die Benutzerinnen und Benutzer ein gewisser Freiraum entstehen, in welchem sie den Verlauf des Spiels auf spielerische Art und Weise beeinflussen können. Dies soll das Spiel besonders für das Kind interessanter machen, um eine aktive und lustige Kommunikation mit der betagten Person zu fördern. 
	\section{Spielvarianten}
	Damit der Einfluss der Video-Telefonie als Teil des Spiels messbar wird, werden unterschiedliche Varianten des Spiels implementiert und getestet: 
\begin{enumerate}
	\item Spiel ohne Video-Chat 
	\item Spiel mit Video-Chat, aber getrennt von der Spiel-Szene 
	\item Spiel mit Video-Chat, in die Spiel-Umgebung integriert 
	\item Spiel mit Video-Chat, in die Spiel-Umgebung integriert und optionale Interaktionsmöglichkeiten mit dem Video-Stream 
	\item Spiel mit Video-Chat, speziell in die Spiel-Umgebung integriert und zum Spielerfolg notwendige Interaktionsmöglichkeiten mit dem Video-Stream 
\end{enumerate}
\emph{\color{blue}Tabelle Spielvarianten}
	\section{Design und Usability}
	
\chapter{Implementation}
	\section{Technologien}
			\subsection{WebRTC}
WebRTC ist ein Open-Source-Projekt, das die Echtzeitkommunikation von Audio, Video und Daten in Web- und nativen Anwendungen ermöglicht \cite{noauthor_webrtc_2011}. Die Technologie ist in allen modernen Browsern sowie auf nativen Clients für alle wichtigen Plattformen verfügbar, ist eine Empfehlung des World Wide Web Consortium (W3C) \cite{noauthor_webrtc_2021} und ein Standard der Internet Engineering Task Force (IETF) \cite{noauthor_support_2021}. Dabei wird zwischen zwei Browsern eine Peer-To-Peer-Verbindung aufgebaut, worüber die Daten gestreamt werden. Auf der Peer-To-Peer-Verbindung lassen sich auch eigene Daten-Kanäle erstellen, die z.B. zur Übertragung von Text-Nachrichten, Positions-Daten von Game-Objekten oder sogar Files zwischen den Peers verwendet werden können \cite{noauthor_webrtc_nodate}.
			\subsubsection{Signaling}
			WebRTC verwendet die RTCPeerConnection JavaScript API, um Streaming-Daten direkt zwischen Browsern zu kommunizieren \cite{noauthor_rtcpeerconnection_nodate}. Zusätzlich wird aber auch einen Mechanismus benötigt, welcher die Kommunikation der beiden Peers koordiniert und Kontrollnachrichten sendet. Dieser Prozess wird als \textit{Signaling} bezeichnet. Signaling-Methoden und -protokolle sind von WebRTC nicht spezifiziert und können je nach Anwendungsfall entsprechend gewählt werden. In Palim-Palim wurde die JavaScript-Library Socket.io für die Signalisierung verwendt. \\		
Socket.io erlaubt eine einfache und bidirektionale Kommunkation zwischen den Clients und dem Signaling-Server. Mit seinem integrierten Room-Konzept eignet sich Socket.io zudem sehr gut für eine Video-Chat-App. Eine Node.js Server-Applikation fungiert als Signaling-Server für Palim-Palim. Der Server hat dabei folgende zwei Aufgaben: er dient als Message Relay und verwaltet alle Videochat-Räume. \\\\
Die Funktionaliät als Message Relay ist wichtig, da sich die Peers vor dem Verbindungsaufbau noch nicht kennen. Ein Signaling-Server, welcher beiden Cleints bekannt ist, wird benötigt. So können initiale Informationen von und zu den Clients gsendet werden, damit diese untereinander eine WebRTC-Peer-Verbindung aufbauen können:
			\begin{lstlisting}
socket.on('message', function (message) {
  socket.to(room).emit('message', message);
});
\end{lstlisting}
Zweitens verwaltet der Signaling-Server alle WebRTC-Videochat-'Räume'. Tritt man zum Beispiel einem Raum bei, so überprüft der Server, ob die Raumkapazität schon erreicht wurde und sendet entsprechende Antworten an den Client:
\begin{lstlisting}
if (numClients === 0) {
  socket.join(room);
  socket.emit('created', room, socket.id);
} else if (numClients === 1) {
  socket.join(room);
  socket.emit('joined', room, socket.id);
  io.sockets.in(room).emit('ready');
} else { // max two clients
  socket.emit('full', room);
}
\end{lstlisting}
Wie im Code ersichtlich ist, erlaubt unsere Applikation maximal zwei Peers in einem Raum. Ist man der erste Spieler, erhält der Client die Anwtort 'created' vom Signaling-Server. Ist bereits ein Client in dem Raum, wird ein 'joined' zurückgegeben. Wurde die Raum-Kapazität erreicht, sendet der Palim-Palim Server dem Client ein 'full'. Diese Nachrichten werden dann Client-seitig entsprechend behandelt.


			\subsubsection{STUN und TURN}			

			WebRTC ist grundsätzlich so konzipiert, dass es Peer-to-Peer funktioniert. Benutzer können sich also auf dem direktesten Weg verbinden. Die Technologie ist jedoch auch bewusst darauf ausgelegt, mit realen Netzwerken zurechtzukommen: Client-Anwendungen müssen NAT-Gateways und Firewalls überwinden, und Peer-to-Peer-Netzwerke benötigen Fallbacks, falls die direkte Verbindung ausfällt. Als Teil dieses Prozesses verwendet die WebRTC-API sogenannte \textit{STUN-Server}, um die IP-Adresse ihres Computers zu ermitteln, und \textit{TURN-Server} (Traversal Using Relay NAT), welche als dedizierte Relay-Server fungieren, falls die Peer-to-Peer-Kommunikation fehlschlägt \cite{noauthor_turn-server_nodate}.
			\begin{figure}[h!]
			\includegraphics[width=\textwidth]{WebRTC_nat_stun_firewall_turn_black}
			\caption[Caption for LOF]{STUN, TURN, und Signalisierung in WebRTC (eigene Darstellung).}
			\end{figure}
			
			Palim-Palim verwendet als STUN-Server öffentlich verfügbare Server von Google. Da diese STUN-Server nur zur Ermittlung der eigenen öffentlichen IP-Addresse dienen, lässt sich hier ohne grosse Auswirkungen auf den Datenschutz oder die Sicherheit öffentlich gehostete Server verwenden. \\	
			
			Als Fallback-Verbindungen wird allerdings beim produktiven Betrieb einer WebRTC-Applikation unbedingt ein privater TURN-Server mit einer öffentlich sichtbaren IP-Adresse benötigt. Es gibt keine allgemein verfügbaren TURN-Server, da der Netzwerkverkehr über so einen Server sehr stark ansteigen kann. Daher lohnt es sich für niemanden, fremden Verkehr über seinen TURN-Server zuzulassen. Es ist also ebenfalls von Vorteil, seinen TURN-Server mit einer ensprechenden Authentifizierung zu versehen.
Palim-Palim verwendet einen auf SwitchEngines gehosteten TURN-Server. Dieser Linux-Server verwendet Coturn, eine Open-Source Implementierung des TURN-Protokolls \cite{noauthor_coturncoturn_2021}.  \\	

Die STUN- und TURN-Server-Adressen müssen bei dem Aufbau der Peer-Connection vom Client an den Signaling-Server übermittelt werden. Dazu übermittelt jeder Palim-Palim-Client beim Verbindungsaufbau folgende pcConfig-Werte an den Server:
			
			\begin{lstlisting}
			this.peerConnectionConfig = {
            'iceServers': [
                {
                  'urls': 'stun:stun.l.google.com:19302'
                },
                {
                  'urls': 'turn:86.119.43.130:3478',
                  'credential': 'ZV78Nz75/3sk<:d.[.#m3;dch4v(2+RdvS9',
                  'username': 'palimpalim'
                }
            ]
        };
			\end{lstlisting}
			
			In etwa 82 Prozent der Fälle ist die Peer-To-Peer-Verbindung stabil genug, und die TURN-Server werden überhaupt nicht benötigt. Jedoch müssen sie trotzdem bei jeder Verbinung angegeben werden. TODO more infos\cite{callstats_why_nodate} https://www.callstats.io/blog/2017/10/26/webrtc-product-turn-server
		\subsubsection{Sicherheit}
		Verschlüsselung ist für alle WebRTC-Komponenten obligatorisch, und seine JavaScript-APIs können nur von sicheren Quellen (HTTPS oder localhost) aus verwendet werden. Video, Audio oder Daten zwischen zwei Peers werden mit Secure Real Time Protocol (SRTP) verschlüsselt. SRTP verschlüsselt die Session, sodass niemand die Nachricht ohne die richtigen Verschlüsselungsschlüssel entschlüsseln kann. Unverschlüsselte Kommunikation gibt es in WebRTC nicht. Schlüssel werden ebenfalls nicht über den Signaling Server ausgetauscht, sonder nur direkt zwischen den Peers. Dazu kommt HTTPS zum Signaling-Server und die im Browser und inn Betriebssystem vorgegebenen Sicherheiten. \cite{noauthor_study_nodate}
		[https://webrtc-security.github.io/]
		[https://webrtc-security.github.io/]
		TODO<Für unser Projekt noch definieren und ausformulieren gemäss Quellen>
		
		
		\newpage
	\section{Architektur}
	\subsection{Gampeplay-Synchronisation}
	Um die Interaktionen der beiden Spieler zu synchronisieren, wird normalerweise ein Server verwendet, der die Inputs der Spieler entgegennimmt, und als zentrale Authorität den Zustand des Spiels bestimmt. Bei Palim-Palim wurde dies aber anders gelöst. Das Spiel nutzt seine Peer-To-Peer-Funktionalität aus und kreiert neben dem Video- und Audiostream auch einen spezifischen DataChannel für Game-Updates. Dieser Channel wird genutzt, um die Objekte beider Szenen zu synchronisieren.
	

	
Der DataChannel überträgt JSON-Strings via UDP. UDP als Protokoll ist sehr schnell und deshalb sind die Änderungen auch fast ohne Verzögerung beim Peer sichtbar. Die Wahl des Protokolls kann beim Erstellen des DataChannels gewählt werden [Code Erstllung DataChannel]. Eine weitere Eigenschaft von UDP ist, dass die Reihenfolge der Übertragung nicht garantiert ist - im Gegensatz zu TCP z.B. Da die Game-Updates sehr oft geschickt werden, ist es in diesem Fall egal, in welcher Reihenfolge die Nachrichten ankommen. (Man muss sich das für jeden Fall überlegen...)
\begin{lstlisting}
const dataChannel = peerConnection.createDataChannel('gameUpdates', {
  ordered: false,
  id: room
  });
dataChannel.onmessage = handleReceiveMessage;
dataChannel.onerror = handleError;
dataChannel.onopen = handleDataChannelStatusChange;
dataChannel.onclose = handleDataChannelStatusChange;
\end{lstlisting}

Diese Art der Gameplay-Synchronisation erlaubt es den Clients, total unabhängig von einem Server zu spielen. Da der Server nur fürs Signaling benutzt wird, ist auch eine grosse Skalierung der Spieleranzahl denkbar. Der Server muss die Spieler nur inital vermitteln, was keine grosse Sache ist. Alle anderen Berechnungen sowie der Abgleich der Spielwelt werden auf den Clients direkt vorgenommen. Da kein Umweg über einen Server genommen werden muss, ist auch die Latenz niedrig. (Einziges Manko: es gibt keine zentrale Authorität. Das heisst Spieler könnten den Client-Code manipulieren und sich unfaire Vorteile verschaffen. Da Palim-Palim und sein Publikum aber nicht kompetitiv sind, ist diese Gefahr des Cheatings vernachlässigbar.)

	\subsection{3D-Objekte laden}
	(Backend/Frontent Kapitel)? (vieleicht als subsection von Frontend)? 
	-- Aufbau einer 3D-Szene in Three.js erklären
	-- Beschaffenheit und laden von GLTF-Files in Palim-Palim erklären 
	https://discoverthreejs.com/book/first-steps/load-models/
	\section{Sicherheit}
	\section{Testing}
	\section{Deployment und Betrieb (Anhang?)}			
			
			
\chapter{Fazit}
	\begin{itemize}
		\item Zusammenfassung des Erreichten / Zielerreichung
		\item Zentrale Erkenntnisse
		\item Reflektion
		\item Mögliche Weiterentwicklungen (bezogen auf die Software)
		\item Weiterführende Forschung
	\end{itemize}


\chapter{Literaturverzeichnis}
\printbibliography[heading=none]

\appendix

\chapter{Ehrlichkeitserklärung}

\chapter{Testprotokolle, weitere Spielkonzepte/Ideenfindung... }


\end{document}