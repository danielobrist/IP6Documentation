\documentclass[12pt,a4paper]{report}

% -- Imports biblatex and defines bib file --
\usepackage[backend=bibtex,style=numeric,language=german,sorting=none]{biblatex}
\addbibresource{references.bib}
% http://tug.ctan.org/info/biblatex-cheatsheet/biblatex-cheatsheet.pdf

% -- Language --
\usepackage[utf8]{inputenc}
\usepackage[ngerman]{babel}

% -- Images --
\usepackage{graphicx}
\graphicspath{ {./images/} }

% -- Continuous figure numbering --
\usepackage{chngcntr}
\counterwithout{figure}{chapter}

% -- Code blocks --
\usepackage{listings}
\usepackage{color}
\definecolor{lightgray}{rgb}{.9,.9,.9}	
\definecolor{darkgray}{rgb}{.4,.4,.4}
\definecolor{purple}{rgb}{0.65, 0.12, 0.82}

% -- Code blocks Stlye --
\renewcommand\lstlistingname{Codefragment}
\renewcommand\lstlistlistingname{Codefragmente}

\lstdefinelanguage{JavaScript}{
  keywords={typeof, new, true, false, catch, function, return, null, catch, switch, var, if, in, while, do, else, case, break},
  keywordstyle=\color{blue}\bfseries,
  ndkeywords={class, export, boolean, throw, implements, import, this},
  ndkeywordstyle=\color{darkgray}\bfseries,
  identifierstyle=\color{black},
  sensitive=false,
  comment=[l]{//},
  morecomment=[s]{/*}{*/},
  commentstyle=\color{purple}\ttfamily,
  stringstyle=\color{red}\ttfamily,
  morestring=[b]',
  morestring=[b]"
}
\lstset{
   language=JavaScript,
   backgroundcolor=\color{lightgray},
   extendedchars=true,
   basicstyle=\footnotesize\ttfamily,
   showstringspaces=false,
   showspaces=false,
   numbers=left,
   numberstyle=\footnotesize,
   numbersep=9pt,
   tabsize=2,
   breaklines=true,
   showtabs=false,
   captionpos=b
}

\newcommand{\paragraphwithnewline}[1]{\paragraph{#1}\mbox{}\\}

% -- comments/todos --
\usepackage{xcolor}
\newcommand\todo[1]{\textcolor{red}{#1}}



\begin{document}

\begin{titlepage}
    \begin{center}
        \vspace*{1cm}
            
        \Huge
        \textbf{Palim-Palim}
            
        \vspace{0.5cm}
        \Large
        Ein Videospiel mit integriertem Videochat für Kinder und betagte Menschen 
            
        \vspace{1.5cm}
            
        \textbf{Daniel Obrist}\\
        \textbf{Severin Peyer}

        \vfill
            
        Bachelorthesis
        \vspace{0.8cm}

        \includegraphics[width=0.65\textwidth]{fhnw_ht_10mm}

        \vspace{0.8cm}
            
            
        \Large
        Studiengang Informatik\\
        Profilierung iCompetence\\
        
        \vspace{0.8cm}

		Betreuende: Marco Soldati, Tabea Iseli\\
		Im Auftrag der Fachhochschule Nordwestschweiz
		
		\vspace{0.8cm}        
        
        Brugg, 20.08.2021\\
        
        \vspace{0.8cm}

    \end{center}
\end{titlepage}

\chapter*{Projektinformationen}
Titel: Palim-Palim\\
Projektnummer: 21FS\_I4DS08

\paragraphwithnewline{Projekt-Team}
Daniel Obrist, 8iCbb\\
daniel.obrist@students.fhnw.ch\\\\
Severin Peyer, 8iCbb\\
severin.peyer@students.fhnw.ch

\paragraphwithnewline{Auftraggeber und Betreuung FHNW}
Marco Soldati\\
Fachhochschule Nordwestschweiz FHNW\\
Hochschule für Technik\\
Bahnhofstrasse 6\\
CH-5210 Windisch\\
+41 56 202 77 31\\
marco.soldati@fhnw.ch\\\\
Tabea Iseli\\
Fachhochschule Nordwestschweiz FHNW\\
Hochschule für Technik\\
Bahnhofstrasse 6\\
CH-5210 Windisch\\
+41 56 202 86 53\\
tabea.iseli@fhnw.ch\\

\paragraphwithnewline{Experte}
Jonas Weibel\\
joweibel@microsoft.com

\paragraphwithnewline{Zeitbudget}
Das Projekt wird im Rahmen des Frühlingssemesters 2021 durchgeführt. Nominell sind für das Projekt 360~Arbeitsstunden pro Teammitglied veranschlagt. 

\paragraphwithnewline{Wichtigste Daten}
Beginn:	22. Februar 2021\\ 
Ende:	20. August 2021 \\

\begin{abstract}	
Palim-Palim ist ein Videochat-Spiel für Kinder und betagte Personen. Es kombiniert einen Videochat mit einem Kaufladen-Spiel. Mit dem Spiel wurde untersucht, wie Kinder und betagte Personen auch über grosse Distanzen miteinander spielen können. Erste Spieletests und ergänzende Literaturrecherchen zeigen, dass Videochat-Spiele wie Palim-Palim ein grosses Potenzial haben, Beziehungen zwischen Generationen auch über weite Distanzen zu fördern. Ebenfalls befasst sich Palim-Palim mit den Eigenheiten der Integration eines Videochats in ein Videospiel. Das Projekt dient somit auch als eine Wissensbasis für die Entwicklung weiterer Videochat-Spiele.
\end{abstract}


\tableofcontents

\break

\chapter{Einleitung}
Palim-Palim ist ein Videospiel mit integriertem Videochat für Kinder und betagte Personen. Zwei Spielende können damit in einer virtuellen Umgebung Kaufladen spielen. Umgesetzt als Tablet-Webanwendung, ermöglicht das Spiel Grosseltern sich mit ihren Enkelkindern über weite Distanzen hinweg auszutauschen und gleichzeitig etwas zu spielen. Als Proof of Concept setzt Palim-Palim den Grundstein zur Entwicklung eines entsprechenden Videochat-Spiele-Frameworks. In Abbildung \ref{PalimPalimScreens} ist Palim-Palim aus den Perspektiven beider Spielenden zu sehen.
\begin{figure}[!htb]
\includegraphics[width=\textwidth]{PalimPalim_screens_ipad}\label{PalimPalimScreens}
\caption[Caption for LOF]{Das Videospiel Palim-Palim. Perspektive der betagten Person (links) und des Kindes (rechts).}
\end{figure}\\\\
Mit dem Spiel wurde untersucht, welchen Einfluss der Videochat in Palim-Palim auf die User Experience und die Kommunikation zwischen den Spielenden hat. Resultate aus den durchgeführten Spieletests deuten darauf hin, dass der Videochat vor allem für die Kinder auch Einsatzmöglichkeiten als Gameplay-Element hat. Ein Einfluss auf die Kommunikation zwischen den Spielenden konnte nicht nachgewiesen werden.\\\\
Ergänzende Literaturrecherchen haben zudem gezeigt, dass Spiele mit Videochats ein grosses Potenzial haben, Beziehungen zwischen Generationen auch über weite Distanzen zu fördern. Durch die spielerische Komponente verlieren Kinder weniger schnell das Interesse an der Unterhaltung. Ältere Leute schätzen dabei generell die sozialen Aspekte des Spielens.\\\\
Videospiele für betagte Menschen und ihre Angehörigen werden an der Fachhochschule Nordwestschweiz im Rahmen des Projekts \flqq Myosotis\frqq{} schon seit einigen Jahren untersucht \cite{soldati_fhnw_2015}. Dabei wurden bereits einige Spiele umgesetzt und mit betagten Menschen getestet \cite{soldati_create_2020}. Jedoch hat sich bisher noch kein Projekt explizit mit der Integration eines Videochats in ein Spiel auseinandergesetzt. Palim-Palim schliesst diese Lücke und liefert wertvolle erste Erkenntnisse über die Kombination von Videochats und Videosspielen mit Kindern und betagten Personen.\\\\
Der Forschungsaspekt von Palim-Palim gewinnt dabei zusätzlich durch den Hintergrund der COVID-19-Pandemie an Relevanz. Eine amerikanische Studie zu dem Thema zeigt, dass 71 Prozent der Grosseltern von Kindern im Alter bis zu fünf Jahren aufgrund von COVID-19-Massnahmen verstärkt auf Videochats zurückgreifen \cite{kakulla_video_nodate}. Allerdings hat Videotelefonie die Eigenheit, dass Interaktionen unnatürlich und teilweise entfremdend wirken. Vor allem für Kinder ist es schwierig, Gesprächsthemen und Kommunikationswege zu finden, die sich so lustig und verbindend anfühlen, wie die Zeit mit der Grossmutter oder dem Grossvater im echten Leben. Oft sind Kinder vom Gespräch schnell gelangweilt – sie würden lieber etwas spielen \cite{tulloch_7_2020}. Palim-Palim als Videospiel bietet hierzu ein Mittel, um Kinder besser in das Gespräch mit ihrem Gegenüber einzubeziehen.\\\\
Um die Einflüsse eines Videochats in einem intergenerationellen Spiel zu untersuchen, wurden mehrere Varianten von Palim-Palim implementiert. Diese unterscheiden sich in der Art und Weise der Integration des Videochats. Mittels Spieletests dieser Varianten wurden deren Einflüsse auf das Spielerlebnis sowie die Kommunikation der Spielenden evaluiert und verglichen.\\\\
Die Implementierung von Palim-Palim als Webapplikation erlaubte zusätzlich die Auseinandersetzung mit den entsprechenden Videochat-Technologien. Dadurch wurde eine Wissensbasis für die Entwicklung weiterer Videochat-Spiele oder eines betreffenden Frameworks geschaffen.\\\\
Anschliessend an diese einleitenden Worte wird in einigen theoretischen Kapiteln die Ausgangslage und der bisherige Forschungsstand der Thematik dargelegt. Diese theoretischen Teile bilden die Grundlage für die Aufstellung der Forschungsfragen in Kapitel \ref{Forschungsfragen}. Anschliessend werden die Fragen anhand der durchgeführten Tests beantwortet. Danach wird die ganze Umsetzung vom Game Design bis hin zur Implementierung der Spiellogik detailliert erläutert. In dem Fazit sind dann nochmals alle zentralen Erkenntnisse sowie mögliche Weiterentwicklungen beschrieben.

% -- Theoretischer Teil --
\chapter{Umfeldanalyse}
Zeit zusammen zu verbringen ist mit Einflüssen wie der COVID-19-Pandemie oder auch für weit entfernt lebende Verwandte gar nicht so leicht. Immer mehr Grosseltern greifen deswegen in der Kommunikation mit ihren Enkelkindern auf Videotelefonie zurück. Dies haben Kakulla et al. in einer empirischen Studie belegt \cite{kakulla_video_nodate}. Ein steigendes Angebot sowie technische Neuerungen wie Smartphones und Tablets erlauben den betagten Personen einen deutlich leichteren Zugang zur Videotelefonie als früher \cite{glaab_silver_2015}. In einem Seniorenzentrum in Bielefeld-Senne beispielsweise wird schon seit Juli 2013 auf innovative Technologien gesetzt. Die Videotelefonie wurde dabei von den Bewohnenden sehr schnell rege genutzt, um ihre sozialen Kontakte aufrecht zu erhalten \cite{michels-ries_altenhilfe_2017}.\\\\
Wie schon in der Einleitung angesprochen ist Palim-Palim Teil des Forschungsprojekts Myosotis der Fachhochschule Nordwestschweiz. Die Videospiele von Myosotis sollen Familienmitgliedern, insbesondere auch Kindern, dabei helfen, mit ihren betagten Angehörigen Zeit zu verbringen \cite{soldati_fhnw_2015}. Die Spiele sollen auch dazu beitragen, durch positive soziale Interaktionen das Wohlbefinden betagter Menschen zu verbessern \cite{soldati_create_2020}. Bezüglich kameragesteuerten Spielen berichten Soldati et al. \cite{soldati_create_2020}, dass mit diesen einige der unterhaltsamsten Spieletests erlebt wurden. Allerdings wurde im Rahmen von Myosotis noch kein Spiel mit integriertem Videochat untersucht.\\\\



\chapter{Intergenerationelles Spielen}\label{IntergenerationellesSpielen}
Um zwischen zwei Generationen eine Brücke zu schlagen, bieten sich Videospiele als eine vielversprechende Möglichkeit an. Denn besonders im familiären Umfeld fördert gemeinsames Spielen die Beziehungen untereinander und ruft positive Emotionen sowohl bei älteren als auch jüngeren Generationen hervor \cite{osmanovic_beyond_2016}. Es hat sich gezeigt, dass Personen eine grössere Zuneigung zu Spielpartner:innen aus einer anderen Altersgruppe entwickeln und sich im Umgang mit der anderen Altersgruppe wohler fühlen, wenn sie regelmässig zusammen spielen \cite{chua_lets_2013}. Eine Literaturstudie von de la Hera et al. \cite{de_la_hera_benefits_2017} fasst diverse Erkenntnisse zu dem Thema sehr gut zusammen. Sie identifiziert folgende vier Vorzüge des intergenerationellen Spielens: eine Stärkung der familiären Bindung, eine Förderung des wechselseitigen Lernens, ein grösseres Verständnis für die andere Generation und eine Verringerung sozialer Ängste. Unter Berücksichtigung dieser Faktoren lässt sich argumentieren, dass Spielen eine wirkungsvolle Methode ist, um jüngere und ältere Menschen auf einem emotionalen, aber auch sozialen Level zu verbinden.\\\\
Durch den gegebenen Altersunterschied zwischen jungen und älteren Spielenden ergibt sich eine besondere Herausforderung. In der Gestaltung von intergenerationellen Spielen müssen stets zwei sehr unterschiedliche Zielgruppen berücksichtigt werden. Denn nicht nur die Werte und Normen sind teilweise zwischen Generationen grundlegend anders. Auch biologische Einflüsse des Älterwerdens sind bei jüngeren Spielenden noch weniger ausgeprägt. Dazu kommen ein unterschiedliches technisches Verständnis sowie divergierende Vorkenntnisse. Personen ab 50 Jahren bevorzugen zum Beispiel Gelegenheitsspiele gegenüber komplexeren und ausdauernden Spielen \cite{schultheiss_entertainment_2012}. Auch Gajadhar et al. \cite{gajadhar_out_2010} kamen zum Schluss, dass Spiele für Senior:innen eher wenig Fokus auf den sozialen Wettbewerb und einen besonderen Fokus auf das kooperative Spiel legen sollten. Diese Ergebnisse wurden von De Schutter \cite{de_schutter_meaningful_2008} bestätigt, der feststellte, dass soziale Interaktion der wichtigste Prädiktor für die Spieldauer bei älteren Erwachsenen ist. Betagte Menschen sind also besonders an Spielerfahrungen interessiert, die eine starke soziale Komponente haben. Dies deckt sich auch mit Empfehlungen aktueller Literatur zur Spieleentwicklung \cite{schell_art_2008}.\\\\
Kinder hingegen haben differenziertere Anforderungen an Videospiele. Es gibt in der Videospielindustrie grundlegende Empfehlungen für gewisse Altersgruppen, welche stark auf der mentalen Entwicklung beruhen. Ab dem vierten Lebensjahr beginnen Kinder ihr erstes Interesse an Spielen zu zeigen. Spiele für 4- bis 6-Jährige sind oft sehr einfach und werden häufiger mit den Eltern gespielt als mit Gleichaltrigen. Denn die Eltern wissen, wie sie die Regeln so gestalten können, dass die Spiele Spass machen und interessant sind \cite{schell_art_2008}. Ab 7 Jahren können Kinder dann in der Regel lesen, Dinge selbst herausfinden und auch schwierigere Probleme lösen. Sie beginnen selbst zu entscheiden, welche Art von Spielzeug und Spielen sie mögen und welche nicht \cite{schell_art_2008}. Im späteren Verlauf des Lebens kristallisieren sich dann weitere Vorlieben heraus, welche für die Zielgruppen von Palim-Palim jedoch nicht von grosser Bedeutung sind. Es lässt sich somit ein klarer Unterschied zu den Vorlieben und Bedürfnissen älterer Leute feststellen.\\\\
Um auf die abweichenden Anforderungen von betagten Personen und Kindern einzugehen, implementiert Palim-Palim eine asymmetrische Gestaltung des Gameplays. Das heisst, beide Spielenden erhalten unterschiedliche Rollen und Aufgaben im Spiel. Die konkrete Gestaltung dieses asymmetrischen Gameplays ist im Kapitel \ref{DesignUndUsability} ausformuliert.

\chapter{Videochats in Videospielen}
Videochats sind ein wichtiges Kommunikationsmittel um zwei oder mehr Menschen über weite Distanzen miteinander zu verbinden. Auch betagte Menschen schätzen und nutzen diese moderne Art der Kommunikation mit der Familie immer mehr – besonders weil es öfters auf ältere Gruppen von Benutzenden angepasste Angebote gibt \cite{glaab_silver_2015}.\\\\
Während Videochats im geschäftlichen Umfeld bereits gut erforscht sind, gibt es im privaten Bereich noch Lücken, wie auch Batcheller et al. \cite{batcheller_testing_2007} in ihrer Arbeit zum Thema «Testing the technology: Playing Games with Video Conferencing» erwähnen. Sie haben in ihrer Studie ebenfalls aufgezeigt, dass Personen, welche das Spiel «Mafia» über eine Videokonferenz spielen, ein ähnliches Mass an Zufriedenheit, Spass und Frustration empfinden wie Teilnehmende, welche in einer echten Umgebung spielen. Dies deutet auf ein grosses Potenzial hin, um ein Multiplayer-Spiel für die Spielenden mit Hilfe eines Videochats reichhaltiger und ansprechender zu gestalten.\\\\
Eine Studie von Veinott et al. \cite{veinott_video_1999} hat zudem aufgezeigt, dass Videogespräche im Vergleich mit Telefonaten eine bessere Basis für Verhandlungen zwischen zwei Personen im Geschäftsumfeld schaffen. In ähnlicher Weise erfordern Multiplayer-Spiele mit aktiver Kommunikation zwischen den Spielenden oft subtile Hinweise, um Spielentscheidungen zu treffen. Dies deutet darauf hin, dass Videochats auch die gemeinsame Entscheidungsfindung in Spielen erleichtern können.\\\\
Ausserdem hat eine Studie von Bos et al. \cite{bos_short_2001} untersucht, wie sich das Vertrauen zwischen Spielenden in einem Social Dilemma-Spiel \cite{harrod_social_1983} via Videochat entwickelt. Dabei hat sich gezeigt, dass zwischen Teilnehmenden des Videochats ein vergleichbares Vertrauen entsteht, wie in Durchgängen, welche Face-To-Face durchgeführt werden. Die Studie notiert allerdings auch, dass es verglichen mit den vor Ort durchgeführten Spielen länger dauert, bis dasselbe Niveau an Vertrauen aufgebaut wird.\\\\
Ergänzend dazu deuten Erkenntnisse aus einer Studie von Derboven et al. \cite{derboven_designing_2012} darauf hin, dass in einem Multiplayer-Videospiel die zusätzliche Kommunikationsfunktionalität durch einen Videochat oft sowohl von älteren als auch von den jüngeren Personen begrüsst wird. Die Videochat-Funktionalität hatte dem intergenerationellen Spiel eine zusätzliche soziale Dimension verliehen. Derboven et al. empfehlen daher in ihren abschliessenden Worten, Videochats in intergenerationellen Spielen vermehrt einzusetzen. Allerdings warnen sie auch davor, dass ein Videochat in gewissen Fällen zu unangenehmen Situationen führen kann – besonders wenn die beiden Spielenden nicht das Bedürfnis haben, miteinander zu sprechen.\\\\
Es lässt sich also zusammenfassen, dass Videochats in diversen Bereichen einen wesentlichen Einfluss auf die Kommunikation zwischen zwei Parteien haben. Besonders in Videospielen, in welchen der aktive Austausch zwischen den Spielenden Teil des Spielgeschehens ist, sollte dieser Einfluss im Game Design berücksichtigt werden.\\\\

\chapter{Aufstellung der Forschungsfragen}\label{Forschungsfragen}
Wie in den vorhergehenden Kapiteln theoretisch dargelegt wird, bietet die Kombination von Videospielen mit Videotelefonie eine bedeutende Möglichkeit für intergenerationelle Spiele. Die zentralen Forschungsfragen von Palim-Palim fokussieren deshalb speziell auf den Aspekt der Videotelefonie in Videospielen. Dieses Kapitel fasst hierzu alle Fragestellungen zusammen, welche Palim-Palim untersucht.\\

\section{Hauptfragestellung}\label{hauptfragestellung}
Palim-Palim untersucht, wie Videospiele und Videotelefonie kombiniert werden können. Zusätzlich werden die Auswirkungen auf die Interaktionen zwischen betagten Menschen und Kindern erforscht. Daraus ergibt sich folgende übergeordnete Fragestellung:
\begin{quote}
\textbf{Wie lässt sich Videotelefonie mit Videospielen kombinieren, damit Kinder und betagte Personen übers Internet miteinander spielen und sich gleichzeitig unterhalten können?}
\end{quote}
			

\section{Einzelfragestellungen im thematischen Zusammenhang}
Um die übergeordnete Aufgabenstellung messbar zu machen, setzt sich Palim-Palim mit spezifischen Fragestellungen zu den Themen User Experience (siehe \ref{fragestellungenUserExperience}) und Kommunikation (siehe \ref{fragestellungenKommunikation}) auseinander. Die dazu formulierten Hypothesen werden durch die Resultate aus den Spieltests in Kapitel \ref{spieletestsUndResultate} verifiziert oder widerlegt.

\subsection{Fragestellungen zur User Experience}\label{fragestellungenUserExperience}
Folgende Einezlfragestellungen befassen sich mit den Auswirkungen des Videochats auf das Spielerlebnis.

\paragraph{1. Wie beeinflusst die Einbindung von Videotelefonie die User Experience in Videospielen?}
	\subparagraph{Fragestellung 1a:} Welche Wirkung hat ein integrierter Videochat auf das Spielerlebnis von betagten Menschen?
	\subparagraph{Hypothese 1a:} Ein integrierter Videochat hat eine positive Wirkung auf das Spielerlebnis von betagten Menschen.

	\subparagraph{Fragestellung 1b:} Welche Wirkung hat ein integrierter Videochat auf das Spielerlebnis von Kindern?
	\subparagraph{Hypothese 1b:} Ein integrierter Videochat hat eine positive Wirkung auf das Spielerlebnis von Kindern.

	\subparagraph{Fragestellung 1c:} Welche Auswirkungen hat die Art und Weise, wie der Videochat in das Spiel integriert ist, auf das Spielerlebnis?
	\subparagraph{Hypothese 1c:} Je stärker der Videochat ins Gameplay integriert ist, desto positiver ist das Spielerlebnis.


\subsection{Fragestellungen zur Kommunikation}\label{fragestellungenKommunikation}
Folgende Einezlfragestellungen befassen sich mit den Auswirkungen des Videochats auf das Kommunikationsverhalten der Spielenden.

\paragraph{2. Wie beeinflusst die Einbindung von Videotelefonie die Kommunikation in Videospielen?}
\subparagraph{Fragestellung 2a:} Welche Wirkung hat ein integrierter Videochat auf die Kommunikation zwischen den Spielenden? 
\subparagraph{Hypothese 2a:} Ein im Spiel integrierter Videochat fördert die Kommunikation zwischen den Spielenden. 

	\subparagraph{Fragestellung 2b:} Wird ein integrierter Videochat aktiv als Kommunikationsmittel zur Bewältigung von Spielaufgaben genutzt? 
	\subparagraph{Hypothese 2b:} Der Videochat wird aktiv als Kommunikationsmittel zur Bewältigung der Spielaufgabe verwendet. 

	\subparagraph{Fragestellung 2c:} Welche Auswirkungen hat die Art und Weise, wie der Videochat in das Spiel integriert ist, auf die Förderung der Kommunikation? 
	\subparagraph{Hypothese 2c:} Je stärker der Videochat ins Spiel integriert ist, desto angeregter ist der Austausch zwischen den Spielenden. 
	
\chapter{Methoden}
Wie Jesse Schell in «The Art of Game Design» beschreibt, ist die Selbstbeobachtung eine wichtige Methode, um sich schnell einen Überblick zu verschaffen, was im entwickelten Spiel funktioniert und was nicht. Dabei sollte jedoch die Gefahr der Subjektivität nicht unterschätzt werden, denn «was nach eigener Erfahrung wahr ist, mag für andere nicht wahr sein» \cite[56]{schell_art_2008}. Aus diesem Grund ist es nützlich und notwendig der Zielgruppe des Spiels sehr gut zuzuhören. Dafür eignen sich Spieletests. Dabei werden Angehörige der Zielgruppe eingeladen, das Spiel zu spielen, während sie beobachtet werden. So kann festgestellt werden, ob das beabsichtigte Spielerlebnis vermittelt wird \cite{schell_art_2008}.\\\\
Palim-Palim setzt auf einen qualitativen statt auf einen quantitativen Testansatz. Einerseits weil die Projektzeit kein vergleichendes Testing mit mindestens 10 Personen \cite{noauthor_how_2009} zuliess, andererseits weil Palim-Palim als ein erster funktionsfähiger Prototyp und nicht als marktreifes Videospiel verstanden wird. Um Palim-Palim zu testen und die in Kapitel \ref{hauptfragestellung} aufgestellten Hypothesen zu überprüfen wurde ein kontrollierter Feldversuch mit zwei Testpaaren durchgeführt (betagte Menschen: 1 Frau, 1 Mann, Alter: 80+, Kinder: 2 Jungen, 9 und 10 Jahre alt). Die teilnehmenden betagten Menschen wiesen keine grösseren gesundheitlichen Probleme auf.\\\\
Die Spieletests wurden in drei Phasen gegliedert, in welchen unterschiedliche Videomodi in einer zufälligen Reihenfolge gespielt wurden. Der Modus 1 beinhaltet kein Video, sondern nur Audio. Im Modus 2 ist der Videochat getrennt von der Spielszene eingebunden und in Modus 3 ist dieser in die Spielumgebung integriert. Eine genauere Erklärung der Videomodi befindet sich in Kapitel \ref{spielvarianten}.\\\\
Die Tests wurden an einem für die Teilnehmenden vertrauten Ort durchgeführt. Dadurch kann das Spiel in einer ihm zugedachten natürlichen Umgebung stattfinden, was gemäss Jesse Schell ein grosser Vorteil sein kann \cite{schell_art_2008}. Die eine Testsequenz dauerte rund 19 Minuten, die andere rund 28 Minuten. Die Teilnehmenden erhielten vor den Spieletests eine kurze Einführung, was Palim-Palim ist und eine Erklärung, wie die Spieletests ablaufen werden \todo{[Verweis auf «Information zum Spieletest von Palim-Palim» im Anhang]}. Ebenfalls wurden diese über die Freiwilligkeit der Tests und die Möglichkeit, die Tests jederzeit und ohne Konsequenzen abzubrechen, informiert. Alle Teilnehmenden, respektive ihre Erziehungsberechtigten, unterzeichneten auch eine Einverständniserklärung für die Verwendung ihrer Daten im Rahmen von Palim-Palim \todo{[Verweis auf die Einverständniserklärungen im Anhang]}.\\\\
Die Spieletests wurden auf Video aufgezeichnet, um die Aussagen und Handlungen der Testpersonen möglichst vollständig zu erfassen und dokumentieren. Die Teilnehmenden wurden nach jedem Videomodus gebeten, ihr Spielerlebnis zu beurteilen. Zusätzlich wurden den Testpersonen qualitative Fragen gestellt. Dabei ging es darum Erkenntnisse zu positiven sowie negativen Gameplay-Elementen zu gewinnen und herauszufinden, ob Palim-Palim den Testpersonen zusagt oder nicht. Nach den drei gespielten Runden wurden die Proband:innen zusätzlich nach möglichen Ideen und Wünschen gefragt. Die Fragen wurden sehr offen gehalten, um diese auf die Teilnehmenden anpassen zu können. Die Fragebögen und die Transkription der Spieletests sind im Anhang zu finden. \todo{Verlinkung in den Anhang}\\\\
Für die Auswertung wurden die Videoaufnahmen transkribiert und analysiert. Zentrale Beobachtungen sowie wichtige Aussagen, Ideen und Verbesserungswünsche der Testpersonen sind im nachfolgenden Kapitel \ref{spieletestsUndResultate} dokumentiert.

% -- Praktischer Teil --
\chapter{Spieletests und Resultate}\label{spieletestsUndResultate}

In diesem Kapitel werden die Ergebnisse der Spieletests von Palim-Palim präsentiert. Dabei ist sehr wichtig, diese Resultate mit besonderer Vorsicht zu geniessen. Die sehr kleine Anzahl Testpersonen lässt es nicht zu, aussagekräftige Schlüsse aus den Resultaten zu ziehen. Es wurden daher die zentralsten Aussagen und Beobachtungen zusammengefasst und mit Erkenntnisse aus der Literatur ergänzt.\\\\
Als erstes werden die Resultate aus den Spieletests mit den betagen Menschen beleuchtet. Anschliessend wird die Seite der Kinder betrachtet. Abschliessend werden dann die gestellten Forschungsfragen beantwortet.

\section{Resultate der Spieletests mit betagten Menschen}
\label{resultateBetagteMenschen}
Bereits in der gemeinsamen Gamelobby sowie auf dem Erklärungsscreen ist aufgefallen, dass die betagten Menschen eher viel Zeit für das Lesen der Anweisungen benötigen. Liu et al. bestätigen hierzu, dass ältere Personen einiges langsamer lesen als jüngere Menschen \cite{liu_age-related_2017}.\\\\
Die Verbindung zwischen den zwei Spielenden findet über eine gemeinsame Raumnummer statt. Die Formulierung «Raum» ist für die Testpersonen undeutlich und wurde als unschön sowie kalt bezeichnet. Bei der Auswahl des Spielmodus verstand die eine betagte Person nicht, dass diese Entscheidung zusammen mit dem Kind getroffen werden kann.\\\\
Für beide Testpersonen ist unklar, dass die Abbildung mit den Erklärungen noch nicht zum Spiel gehört. Sie versuchten bereits das Einkaufsobjekt zu verschieben und die Einkaufliste mit tippen zu öffnen. Die Anweisung «Frag dein Enkelkind nach Objekten» war für eine Person unklar, denn sie fragte das Kind, was dieses gerne hätte, obwohl sie nach einem Gegenstand ihrer Einkaufsliste fragen müsste. Die Senior:innen waren sich uneinig, ob es sinnvoll ist, die Anleitung vor jedem Spielstart anzuzeigen. Eine Testperson findet dies wichtig, damit klar ist, was gemacht werden soll, die andere Person ist der Meinung, dass es mühsam sei und sie in dieser kurzen Zeit die simplen Funktionalitäten nicht vergessen habe.\\\\
Zu Beginn des Gameloops ist es für die Testpersonen trotz des vorab angezeigten Erklärungsscreens unklar, dass sie nach einem Gegenstand fragen sollen, welcher auf der Einkaufsliste zu finden ist. Stattdessen fragten sie nach einem beliebigen Produkt.\\\\
Der illustrierte und zu 70 Prozent transparente Hintergrund sorgte für Unklarheiten. Einerseits wurde von den Senior:innen bemängelt, dass dieser thematisch zu wenig auf den Spielmodus abgestimmt ist. Andererseits sei es schwierig diesen als Hintergrund zu erkennen. Der niedrige Kontrast, welcher durch die hohe Transparenz entsteht, führt bei den betagten Personen zu Schwierigkeiten, die einzelnen Hintergrundobjekte auseinanderzuhalten. Dass ältere Personen Mühe mit dem Kontrastsehen haben, bestätigen auch Ijsselsteijn et al. \cite{ijsselsteijn_digital_2007}.\\\\
Zu den Interaktionen konnten die drei folgenden Beobachtungen gemacht werden. Einmal war es unklar, dass ein Objekt mit Ziehen verschoben werden kann. Stattdessen wurde zuerst auf das Objekt und danach an den gewünschten Zielort getippt. Als zwei gleiche Objekte sehr nah nebeneinander platziert waren, wurden diese als Gruppe wahrgenommen und versucht zusammen an den gewünschten Zielort zu ziehen — was jedoch nicht funktionierte. Des Weiteren hat eine Testperson die Einkaufsliste zuerst mit Ziehen statt mit Tippen versucht zu öffnen.\\\\
Das mitspielende Kind zu sehen, ist laut den Aussagen der beiden Senior:innen nicht zwingend notwendig und führt daher auch zu keiner veränderten Wirkung des Spielerlebnisses. Die eine Testperson spielte zuerst die Variante ohne Video. Dies erwies sich jedoch als schwierig, da unklar war, dass trotz fehelndem Video eine Audioübertragung vorhanden ist. Dies lässt darauf schliessen, dass ein Videochat-Element auch als Indikator für die aufgebaute Verbindung dient.\\\\
Nach dem Konfettiregen, welcher das Ende des Spiels anzeigt, war für eine betagte Testperson unklar, dass das Spiel nun zu Ende ist, respektive wie es nun weitergeht.\\\\
Positive Rückemeldungen erhielt Palim-Palim bezüglich dem Abstreichen des gekauften Gegenstands auf der Liste. Ebenfalls wurde das Verschieben der Objekte als sehr flüssig wahrgenommen. \\\\
Vebesserungsvorschläge der Testpersonen bezogen sich in erster Linie auf mehr Inhalt. Dies zeigt, dass Palim-Palim mehr Verkaufsartikel, verschiedene Aufgaben je nach Spielfortschritt und vielleicht sogar andere Szenarien als nur einen Einkaufsladen anbieten sollte. Die Senior:innen wünschten sich ausserdem, selbst entscheiden zu können, was sie einkaufen möchten.\\\\
Es lässt sich zusammenfassen, dass Palim-Palim aus Sicht der betagten Testpersonen noch sehr viel Verbesserungspotenzial hat. Allerdings wurden auch die Möglichkeiten eines solchen Spieles erkannt. Um eine Testperson zu zitieren: 
\begin{quote}
\textit{«Also gerade jetzt in der Covid-Zeit, in der man sich nicht gesehen hat, dann wäre das jetzt noch ziemlich gut gewesen, dann hätte man die Oma gesehen oder die Oma hätte den Enkel gesehen und dann hätte man noch ein Spiel machen können. Da würde ich sagen, ja… es hat Potenzial.»}
\end{quote}
	
\section{Resultate der Spieletests mit Kinder}
\label{resultateKinder}
Beim Beobachten der Kinder ist aufgefallen, dass ihr Spielerlebnis schon zu Beginn durch die einseitige Bedienung der Gamelobby beeinträchtigt wird. Das Kind kann in Palim-Palim nicht mit der Gamelobby interagieren und sieht ausser dem Video auch keine visuellen Indikatoren darüber, was das Gegenüber gerade auswählt. Kombiniert mit dem Fakt, dass sein betagtes Gegenüber teilweise ein bisschen länger braucht, um sich zurechtzufinden, führt dies schon vor dem Spielbeginn zu Interessensverlust.\\\\
Nach dem Start des Spiels wird in Palim-Palim zuerst ein Screenshot des Spiels mit grafischen Elementen als Erklärung der Spielmechanik gezeigt. Die Kinder versuchten, wie die betagten Personen, auf dieser Abbildung bereits die Objekte zu verschieben. Sie waren dementsprechend verwirrt als nichts passierte. Hieraus lässt sich folgern, dass aus bisherigen Erfahrungen mit Videospielen ein interaktives Tutorial erwartet wird. Ebenfalls kommt hier wieder die einseitige Bedienung ins Spiel: die Kinder wussten nicht, dass nur die betagte Person am anderen Ende die Spielanleitung schliessen kann. Dadurch entstand der Eindruck, dass das Spiel nicht responsiv sei.\\\\
Sobald sich die Kinder in der 3D-Szene befanden, haben sie die Spielmechanik von Palim-Palim sehr schnell begriffen. Sie hatten keinerlei Probleme die Verkaufsgegenstände dem Gegenüber anzubieten. Das Ziehen der Objekte funktionierte einwandfrei, jedoch fehlten Reaktionen des Spieles. Es wurde bemängelt, dass nichts passiert.\\\\
Das fehlende Video in den Modi 1 und 2 wurde von den testenden Kindern bemerkt. Im Modus 3 wurde sogar mehrmals versucht, mit dem Video der betagten Person zu interagieren. Zum Beispiel wurde probiert, die Gegenstände dem Gegenüber zu füttern oder damit das Gesicht lustig zu gestalten. Dies zeigt, dass ein Videochat als Teil der Gameplay-Szene für die Kinder kein fremdes Element ist. Vielmehr bietet sich hier eine grosse Möglichkeit, noch mehr Interaktionen einzubauen.\\\\
Aus diesen Beobachtungen sowie dem qualitativen Feedback der Kinder lässt sich folgern, dass ein Videochat durchaus positive Wirkungen auf das Spielerlebnis haben kann. Jedoch muss darauf geachtet werden, dass Kinder hohe Erwartungen an ein Spiel haben. Die beiden Kinder forderten mehr Inhalte und Interaktionsmöglichkeiten. Es sollten also bewusst solche geschaffen werden. Insbesondere Interaktionen mit dem Video fühlen sich für sie natürlich an und sollten unbedingt in das Gameplay integriert werden.\\\\

	
\section{Beantwortung der Forschungsfragen}
\subsubsection{Fragestellung 1a: Welche Wirkung hat ein integrierter Videochat auf das Spielerlebnis von betagten Menschen?}
\subsubsection{Hypothese 1a: Ein integrierter Videochat hat eine positive Wirkung auf das Spielerlebnis von betagten Menschen.}
Die zwei Spieletests konnten diese Hypothese nicht belegen. Aber auch zum Widerlegen der Hypothese fehlen entsprechende Daten. Für den einen Probanden braucht es das Video nicht unbedingt:
\begin{quote}
\textit{«Jetzt hast du den Thommy gar nicht gesehen, sein Bild nicht gesehen. Hast du das jetzt schlechter oder besser gefunden? Oder hat dich das gestört?» - «Nein, spielt eigentlich keine Rolle.» - «Also musst du ihn nicht unbedingt sehen?» - «Nicht unbedingt.»} (Name geändert)
\end{quote}
Gemäss der anderen Probandin hat der integrierte Videochat ebenfalls keine grosse Auswirkung auf ihr Spielerlebnis. Jedoch findet sie den dritten Modus niedlich, da man das Kind hinter dem Tresen sieht.\\\\
Um die Fragestellung 1a jedoch eindeutig beantworten zu können, müssten eine grössere Anzahl Spieletests durchgeführt werden. Ebenfalls sollten dabei unterschiedliche Spiele berücksichtigt werden, um den Einfluss des Spieles selbst auf das Erlebnis auszuschliessen.\\\\

\subsubsection{Fragestellung 1b: Welche Wirkung hat ein integrierter Videochat auf das Spielerlebnis von Kindern?}
\subsubsection{Hypothese 1b: Ein integrierter Videochat hat eine positive Wirkung auf das Spielerlebnis von Kindern.}
Die Beobachtungen der Kinder deuten darauf hin, dass der Videochat sich positiv auf Ihr Spielerlebnis ausgewirkt hat. Besonders aufgefallen sind die Interaktionsversuche mit dem Video, beispielsweise als ein Kind die Banane auf dem Video als Nase der betagten Person platziert hat. Dies hat beim Kind zu einigen Lachern geführt.\\\\
Ebenfalls bewerteten beide Kinder das Spielerlebnis der Durchgänge mit Video höher, als jenes beim Modus ohne Video. Allerdings waren die Bewertungen der Kinder allgemein erstaunlich hoch. Es wird vermutet, dass sie sich nicht getrauten Kritik auszusprechen. Aufgrund der geringen Anzahl Tests kann auch hier die Hypothese nicht abschliessend bestätigt werden. Es zeichnen sich aber Tendenzen zu Erkenntnissen aus den Studien von Follmer et al. \cite{follmer_video_2010} ab, welche besagt, dass Spielen das Engagement der Kinder in einem Videochat erhöht. Hierzu müssten aber weitere Tests durchgeführt werden, welche den Fokus auf die Messung des Engagements legen.\\\\

\subsubsection{Fragestellung 1c: Welche Auswirkungen hat die Art und Weise, wie der Videochat in das Spiel integriert ist, auf das Spielerlebnis?}
\subsubsection{Hypothese 1c: Je stärker der Videochat ins Gameplay integriert ist, desto positiver ist das Spielerlebnis.}
Die Spieltests zeigten keine unterschiedliche Wahrnehmung der Spielerlebnisse, ob der Videostream in die Spielszene integriert oder unabhängig davon ist. Da Palim-Palim über keinen interaktiven Videochat verfügt unterschieden sich die gespielten Videomodi aber auch nur geringfügig. Von den Probanden wurde nicht gross bemerkt oder angesprochen, dass sich beim Videomodus 2 das Video als simples Overlay präsentiert und Videomodus 3 das Video auf eine Ebene in der 3D-Szene hinter dem virtuellen Tresen projiziert. Es wurde lediglich bemerkt, dass die Person hinter dem Tresen zu stehen scheint, was als «niedlich» bezeichnet wurde.\\\\
Dies lässt darauf schliessen, dass man das Video noch stärker als Teil der Gameplay-Szene gestalten solle, um eine solche Frage beantworten zu können. Ein mögliches Freistellen des Videos, oder sogar das Erweitern eines virtuellen Avatars mit einem Ausschnitt des Videos bieten sich als Möglichkeiten an. Da Palim-Palim diese Funktionalitäten noch nicht besitzt, kann die Fragestellung 1c also auch nicht abschliessend beantwortet werden.\\\\


\subsubsection{Fragestellung 2a: Welche Wirkung hat ein integrierter Videochat auf die Kommunikation zwischen den Spielenden?}
\subsubsection{Hypothese 2a: Ein im Spiel integrierter Videochat fördert die Kommunikation zwischen den Spielenden.}
Beobachtungen der Testpersonen haben gezeigt, dass beim Betreten des Videochats jedes Mal eine initiale Kommunikation in Form von «Hallo hörst du mich?» oder «Bist du bereit?» verwendet wurde. In den Testfällen mit Video haben sich die Probanden zudem gegenseitig zugewinkt. Diese Art der nonverbalen Begrüssung wurde auch schon von Derboven et al. \cite{derboven_designing_2012} in ihrer Studie festgestellt. Dieses Verhalten deutet darauf hin, dass ein Videochat bei der Kontaktaufnahme zum Gegenüber einen persönlicheren Eindruck vermitteln kann.\\\\
Im Spiel selbst konnte aber auch bei dieser Fragestellung kein signifikanter Unterschied zwischen den beiden Modi mit und ohne Video festgestellt werden. Der Videochat in Palim-Palim hat die Kommunikation zwischen den Spielenden also nicht gefördert. Es muss allerdings erwähnt werden, dass Palim-Palim sehr wenige Spielelemente besitzt, welche eine Kommunikation zwischen den Spielenden erfordert. Es wäre angebracht, mehr kommunikationsfördernde Teile in das Gameplay zu integrieren, bevor man hierzu eine repräsentative Aussage machen kann.\\\\

\subsubsection{Fragestellung 2b: Wird ein integrierter Videochat aktiv als Kommunikationsmittel zur Bewältigung von Spielaufgaben genutzt?}
\subsubsection{Hypothese 2b: Der Videochat wird aktiv als Kommunikationsmittel zur Bewältigung der Spielaufgabe verwendet.}
Im Hinblick auf die Spielaufgaben wurde der Videochat nicht aktiv verwendet. Allerdings lässt sich dies auch auf die mangelnde Vielfalt an Spielaufgaben zurückführen. Da Palim-Palim zum Testzeitpunkt im Hinblick auf das Gameplay nur das Übergeben der Einkaufsgegenstände implementierte, kann diese Frage nicht abschliessend beantwortet werden. Dafür müssten weitere Spielaufgaben, welche auf verschiedene Arten gelöst werden können, untersucht werden.\\\\

\subsubsection{Fragestellung 2c: Welche Auswirkungen hat die Art und Weise, wie der Videochat in das Spiel integriert ist, auf die Förderung der Kommunikation?}
\subsubsection{Hypothese 2c: Je stärker der Videochat ins Spiel integriert ist, desto angeregter ist der Austausch zwischen den Spielenden.}
Zwischen den Videomodi wurde kein Anstieg des Austausches unter den Spielenden festgestellt. Wie schon bei Forschungsfrage 1c beschrieben, lässt sich hierzu keine Aussage machen, ohne weitere Arten der Videointegration mit einer höheren Anzahl an Testpersonen zu testen.\\\\
			
	
\chapter{Game Design}
Zur Beantwortung der Forschungsfragen wurde das Videospiel Palim-Palim entwickelt. Der kreative Prozess, um ein möglichst dienliches Spielkonzept zu finden, wird im nächsten Kapitel \ref{ideenfindung} beschrieben. Das Konzept der ausgewählten Idee, Palim-Palim, wird anschliessend in Kapitel \ref{spielkonzept} genauer erläutert. Um die Fragestellungen im Zusammenhang mit der Integrationsart des Videochats zu beantworten, wurden unterschiedliche Videomodi umgesetzt. Die Erklärungen zu den einzelnen Modi befinden sich in Kapitel \ref{spielvarianten}. Im letzten Unterkapitel sind die Usability- und Designentscheidungen dokumentiert.

\section{Ideenfindung}\label{ideenfindung}
Parallel zur Entstehung erster Ideen für Palim-Palim wurden etwa 25 weitere Ansätze für intergenerationelle Spiele mit einem Videochat ausgearbeitet. Diese sind im Anhang \ref{WeitereIdeen} inklusive einer kurzen Beschreibung aufgelistet. Bei der Entwicklung dieser Spielkonzepte galt es dabei zwei wesentliche Faktoren zu berücksichtigen. Einerseits sollte der Videochat ein zentrales Element des Spiels sein. Die Spielenden sollen mit dem Videoelement interagieren können und es als Teil der Spielumgebung wahrnehmen. Andererseits musste berücksichtigt werden, dass es sich um zwei sehr unterschiedliche Zielgruppen handelt, welche zusammen etwas spielen.\\\\
Nebst dem umgesetzten Konzept «Kinder-Einkaufsladen», auf welchem Palim-Palim beruht, wurde beispielsweise die Idee für ein «virtuelles Bilderbuch» in Erwägung gezogen. Die Grosseltern und ihr Enkelkind bewegen sich dabei als virtuelle Avatare durch ein Bilderbuch hindurch. Mit Hilfe des Videochats erhalten die Avatare die Gesichter der Grosseltern und des Kindes. Der betagten Person wird der Text der Geschichte zum Vorlesen angezeigt, während das Kind nur die Bilder sieht. Über die Videochat-Verbindung können die beiden Spielenden so die Geschichte gemeinsam entdecken.\\\\
Es wurde entschieden, dass es sinnvoll ist, wenn die ältere Person die Kontrolle über das Spiel übernimmt. Um dies umzusetzen eignet sich eine asymmetrische Gestaltung der Spielmechanik, zum Beispiel durch das Verteilen von spezifischen Rollen. Gleichzeitig können so auch die unterschiedlichen Ansprüche der beiden Zielgruppen an das Game Design berücksichtigt werden. Dies war ebenfalls ein Grund für die Entscheidung eines rollenbasierten Spieles wie Palim-Palim.\\\\
Weiter begründet sich der Entscheid für Palim-Palim in dem Prinzip des Kaufladens, welches beiden Spielenden bereits aus dem realen Leben bekannt ist. Damit sollte eine tiefe Lernkurve für die Spielmechanik sowie ein gewisser Wiedererkennungswert geschaffen werden. Ausserdem verspricht dieses Konzept sehr viele Ausbaumöglichkeiten in Bezug auf die Spielmechanik und die Story. Von der ästhetischen und technologischen Seite her, bietet Palim-Palim mit seiner Sicht über den Tresen die Möglichkeit, den Videochat in unterschiedlichsten Ausprägungen in die Szene zu integrieren, ohne dabei die Spielmechanik grundlegend zu verändern.

\section{Spielkonzept}\label{spielkonzept}
Das Palim-Palim zugrunde liegende Spielkonzept lehnt sich dem Kinder-Kaufladen-Spielprinzip aus dem echten Leben an. Das Kind und die betagte Person schlüpfen dabei in einem virtuellen Einkaufsladen in die entsprechenden Rollen von Verkaufspersonal und Kundschaft. Initial nimmt die betagte Person die Rolle des Kunden oder der Kundin ein. Dies begründet sich darin, dass das Kind die Einkaufsliste eventuell noch nicht lesen kann. Die Spielenden sehen in Palim-Palim jeweils einen gemeinsamen Tresen aus der Perspektive ihrer Rolle (siehe Abbildung 1). Beide Seiten können Objekte auf dem Tresen platzieren, beziehungsweise Objekte vom Tresen wegziehen. Zusätzlich besteht die Möglichkeit, die Objekte auf das Videoelement des Gegenübers zu ziehen.\\\\
Durch die unterschiedlichen Rollen der beiden Spielenden lassen sich verschiedene Aktionen und somit ein abweichendenes Gameplay definieren. Diese Art der Unterteilung eignet sich hervorragend, um das Gameplay jeweils auf die spezifischen Anforderungen von Kindern beziehungsweise betagten Personen anpassen zu können.\\\\
Der Gameplay-Loop von Palim-Palim ist in der nachfolgenden Abbildung \ref{fig:PalimPalimGameplayloop} skizziert. Die Hauptinteraktionen zwischen den beiden Spielenden bestehen aus der Kommunikation der Einkaufsliste und dem Hin- und Herreichen von Objekten auf der gemeinsamen Tresen-Fläche.\\\\
		
\begin{figure}[!htb]
\includegraphics[width=\textwidth]{PalimPalim_Gameplayloop}
\caption[Caption for LOF]{Der Gameplayloop des Videospiels Palim-Palim (eigene Darstellung).}
\label{fig:PalimPalimGameplayloop}
\end{figure}
		
\section{Videomodi}\label{spielvarianten}
Damit der Einfluss der Videotelefonie als Teil des Spiels messbar wird, implementiert Palim-Palim unterschiedliche Videomodi. Dabei sind die ersten drei der folgenden Modi in Palim-Palim umgesetzt, die letzten beiden nur angedacht:
\begin{enumerate}
	\item Spiel ohne Videochat 
	\item Spiel mit Videochat, aber getrennt von der Spielszene 
	\item Spiel mit Videochat, in die Spielumgebung integriert 
	\item Spiel mit Videochat, in die Spielumgebung integriert und optionale Interaktionsmöglichkeiten mit dem Videostream 
	\item Spiel mit Videochat, in die Spielumgebung integriert und zum Spielerfolg notwendige Interaktionsmöglichkeiten mit dem Videostream
\end{enumerate}
In der ersten Variante sehen sich die Spielenden nicht. Sie können aber per Audio miteinander kommunizieren. Wie in der Abbildung \ref{fig:PalimPalimVideomodus1} ersichtlich ist, werden die Verkaufsperson sowie die einkaufende Person durch Illustrationen dargestellt.
\begin{figure}[!h]
\includegraphics[width=\textwidth]{PalimPalim_Videomodus_1}
\caption[Caption for LOF]{Das Videospiel Palim-Palim im Videomodus 1. Perspektive der betagten Person (links) und des Kindes (rechts) (eigene Darstellung).}
\label{fig:PalimPalimVideomodus1}
\end{figure}\\
Der Screenshot des zweiten Videomodus (Abbildung \ref{fig:PalimPalimVideomodus2}) zeigt, dass in diesem Modus die Videos der Spielenden unabhängig von der Gameszene in einer Ecke des Bildschirms dargestellt werden. Identisch zum ersten Modus werden die Personen durch Illustrationen ersetzt.\\\\
\begin{figure}[!h]
\includegraphics[width=\textwidth]{PalimPalim_Videomodus_2}
\caption[Caption for LOF]{Das Videospiel Palim-Palim im Videomodus 2. Perspektive der betagten Person (links) und des Kindes (rechts) (eigene Darstellung).}
\label{fig:PalimPalimVideomodus2}
\end{figure}\\
Der Videomodus 3 rendert die Videos der Spielenden an die Position der ein- und verkaufenden Person. Dadurch soll der Eindruck entstehen, das Kind stünde hinter dem Tresen und verkauft Objekte. Die Grossmutter oder der Grossvater des Kindes ist analog auf der anderen Seite des Tresens platziert und spielt die Kundschaft. Diese Szenerie wird in Abbildung \ref{fig:PalimPalimVideomodus3} gezeigt.
\begin{figure}[!h]
\includegraphics[width=\textwidth]{PalimPalim_Videomodus_3}
\caption[Caption for LOF]{Das Videospiel Palim-Palim im Videomodus 3. Perspektive der betagten Person (links) und des Kindes (rechts) (eigene Darstellung).}
\label{fig:PalimPalimVideomodus3}
\end{figure}\\

\section{Design und Usability}\label{DesignUndUsability}
Wie im Kapitel \ref{spielkonzept} beschrieben, ist ein asymmetrisches Gameplay für intergenerationelle Spiele sehr geeignet. Der gesamte Prozess vom Aufruf der Website bis zum Starten des Spiels ist ebenfalls asynchron gestaltet. Die betagte Person übernimmt dabei die zentrale Rolle. Sie betritt als erstes den Videochat-Raum, startet das Spiel und wählt den Spiel- und Videomodus aus. Dies wurde aufgrund des Alters der Kinder-Zielgruppe so definiert, da sie vielleicht noch nicht lesen können. Wie im Kapitel \ref{resultateKinder} beschrieben, haben die Spieletests jedoch gezeigt, dass diese Umsetzung nicht optimal ist. Wie eine synchrone Umsetzung der Gamelobby aussehen könnte, wird im Kapitel \ref{weiterentwicklungen} beschrieben.\\\\
Palim-Palim ist im Querformat umgesetzt und aktuell nicht für Hochformat optimiert. Der Hauptgrund dafür ist die bessere Stabilität, wenn das Tablet mit einer Halterung aufgestellt wird. Dies ist für ein Videospiel mit einer zentralen Funktionalität der Kamera wichtig, denn wenn das Tablet auf den Tisch gelegt wird, ist der Bildausschnitt suboptimal.\\\\
Die Spielszene von Palim-Palim ist als 3D-Welt umgesetzt. Der wichtigste Grund dafür ist, dass dadurch ein Raumgefühl entsteht. Des Weiteren ist damit die Möglichkeit vorhanden, eine Physik-Engine zu implementieren. Somit könnten Objekte herunterfallen, in alle Richtungen rollen oder geworfen werden. Eine genauere Erklärung einer solchen Erweiterung ist im Kapitel \ref{weiterentwicklungen} zu finden.\\\\
Der Hintergrund der Szene soll zeigen, dass es sich um einen Kaufladen handelt. Die Erkennbarkeit der Details liegt dabei nicht im Vordergrund, da diese für das Spiel nicht entscheidend sind.\\\\
Für die 3D-Objekte wurden möglichst realitätsgetreue Modelle verwendet, unter der Annahme, dass diese von den betagten Menschen einfacher erkannt werden können, im Gegensatz zu abstrakteren Illustrationen.\\\\
Das Verschieben von Objekten mit dem Finger erfordert eine gewisse Präzision. Um eine höhere Trefferwahrscheinlichkeit zu gewährleisten und damit die Benutzerfreundlichkeit zu erhöhen, implementiert Palim-Palim um jedes verschiebbare Objekt eine etwas grössere Hitbox. Diese bietet den Benutzer:innen zusätzliche Fläche zur Auswahl der Objekte. Die genaue Beschreibung der Interaktion mit Objekten und die Implementation der Hitbox ist im Kapitel \ref{interaktionen} dokumentiert.


\chapter{Technologien}
Palim-Palim kombiniert die Funktionalitäten einer Videochat-Anwendung mit einem Multiplayer-Spiel. Um den Anforderungen dieser Kombination gerecht zu werden, wurden entsprechende Technologien zur Umsetzung evaluiert. Dabei fiel die Wahl auf WebRTC (Web Real-Time Communication) \cite{noauthor_webrtc_2011} zur Übertragung der Multimedia-Daten sowie Three.js \cite{noauthor_threejs_nodate-1} als 3D-Bibliothek zur Gestaltung und Umsetzung der Gameplay-Szene. In diesem Kapitel wird als erstes die Technologiewahl begründet. Anschliessend werden die verwendeten Frameworks und deren Funktionen vorgestellt.
	
\section{Technologiewahl}
Die Auswahl von WebRTC begründet sich hauptsächlich in der hohen Etablierung des Standards im Web. Zudem wird es von allen wichtigen Browsern unterstützt \cite{noauthor_webrtc_nodate-3}.\\\\
Für die Umsetzung der Spiellogik wurde vorerst Unity \cite{technologies_unity_nodate} als Game-Engine in Betracht gezogen. Denn mit dem «WebRTC for Unity»-Package \cite{noauthor_webrtc_2021-1} besteht die Möglichkeit, einen Videochat in ein Unity-Spiel zu integrieren. Bei der Evaluierung der Technologien wurde klar, dass eine möglichst hohe Kontrolle über den Video-Stream benötigt wird, weil dieser interaktiv und Teil des Gameplays von Palim-Palim sein sollte. Die Ansprüche eines Spieles, in dem der Videochat eine sehr zentrale Rolle spielt, sind Funktionen wie die Gesichtserkennung für eventuelle Filtereffekte sowie die Segmentierung des Hintergrundes zum Freistellen der Person im Video. Hierzu bietet Unity selbst wenig Unterstützung. Es braucht dafür kostenpflichtige Plugins wie das Face Tracking Plugin von Banuba \cite{noauthor_introducing_nodate}. Allerdings ist unklar, ob mit solchen Plugin-Lösungen für Unity alle Bedürfnisse von Palim-Palim erfüllt werden können. Zusätzlich wurde entschieden, keine Abhängigkeiten zu den Autor:innen dieser Plugins aufzubauen. Aufgrund dessen wurde Unity als unpassende Lösung eingestuft.\\\\
Alternativ wurde die Umsetzung von Palim-Palim als Webanwendung mit JavaScript geprüft. JavaScript ermöglicht die Nutzung von unzähligen Open-Source-Bibliotheken und bietet eine hohe Flexibilität in der Umsetzung. Bestehende Bibliotheken können beispielsweise für Gesichtsfiltereffekte \cite{noauthor_tfjs-modelsface-landmarks-detection_nodate} oder zur Segmentierung des Hintergrundes \cite{noauthor_tfjs-modelsbody-pix_nodate} eingesetzt werden. Aus diesen Gründen wurde JavaScript zur technischen Umsetzung von Palim-Palim gewählt.\\\\
Mit dem Ziel, eine realistisch wirkende Spielumgebung für Jung und Alt zu schaffen, wurde nach einem passenden Framework gesucht. Die Wahl fiel dabei auf Three.js \cite{noauthor_threejs_nodate}, da diese Bibliothek eine zeitgemässe Möglichkeit zur Erschaffung einer immersiven 3D-Spielwelt bietet.
	
\section{WebRTC}
\label{webRTC}
WebRTC ist ein Open-Source-Projekt, das die Echtzeitkommunikation von Audio, Video und Daten in Web- und nativen Anwendungen ermöglicht \cite{noauthor_webrtc_2011}. Die Technologie ist eine Recommandation des World Wide Web Consortium (W3C) \cite{noauthor_webrtc_2021} und ein Standard der Internet Engineering Task Force (IETF) \cite{noauthor_support_2021}. WebRTC ist in allen modernen Browsern sowie auf nativen Clients für alle wichtigen Plattformen verfügbar. Dabei wird zwischen zwei Browsern eine Peer-to-Peer-Verbindung aufgebaut, worüber die Daten gestreamt werden. Auf der Peer-to-Peer-Verbindung lassen sich auch eigene Datenkanäle erstellen, die zum Beispiel zur Übertragung von Textnachrichten, Positionsdaten von Spielobjekten oder sogar Dokumenten verwendet werden können \cite{noauthor_webrtc_nodate}. Damit der Verbindungsaufbau reibungslos funktioniert, bedient sich WebRTC einiger Serververbindungen, welche in den folgenden beiden Kapiteln erläutert werden.

\subsection{Signaling-Server}\label{signaling-server}
WebRTC verwendet die RTCPeerConnection JavaScript API (Application Programming Interface), um Streaming-Daten direkt zwischen Browsern zu übermitteln \cite{noauthor_rtcpeerconnection_nodate}. Um eine direkte Verbindung zwischen zwei sich unbekannten Clients herzustellen, wird ein Prozess benötigt, welcher die Kommunikation der beiden Peers koordiniert und Kontrollnachrichten sendet. Dieser Prozess wird in WebRTC als \textit{Signaling} bezeichnet. Signaling-Methoden und -protokolle sind von WebRTC nicht spezifiziert und können je nach Anwendungsfall entsprechend gewählt werden. WebRTC-Anwendungen, welche skalierbar sein sollen, benötigen Signaling-Server, welche in der Lage sind eine erhebliche Last zu bewältigen. Solche Lösungen wurden im Rahmen von Palim-Palim nicht benötigt, weshalb an dieser Stelle lediglich auf entsprechende Ressoucen verwiesen wird \cite{article_build_nodate} \cite{venema_how_nodate}.\\\\
Für einfache Anwendungen mit geringen Anforderungen an die Skalierung der Benutzeranzahl bietet sich die JavaScript-Library Socket.io für das Signaling an \cite{arrachequesne_socketio_2021}. Socket.io erlaubt eine einfache und bidirektionale Kommunkation zwischen den Clients und einem Signaling-Server. Mit seinem integrierten Room-Konzept eignet sich Socket.io zudem sehr gut für eine Videochat-App. Der Server hat dabei folgende zwei Aufgaben: er leitet Relais Nachrichten an die Peers weiter und verwaltet alle Videochat-Räume. \\\\
Die Funktionaliät als Nachrichten-Relais ist wichtig, da sich die beiden Clients vor dem Verbindungsaufbau noch nicht kennen. Der Signaling-Server ist dabei für beide Clients der erste gemeinsame Kontaktpunkt. Nur so können initiale Informationen der Clients ausgetauscht werden, damit diese untereinander eine Peer-to-Peer-Verbindung aufbauen können.
		

\subsection{STUN- und TURN-Server}			
Nebst einem Signaling-Server werden für WebRTC-Anwendungen zusätzlich ein STUN- sowie ein TURN-Server benötigt. WebRTC ist grundsätzlich so konzipiert, dass es Peer-to-Peer funktioniert. Zwei Clients können sich also auf direktem Weg verbinden. Die Technologie ist jedoch auch bewusst darauf ausgelegt, mit komplexeren Netzwerken zurechtzukommen: Client-Anwendungen müssen NAT-Gateways (Network Address Translation) und Firewalls überwinden. Des weiteren benötigen Peer-to-Peer-Netzwerke eine Absicherung, falls die direkte Verbindung ausfällt. Als Teil dieses Prozesses verwendet die WebRTC-API zwei Netzwerkprotokolle als Hilfsmittel: 			
\begin{itemize}
  \item \textit{STUN} (Session Traversal Utilities for NAT), um die öffentliche IP-Adresse (Internet Protocol) und Portnummer für den direkten Kontaktaufbau zu ermitteln  \cite{noauthor_stun_2021}.
  \item \textit{TURN} (Traversal Using Relay NAT), welches die Kommunikation über NAT- oder Firewallgrenzen hinweg ermöglicht und als Fallback-Relay genutzt werden kann, falls eine Peer-to-Peer-Verbindung nicht aufgebaut werden kann \cite{noauthor_traversal_2021}.
\end{itemize}
Für beide Protokolle müssen Server bereitgestellt werden, welche den WebRTC-Clients bekannt sein müssen. Die STUN- und TURN-Adressen müssen beim Aufbau der Peer-Connection der Clients an den Signaling-Server übermittelt werden \cite{noauthor_turn-server_nodate-1}. Wie dies in Palim-Palim umgesetzt wurde, ist im Kapitel \ref{server} genauer beschrieben.\\\\
Als Fallback-Verbindung wird im produktiven Betrieb einer WebRTC-Anwendung unbedingt ein eigener TURN-Server mit einer öffentlich sichtbaren IP-Adresse benötigt \cite{noauthor_webrtc_2020}. Da der Netzwerkverkehr über einen solchen Server sehr stark ansteigen kann, werden keine zuverlässigen, kostenlosen, öffentlich gehosteten TURN-Server angeboten.\\\\
Gemäss Auswertungen des WebRTC Call Quality Dienstleisters callstats.io haben im Durchschnitt etwa 30 Prozent aller WebRTC-Sessions einen Client, der sich über einen TURN-Server verbindet \cite{callstats_why_nodate}. Das heisst in etwa 70 Prozent der Fälle kann die Peer-to-Peer-Verbindung direkt aufgebaut werden. In diesem Fall wird ein TURN-Server überhaupt nicht benötigt. Trotzdem muss bei jeder Verbindung mindestens eine TURN-Serveraddresse angegeben werden. So kann garantiert werden, dass alle Clients in der Lage sind, miteinander zu kommunizieren.\\\\
\begin{figure}[h!]
\includegraphics[width=\textwidth]{WebRTC_nat_stun_firewall_turn_black}
\caption[Caption for LOF]{STUN, TURN, und Signaling in WebRTC (eigene Darstellung).}
\end{figure}
			
\subsection{Sicherheit}
Auf der Protokollebene gilt WebRTC als sehr sicherer und auch etablierter Standard. Verschlüsselung ist für alle WebRTC-Komponenten obligatorisch und seine JavaScript-APIs können nur von sicheren Quellen (HTTPS \cite{rescorla_http_2000} oder localhost) aus verwendet werden \cite{noauthor_webrtc_2020-1}. Das verwendete Verschlüsselungsprotokoll hängt dabei vom Kanaltyp ab. Datenkanäle werden mit Datagram Transport Layer Security (DTLS) \cite{rescorla_datagram_2012} und Medienkanäle mit Secure Real-time Transport Protocol (SRTP) \cite{norrman_secure_2004} verschlüsselt. Die Verbindung zum Signaling-Server ist allerdings nicht standardmässig verschlüsselt. Bei Anwendungen mit Schwerpunkt auf die Vertraulichkeit sollte deshalb für die Kommunikation mit dem Signaling-Server ein sicheres Protokoll wie SIPS \cite{schooler_sip_2002}, OpenSIP \cite{noauthor_opensips_nodate}, HTTPS oder WebSockets over SSL/TLS (WSS) \cite{melnikov_websocket_2011} verwendet werden \cite{noauthor_study_nodate}.\\\\
Bei der Umleitung des Datenverkehrs über den TURN-Server die Daten nicht interpretiert oder modifiziert. Dies ist so im TURN-Standard definiert \cite{noauthor_rfc5766_nodate}. Das heisst, die Verwendung eines TURN-Servers fügt dem WebRTC-Datenverkehr keine Sicherheitsschwachstellen hinzu.\\\\
Somit lässt sich WebRTC als eigenständiges Framework für Videotelefonie als sehr sicher bezeichnen. Allerdings hängt diese Sicherheit auch von der umliegenden Webanwendung und diese wiederum vom verwendeten Browser ab. Ein im Internet zugängliches Spiel sollte deshalb über eine entsprechende Benutzerauthentifizierung verfügen, wenn es dauerhaft produktiv betrieben wird. Eine solche Authentifizierung darf nicht über den Signaling-Server laufen, denn der Datenverkehr zu ihm könnte abgefangen werden. Jede Client-Anwendung muss in der Lage sein, die Authentifizierung der potenziellen Gesprächspartner unabhängig vom Signaling-Server durchzuführen. Dies kann durch die Verwendung eines webbasierten Identitätsanbieters (IdP) erreicht werden, wie zum Beispiel Facebook Connect \cite{noauthor_facebook_nodate} oder OAuth \cite{noauthor_oauth_nodate}. Ebenfalls empfiehlt es sich, die WebRTC-Bibliothek und andere Abhängigkeiten aktuell zu halten \cite{noauthor_study_nodate}.
	
\section{Three.js}\label{threejs}
Three.js ist eine 3D-Bibliothek, welche die Web Graphics Library (WebGL) \cite{noauthor_notitle_nodate} verwendet, um 3D-Inhalte im Browser darzustellen \cite{noauthor_threejs_nodate}. WebGL ist eine Rasterisierungs-Engine, das bedeutet, dass sie nur Punkte, Linien und Dreiecke zeichnet. Erst auf Basis des Codes entstehen dreidimensionale Objekte \cite{noauthor_webgl_nodate}. In Three.js stehen primitive 3D-Objekte genauso wie beispielsweise Lichter, Schatten, Materialien oder Texturen als vorgefertigte Ressourcen zur Verfügung \cite{noauthor_threejs_nodate}. Diese werden in einer Struktur angeordnet, welche beispielhaft in Abbildung \ref{fig:threejsstructure} ersichtlich ist.\\\\
\begin{figure}[h!]
\includegraphics[width=\textwidth]{threejs_structure}
\caption[Caption for LOF]{Die Struktur einer typischen three.js-Anwendung (Quelle \cite{noauthor_threejs_nodate}).}
\label{fig:threejsstructure}
\end{figure}
Das oberste Element einer Three.js-Struktur ist der Renderer. Dieser nimmt eine Szene (Scene) und eine Kamera (Camera) entgegen. Er zeichnet den im Blickfeld befindlichen Teil der 3D-Scene auf eine 2D-Leinwand. In der Scene werden Lichter (Light), 3D-Objekte (Object3D) und Kameras (Camera) platziert. Die Scene ist die Wurzel einer Baumstruktur, in welcher Kinder relativ zu ihren Eltern ausgerichtet werden. Mesh-Objekte sind gezeichnete Formen (Geometry) mit einem bestimmten Material. Geometry- und Material-Objekte können von verschiedenen Mesh-Objekten verwendet werden. Three.js bietet bereits einige primitive Geometry-Objekte an, wie beispielsweise Würfel (Cube), Zylinder (Cylinder) oder Kegel (Cone). Es können aber auch eigene Formen erstellt oder aus einer Datei importiert werden. Material-Objekte repräsentieren die Oberflächenbeschaffenheit eines Objekts, wie zum Beispiel die Farbe oder wie fest ein Objekt spiegelt \cite{noauthor_threejs_nodate}.

\chapter{Implementation}
In diesem Kapitel werden die einzelnen Komponenten von Palim-Palim und deren Implementierung vorgestellt. Als erstes werden in den Kapiteln \ref{PeerConnectionManager} und \ref{game} alle zentralen Client-Komponenten vorgestellt. In Kapitel \ref{server} wird anschliessend das serverseitige Backend erklärt.\\
Die Struktur von Palim-Palim ist bewusst so aufgebaut, dass die Videochat-Funktionalität möglichst von der Gamelogik entkoppelt ist. Dadurch ist es einfacher, den Videochat auch in anderen Spielen zu integrieren. Ebenfalls wurde damit eine möglichst solide Grundlage zur Entwicklung eines entsprechenden Frameworks gelegt. Dies spiegelt sich auch im Klassndiagramm wieder (siehe Abbildung \ref{fig:AppClassDiagram}). Das Spiel hat dabei nur eine einzige Abhängigkeit zur PeerConnectionManager-Klasse, welche den Videochat implementiert.
\begin{figure}[!h]
\includegraphics[width=\textwidth]{AppClassDiagram}
\caption[Caption for LOF]{Klassendiagramm der wichtigsten Komponenten von Palim-Palim (eigene Darstellung).}
\label{fig:AppClassDiagram}
\end{figure}

\section{PeerConnectionManager}\label{PeerConnectionManager}
Die PeerconnectionManager-Klasse übernimmt zentrale Funktionen wie das Betreten eines Raumes, die Etablierung der Video-Streams zwischen den Spielenden sowie den Aufbau von dedizierten Datenkanälen (sogenannten DataChannels).\\\\ 
Um alle diese Funktionen innerhalb dieser Klasse möglichst übersichtlich umzusetzen, ist die PeerConnectionManager-Klasse von Palim-Palim selbst in einzelne Unterklassen mit den entsprechenden Teilverantwortlichkeiten aufgeteilt. Diese Komponenten sind in der Abbildung \ref{fig:PeerConnectionManagerDiagram} dargestellt. Die einzelnen Zuständigkeiten der Klassen werden in den nachfolgenden Kapiteln \ref{roomManager}, \ref{videoChatManager} und \ref{dataChannelManager} vorgestellt.
\begin{figure}[!htb]
\includegraphics[width=\textwidth]{PeerConnectionManagerDiagram}
\caption[Caption for LOF]{Klassendiagramm der PeerConnectionManager-Klasse und deren Subklassen (eigene Darstellung).}
\label{fig:PeerConnectionManagerDiagram}
\end{figure}\\\\
In der PeerConnectionManager-Klasse selbst findet der Aufbau der Peer-to-Peer-Verbindung statt. Die Klasse instanziiert hierzu ein PeerConnection-Objekt, welches die RTCPeerConnection erweitert. RTCPeerConnection ist die API, welche von WebRTC-Anwendungen verwendet wird, um eine Verbindung zwischen Peers herzustellen sowie Audio und Video zu übertragen \cite{noauthor_rtcpeerconnection_nodate-1}. Dieses PeerConnection-Objekt stellt eine WebRTC-Verbindung zwischen dem lokalen Computer und einem entfernten Client dar. Sie bietet Methoden, um eine Verbindung zur Gegenseite herzustellen, die Verbindung aufrechtzuerhalten und zu überwachen sowie die Verbindung zu schliessen, wenn sie nicht mehr benötigt wird. Der PeerConnectionManager orchestriert über dieses Objekt den Verbindungsaufbau und dient als einzige Schnittstelle des Videochats zum Rest der Applikation.

\newpage
\subsection{RoomManager}\label{roomManager}
Der RoomManager ist die einzige Schnittstelle zum Server und für das Signaling zuständig. Er besitzt die Verantwortung über die clientseitige Socket.io-Verbindung. Der RoomManager ermöglicht es somit einem Client via Socket.io einem Raum beizutreten. Ebenfalls wird er vom PeerConnectionManager verwendet, um Signaling-Nachrichten an den Server zu versenden sowie diese zu empfangen. Auf dem Server werden die Anfragen der einzelnen Clients verarbeitet, wie in Kapitel \ref{server} beschrieben ist.\begin{figure}[!h]
\includegraphics[width=\textwidth]{RoomManagerSd_noFrame}
\caption[Caption for LOF]{Nachrichtenfluss beim Betreten eines Raumes (eigene Darstellung).}
\label{fig:sequenzJoinRoom}
\end{figure}\\\\
Ein typischer Nachrichtenfluss ist in Abbildung \ref{fig:sequenzJoinRoom} dargestellt. Client A beginnt denn Prozess mittels einer «create or join»-Nachricht. Diese bewirkt das Erstellen eines neuen Raumes auf dem Signaling-Server. Die Nachricht enthält zusätzlich einen Identifikator des Raumes. Der Server erstellt diesen Socket.io-Raum und bestätigt dies Client A mit einer «created»-Nachricht. Client B möchte nun dem Raum mit dem gleichen Identifikator beitreten. Der Server erkennt, dass dieser Raum bereits existiert. Er schickt daher eine «joinRequest»-Nachricht an Client A. An diesem Punkt könnte von Client A die Authentifizierung des potenziellen Peers stattfinden. Palim-Palim implementiert hierzu allerdings noch keine Logik. Client B wird automatisch dem Raum von Client A zugewiesen. Über den gemeinsamen Socket.io-Raum wird anschliessend der typische WebRTC-Signaling-Prozess \cite{noauthor_webrtc_nodate-4} durchgeführt.

\subsection{VideoChatManager}\label{videoChatManager}
Sobald der Spielende einen Raum betritt, wird via VideoChatManager der lokale Video- und Audiostream initiiert. Dies geschieht via Navigator \cite{noauthor_navigator_nodate}, wodurch der Browser um die Erlaubnis zur Verwendung des Mikrofons und der Kamera bittet. Erst wenn diese Erlaubnis gegeben wurde, kann auf die Daten von Kamera und Mikrofon zugegriffen werden. Der VideoChatManager fügt dann den Stream dem entsprechenden DOM-Element (Document Object Model) hinzu, wodurch die Spielenden ihr eigenes Video sehen können. Ebenfalls löst dies in Palim-Palim ein «got user media»-Event aus, wodurch der PeerConnectionManager sowie der Signaling-Server informiert werden, dass dieser Client jetzt Zugriff auf lokale Video- und Audio-Daten hat. Diese Daten werden als MediaStreamTrack-Objekt \cite{noauthor_media_nodate} der PeerConnection hinzugefügt.\\\\
Auf der Gegenseite wird durch das Hinzufügen des Tracks ein "trackAdded"-Event ausgelöst \cite{noauthor_webrtc_nodate-1}. Aus diesem Event extrahiert der VideoChatManager den Remotevideo-Stream. Dieser wird dem DOM hinzugefügt, wodurch eine Videokonferenz zwischen den beiden Mitspielenden entsteht. Dieser Prozess ist als Sequenzdiagramm in Abbildung \ref{fig:videoSequenz} veranschaulicht.
\begin{figure}[!h]
\includegraphics[width=\textwidth]{VideoChatManager}
\caption[Caption for LOF]{Sequenzdiagramm zur Erstellung der Video-Verbindung (eigene Darstellung).}
\label{fig:videoSequenz}
\end{figure}\\\\

\subsection{DataChannelManager}\label{dataChannelManager}
Die dritte Teilkomponente des PeerConnectionManagers ist der DataChannelManager, welcher für das Erstellen der DataChannels zuständig ist. Diese erlauben das Versenden beliebiger Daten über die PeerConnection \cite{noauthor_webrtc_nodate}. In Palim-Palim wird diese Funktion genutzt, um die 3D-Positionen der Gegenstände zu synchronisieren, sowie den anderen Client über spezifische Events im Spiel zu informieren. Dazu werden zwei einzelne DataChannels erstellt, welche unabhängig voneinander, aber auf der gleichen PeerConnection laufen.\\\\
Ein DataChannel bietet Properties, um die Übertragungsart genau an die Anforderungen anzupassen. Zum Beispiel kann mittels dem Property «ordered» bestimmt werden, ob die Nachrichten in der gleichen Reihenfolge ankommen wie sie versendet wurden. WebRTC baut dann im Hintergrund eine Verbindung auf. Je nach Konfiguration des DataChannel-Objekts wird für diese TCP (Transmission Control Protocol) oder UDP (User Datagram Protocol) verwendet. \cite{noauthor_rtcdatachannel_nodate}. Beim Erstellen eines DataChannels muss dabei die Callback-Funktion angegeben werden, mit welcher eingehende Nachrichten behandelt werden. Deshalb wird im DataChannelManager eine Referenz auf den GameSyncManager von Palim-Palim benötigt, welcher im nachfolgenden Kapitel \ref{game} genauer beschrieben ist.\\\\

\section{Game}
\label{game}
Neben dem PeerConnectionManager als zentrale Steuereinheit der Videochat-Funktionalität, bildet der GameManager die zweite zentrale Klasse. Er übernimmt die zentralen Spiel-Funktionen von Palim-Palim. Dazu gehört das Starten des Spiels, die Verwaltung der Gamelobby und die Koordination von Spielgeschehnissen. Um diese Funktionen und das komplette Handling des Spiels übersichtlich zu gestalten, besitzt der GameManager je eine Instanz von weiteren Managern. Diese übernehmen wiederum Aufgaben, welche nachfolgend genauer erläutert werden.\\\\
Der \textbf{GameLobbyManager} dient zur Verwaltung der Gamelobby. Er besitzt die Möglichkeit den Eröffnungsscreen, die Einstellungsscreens, die Erfolgsmeldung sowie auch den Screen für das Ende des Spiels und den Neustart anzuzeigen. Ebenfalls verwaltet er die getätigten Eingaben der Benutzenden. Einige Eingaben führen zu weiteren Screens, andere lösen über den GameSyncManager eine Nachricht an den anderen Client aus.\\\\
Der \textbf{GameSyncManager} hält die beiden Clients synchron. Vom GameLobbyManager oder vom GameManager aus können über ihn Nachrichten an den anderen Client gesendet werden. Bei einkommenden Nachrichten entscheidet der GameSyncManager, welche Events dadurch ausgelöst werden. Wie die genaue Synchronisation der Spielobjekte funktioniert, ist im Kapitel \ref{gameplaySync} beschrieben.\\\\
Der \textbf{SceneManager} ist die zentrale Steuereinheit der 3D-Spielszene. Alle Änderungen an der Scene geschehen über ihn. Das Laden der 3D-Objekte, welches ebenfalls vom SceneManager übernommen wird, ist im nachfolgenden Kapitel \ref{3DObjekteLaden} genauer erklärt.\\\\
Der \textbf{AudioManager} ist befähigt Audio-Dateien abzuspielen.\\\\
Über den \textbf{ShoppingListManager} wird die Einkaufsliste zufällig generiert. Dabei werden zwischen drei und fünf Verkaufsobjekte aus den möglichen Objekten ausgewählt, in einer Map gespeichert und über den SceneManager im Spiel angezeigt.\\\\
Der \textbf{GameStateManager} überprüft, ob das Ziel des Spiels bereits erreicht wurde. Dabei wird die Einkaufslisten-Map mit den Objekten im virtuellen Einkaufskorb (Instanz der Basket-Klasse) abgeglichen. Dies erfolgt jeweils beim Hinzufügen eines Objekts in den Einkaufskorb. Im Erfolgsfall wird über den GameSyncManager eine Nachricht an den anderen Client verschickt. Gleichzeitig wird lokal ein entsprechendes Event ausgelöst. Der gesamte Prozess vom Ziehen eines Objekts in den Einkaufskorb bis zum Ende des Spiels wird auf dem Sequenzdiagramm im Anhang \ref{Sequenzdiagramme} auf Abbildung \ref{fig:SequenzdiagrammEinkaufskorbSpielEnde} gezeigt.\\\\
Der \textbf{InteractionManager} sorgt dafür, dass Benutzerinteraktionen in der 3D-Welt zu einer Aktion führen. Das genaue Handling dieser Interaktionen wird im Kapitel \ref{interaktionen} beschrieben. In welchen Klassen dabei welche Methoden und Funktionen aufgerufen werden und was alles mit dem Verschieben eines Objekts zusammenhängt, zeigt das Squenzdiagramm im Anhang \ref{Sequenzdiagramme} auf Abbildung \ref{fig:SequenzdiagrammVorratTresen}.\\\\
	
\subsection{3D-Objekte}
\label{3DObjekteLaden}
Um lange Ladezeiten zu verhindern, sollte in Webanwendungen die Datenmenge möglichst gering gehalten werden. Da 3D-Objekte eher grosse Dateien sind, ist es gerade bei deren Verwendung wichtig, einige Punkte zu beachten. Nachfolgend ist dokumentiert welche Optimierungen in Palim-Palim diesbezüglich implementiert sind.\\\\
Während sich die Spielenden noch in der Gamelobby befinden, lädt Palim-Palim alle Spielmodi-übergreifenden 3D-Objekte wie den Tresen oder den Einkaufskorb. Die vom Spielmodi abhängigen 3D-Objekte (Einkaufsgegenstände) werden geladen, während die betagte Person die Anleitung liest und das Kind das Spiel erklärt bekommt.\\\\
Für die 3D-Objekte verwendet Palim-Palim Dateien des Graphics Language Transmission Formats (glTF). Dieses Dateiformat kann 3D-Modelle sehr effizient übertragen und laden, weswegen es sich sehr gut für Webapplikationen eignet \cite{noauthor_gltf_2020}. In einer glTF-Datei können nicht nur die Geometrie eines 3D-Objekts gespeichert werden, sondern auch Szenen, Kameras, Materialien, Texturen oder Animationen (nicht abschliessende Liste) \cite{noauthor_khronosgroupgltf_2021}. Deswegen muss in Palim-Palim beim Laden der 3D-Objekte zuerst die Szene der glTF-Datei ausgelesen und danach in dieser nach Meshes gesucht werden (Siehe Codefragment \ref{lst:loadGLTF}).
\begin{lstlisting}[caption={glTF-3D-Dateien laden durch traversieren der glTF-Szene.},label={lst:loadGLTF},language=JavaScript]
const loadedData = await loader.loadAsync(model);
loadedData.scene.traverse((node) => {
    if (node.isMesh) {
	 	...
    }
});
\end{lstlisting}
Die in Palim-Palim verwendeten 3D-Objekte wurden mit Blender \cite{foundation_blenderorg_nodate} auf eine möglichst kleine Dateigrösse reduziert, um die Ladezeit zu optimieren. Dazu wurde der «Decimate Modifier» angewendet, welcher erlaubt, die Polygone einer Geometrie zu reduzieren und dabei den ursprünglichen Körper nur minimal zu verändern \cite{noauthor_decimate_nodate}. Zur besseren Verwaltung in Palim-Palim wurden in Blender zusätzlich die zwei folgenden Modifikationen an den ursprünglichen glTF-Dateien durchgeführt. Einerseits wurden Objekte, die aus mehreren Meshes bestanden, zu einem Mesh zusammengefügt. Anderseits wurde der Mittelpunkt des Meshes auf den Koordinatenursprung gelegt. Die Skalierung des Meshes hingegen kann über einen Parameter in der Konfigurationsdatei von Palim-Palim festgelegt werden. Somit können die Grössenverhältnisse der einzelnen Objekte aufeinander abgestimmt werden.
	
\subsection{Interaktionen}
\label{interaktionen}
Es nicht intuitiv, Objekte auf einem Screen so zu steuern, dass diese in einem 3D-Raum bewegt werden. Deswegen gibt es verschiedene Lösungsmöglichkeiten, wie dies bewerkstelligt werden kann. Beispielsweise ist es möglich, das Gyroskop in die Steuerung miteinzubeziehen. Dabei könnte die Dimension in die Tiefe nur angesteuert werden, wenn das Tablet flach liegt (Winkel zur Horizontalen kleiner als 20 Grad), die vertikale Dimension nur, wenn das Tablet «steht» (Winkel zur Horizontalen grösser als 20 Grad). Dies ist für Palim-Palim allerdings nicht geeignet, da so die Videoaufnahme sehr unruhig wird. Eine weitere Möglichkeit ist es, die Ansteuerung der dritten Dimension in die Tiefe per Ziehen vom Mittelpunkt zu einer äusseren Ecke zu bewerkstelligen. Das Ziehen von der Mitte zur rechten oberen Ecke würde das Element nach hinten bewegen. Und das Ziehen zur rechten unteren Ecke würde das Element nach vorne bewegen. Dies haben beispielsweise Tseng et al. 2018 in ihrem Forschungsartikel so beschrieben \cite{tseng_ez-manipulator_2018}. Die Zielgruppe von Palim-Palim soll nicht mit neuen Gesten überfordert werden, weswegen diese Technologie ebenfalls ungeeignet ist.\\\\
Palim-Palim setzt auf eine simple 2D-Steuerung, welche sich jedoch auf einer gekippten Ebene abspielt. Diese Ebene, genannt Interaktionsebene, ist vom Bildschirm aus gesehen um 45 Grad nach hinten gekippt. Das gelbe Linienkonstrukt in Abbildung \ref{fig:SpielszeneMitHilfslinien} zeigt die Interaktionsebene des Verkaufspersonals. Wird nun ein Objekt angetippt, wird die Methode onPointerDown des InteractionManagers aufgerufen. Anhand eines Raycasts \cite{noauthor_raycaster_nodate} wird das Element ausgewählt, welches sich am nächsten zur Kamera des Interagierenden befindet. Dieses wird «aufgenommen» und kann nun in der Interaktionsebene bewegt werden. Da die Ebene nach hinten geneigt ist, gewinnt oder verliert das Objekt abhängig von der Höhe an Tiefe.\\\\
Wie im Kapitel \ref{DesignUndUsability} beschrieben, benötigen die Verkaufsobjekte für eine einfachere Bedienung eine grössere Touch-Fläche. Dazu errichtet Palim-Palim um jeden Verkaufgegenstand eine Hitbox in Form einer Kugel. Diese Kugel ist eine Begrenzungskugel um den kleinst möglichen Begrenzungsquader der Geometrie. Three.js stellt dafür die Methoden computeBoundingBox \cite{noauthor_buffergeometrycomputeboundingbox_nodate} und getBoundingSphere \cite{noauthor_box3getboundingsphere_nodate} zur Verfügung. Für den Einkaufskorb ist ebenfalls eine Hitbox vorhanden (roter Linienquader in Abbildung \ref{fig:SpielszeneMitHilfslinien}) um das Ablegen der Objekte im Korb zu erleichtern.\\\\
Als virtuelles Regal dient eine Box (blauer Linienquader in Abbildung \ref{fig:SpielszeneMitHilfslinien}) auf der Seite des Verkaufspersonals. Werden Objekte in den Bereich dieser Box gezogen, werden sie wieder an ihre Ursprungsposition im vituellen Regal gesetzt.
\begin{figure}[!htb]
\includegraphics[width=0.6\textwidth]{Spielszene_mit_Hilfslinien}
\caption[Caption for LOF]{Die Spielszene des Verkaufspersonals inklusive Hilfslinien, um virtuelle Ebenen und Quader zu sehen (eigene Darstellung).}
\label{fig:SpielszeneMitHilfslinien}
\end{figure}\\\\

\subsection{Gampeplay-Synchronisation}
\label{gameplaySync}
Um die Interaktionen der beiden Spielenden zu synchronisieren, nutzt Palim-Palim seine Peer-to-Peer-Funktionalität aus. Das Spiel kreiert dazu zwei spezifische DataChannels. Der eine Channel wird genutzt, um die Objektpositionen beider Clients zu synchronisieren. Jede Interaktion mit einem Verkaufsgegenstand bewirkt den Versand eines JSON-Objekts. Dieses beinhaltet den Objektschlüssel und die Positionsdaten des Verkaufsobjekts. Das JSON-Objekt wird vom anderen Client empfangen und über den GameSyncManager verarbeitet. Der zweite DataChannel sendet Spielevents wie «startGame», «basketAdd» oder «gameOver». Diese werden im Gegensatz zu den Objektpositionen nur punktuell versendet.\\\\
Diese Art der Gameplay-Synchronisation erlaubt es den Clients, total unabhängig von einem Server zu spielen. Da der Server nur fürs Signaling zuständig ist, wird eine Skalierung der Spieleranzahl vereinfacht. Der Server muss die Spielenden nur inital vermitteln. Alle Spieleinscheidungen sowie der Abgleich der Spielwelt werden von den Clients direkt vorgenommen.
		
\section{Server}\label{server}
Da Palim-Palim über das Internet zugänglich ist, wird ein Server benötigt, welcher den Client-Code zur Verfügung stellt. Der Palim-Palim-Server nutzt Express.js \cite{noauthor_express_nodate}. Damit wird der JavaScript-Code den Benutzer:innen beim Aufruf der Webapplikation zur Verfügung gestellt.\\\\
Der Server dient als Vermittler für den Verbindungsaufbau des Videochats. Dies wird auch Signaling genannt, wie bereits in Kapitel \ref{signaling-server} erläutert wurde. In Palim-Palim funktioniert das Signaling folgendermassen: Sobald zwei Spielende einem Raum betreten, wird dieser genutzt, um Verbindungsinformationen zwischen den Clients auszutauschen. Anhand dieser Informationen können die Clients dann untereinander eine direkte Peer-to-Peer-Verbindung herstellen. Palim-Palim setzt dies gemäss einem Google Codelab Beispiel mit Socket.io um \cite{arrachequesne_socketio_2021} \cite{noauthor_real_nodate}.  \\\\
Um unabhängig von kostenpflichtigen Anbietern zu funktionieren, wird für Palim-Palim ein eigener TURN-Server betrieben. Dieser läuft mit Linux Ubuntu und benutzt Coturn, eine Open-Source Implementierung des TURN-Protokolls \cite{noauthor_coturncoturn_2021}. Damit der TURN-Server von allen Clients erreichbar ist, verfügt er über eine öffentliche IP-Adresse. Er ist mit einer ensprechenden Authentifizierung versehen, um unerwünschten Datenverkehr zu unterbinden. Eine Anleitung zum Einrichten und Konfigurieren eines Linux-TURN-Servers inklusive Verweise auf weitere Ressourcen zu dem Thema sind im Anhang \ref{readme} dieser Arbeit zu finden.\\\\
In Palim-Palim werden die Adresse und die Zugangsdaten des TURN-Servers beim Erstellen der PeerConnection in deren Konfiguration mitgegeben (siehe Codefragment \ref{lst:peerConfig}).\\\\
\begin{lstlisting}[caption={Konfiguration der PeerConnection mit der TURN-Server Adresse},label={lst:peerConfig},language=JavaScript]
this.peerConnectionConfig = {
	'iceServers': [
		{
			'urls': 'stun:stun.l.google.com:19302'
		},
		{
			'urls': 'turn:86.119.43.130:3478',
			'credential': '*****************',
			'username': 'palimpalim'
		}
	]
};
\end{lstlisting}
			
			
\chapter{Fazit}	
Mit Palim-Palim wurde ein Videochat-Spiel für Kinder und betagte Menschen konzpiert und entwickelt. Anhand von Spieletests wurden Einflüsse des Spiels auf die Kommunikation und das Spielerlebnis erforscht. Dabei ist ein funktionaler Prototyp entstanden, der weiteren Videochat-Spielen als Inspiration oder als Vorlage für ein entsprechendes Framework dient. Die Spieletests haben Möglichkeiten offengelegt, spielerische Elemente weiter zu verbessern. Die mit der Arbeit verbundene Literaturrecherche zeigt ausserdem auf, dass Spiele mit integriertem Videochat ein hohes Potenzial haben, um Beziehungen zwischen Generationen auch über weite Distanzen zu pflegen.\\\\
Der intergenerationelle Aspekt des Spielens rückte dabei schon in der Konzeptphase des Projektes in den Mittelpunkt. Mit Palim-Palim wurde festgestellt, dass ein asymmetrisches Game Design eine Möglichkeit bietet, den unterschiedlichen Anforderungen zweier Zielgruppen gerecht zu werden.\\\\
Die Ergebnisse der Spieletests mit betagten Personen und Kindern zeigen, dass Palim-Palim als Spiel noch viel Verbesserungspotenzial hat. Bezüglich Game Design steht es noch am Anfang seines möglichen Weges. Die Videochat-Funktion wurde von den betagten Personen aber als Chance wahrgenommen, in Zeiten erhöhter sozialer Distanzen in Kontakt zu bleiben. Bei den Kindern gab es sogar erste Interaktionversuche mit dem Video. Dies lässt darauf schliessen, dass ein Videochat durchaus auch Potenzial als Gameplay-Element hat. Wegen der geringen Anzahl Tests kann dies aber nicht abschliessend belegt werden. Gleichwohl können deswegen die aufgestellten Forschungsfragen nicht vollständig beantwortet werden. Eine Fortführung des Projekts sollte aber unbedingt regelmässige Spieletests in den Entwicklungsprozess integrieren. So kann eine grössere Menge an fortlaufendem Feedback der Zielgruppen eingeholt und in der Weiterentwicklung berücksichtigt werden.\\\\
Während der Implementation von Palim-Palim wurden diverse Eigenheiten und Möglichkeiten der verwendeten Technologien exploriert. Es zeigte sich, dass trotz bestehender Frameworks die Integration eines Videochats in ein webbasiertes Spiel nicht trivial ist. WebRTC empfiehlt sich nach wie vor als etablierter Standard. Es wird empfohlen, ein Framework zu entwickeln, welches WebRTC-Videochats möglichst einfach in bestehende oder auch neu entwickelte Multiplayer-Spiele integriert. Die Serverinfrastruktur ist dabei essenziel für die Funktionsweise des Videochats über das Web. Die Signaling- und TURN-Server könnten dabei mittels des entwickelten Frameworks auch spielübergreifend genutzt werden.

Eine Reflektion der gewählten Ansätze und Lösungen sowie mögliche Weiterentwicklungen von Palim-Palim sind in den nachfolgenden Kapiteln beschrieben.

\section{Reflektion}
\todo{-	Testing : Unpassende Zielgruppen (Erfahrene Betagte, Zu alte Kinder, nicht verwandt), Umgebung nicht Remote, Schlechte Verfügbarkeit von Testpersonen (nur kurz)
-	Implementation: JavaScript hat funktioniert, aber besser ein Framework wie Phaser verwenden, um ein ganzes Spiel zu programmieren. Zudem evtl. TypeScript. 3D bewirkte eine sehr hohe zusätzliche Komplexität, die nicht nötig gewesen wäre und viel Projektzeit gefressen hat. Peer-To-Peer-Synchronisation nicht ideal für Physik – besser wie bestehende Games serverseitig arbeiten?
}
	


\section{Mögliche Weiterentwicklungen von Palim-Palim}
\label{weiterentwicklungen}
Das Videospiel Palim-Palim bietet noch viel Potenzial für Weiterentwicklungen. Mögliche Verbesserungen, Erweiterungen und Ideen sind in diesem Kapitel dokumentiert.\\\\
Das Raumkonzept aus dem Backend wird in Palim-Palim auch im Frontend verwendet. Es wurde während der Spieletests festgestellt, dass der Austausch der Raumnummer ein Problem darstellt. Die betagte Person, welche die Raumnummer wählen kann, muss diese auf einem zweiten Kommunikationskanal dem Enkelkind übermitteln. Palim-Palim sollte daher eine Benutzerauthentifizierung implementieren und den Spielenden die Möglichkeit bieten, sich mit anderen registrierten Benutzer:innen zu verbinden. Mit einer Benutzerauthentifizierung könnte zusätzlich das Problem umgangen werden, dass man aus Versehen mit einer fremden Person ein Spiel startet.\\\\
Die Spieletests haben gezeigt, dass der Erklärungsscreen durch ein interaktives Tutorial ersetzt werden sollte. Es besteht die Vermutung, dass durch diese Art der Spielerklärung den Spielenden der Start einfacher fallen würde. Spielende können so die von ihnen verlangten Aktionen direkt ausprobieren.\\\\
Bezüglich der einseitigen Menuführung hat sich gezeigt, dass diese für die Kinder nicht ideal ist. In einer Weiterentwicklung von Palim-Palim sollte die Auswahl der Spielmodi gemeinsam stattfinden. Die Text-Buttons sollten zusätzlich ein Bild erhalten. Beide Spielende könnten ihren gewünschten Modus auswählen. Sie würden visuell auch sehen, was der Mitspielende ausgewählt hat. Die Auswahl würde von der betagten Person bestätigt werden.\\\\
Betreffend Design und Usability gibt es ebenfalls noch einige Verbesserungsideen. Beispielsweise sollten die Verkaufsobjekte auf einem Regalboden platziert werden, um sie besser ins Spiel zu integrieren. Der Hintergrund sollte gemäss den Erkenntnissen in den Spieletests überarbeitet werden. Entweder würde sich ein interaktiver Hintergrund anbieten oder es müssten genauere Recherchearbeiten betrieben werden, wie ein Hintergrund als solcher und nicht als Teil des Gameplays erkannt wird. Das Freistellen der Person auf dem Video würde im Videomodus 3 das Video noch besser in die Spielszene integrieren. Dafür wurde bereits Recherchearbeit betrieben und eine Implementation von Zhu und Oved des Frameworks TensorFlow \cite{noauthor_tfjs-modelsbody-pix_nodate} als geeignete Technologie erachtet.\\\\

\todo{Abschnitt für Videomodi 4 und 5}

\todo{Mögliche Gameplay-Erweiterungen als Liste}

\todo{Offene Probleme/Bugs als Liste}

\todo{Inspirierender Schlusstext}

Die Analyse des momentanen Standes von Palim-Palim wirft einige Fragen auf: Was soll passieren, wenn zu viele nicht auf der Einkaufsliste stehenden Elemente eingekauft wurden und dadurch kein Platz mehr im Einkaufskorb ist? Momentan werden die Objekte zur virtuellen Liste weiter hinzugefügt, aber nach sechs Objekten nicht mehr visuell im Einkaufskorb angezeigt. So kann das Spiel noch erfolgreich beendet werden. Eine mögliche Lösung ist, dass das Spiel dann abzubrechen. Eine andere Idee ist es, dass diese Objekte dann wieder zurückgelegt werden können. Eine weitere Frage ist es, wie sich die Spielenden gegenseitig helfen können, beispielsweise wenn das Kind einen gewünschten Gegenstand nicht findet. Hier wäre denkbar, dass zuerst die betagte Person animiert wird, dem Kind den Gegenstand zu beschreiben und wenn das auch nicht erfolgreich ist, könnte die betagte Person ein Pop-up beim Kind auslösen, auf welchem das gewünschte Objekt ersichtlich ist.\\\\
Um die Story des Einkaufserlebnisses abzuschliessen ist eine Erweiterung des Gameplays, respektive das Hinzufügen eines zusätzlichen sekundären Gameplayloops, angedacht. Wie in Abbildung \ref{fig:PalimPalimGameplayloopBezahlen} im Anhang beschrieben, soll dieser den Zahlungsprozess in Palim-Palim integrieren. Dieser würde nach dem primären Gameplayloop folgen, das heisst nachdem alle benötigten und eventuell noch zusätzliche Objekte sich im Einkaufskorb befinden, würde der Zahlungsprozess gestartet.\\\\
Nicht nur der Zahlungsprozess ist eine mögliche Erweiterung des Spielerlebnisses. Die Frage tritt auf, wie die Motivation der Spielenden, das Spiel erneut zu spielen, gefördert werden kann. Welche Anreize können werden hierfür geschaffen? Drei Möglichkeiten dafür werden kurz beschrieben, viele weitere aber sind denkbar. Zunächst einmal ist bereits die komplette Vorarbeit für verschiedene Spielmodi bewältigt. Die offene Arbeit, um weitere Spielszenarien (zum Beispiel ein Gemüseladen oder eine Bäckerei) zu implementieren, besteht hierbei aus der Suche nach zusätzlichen Assets für die Verkaufsobjekte und der Gestaltung von unterschiedlichen Hintergründen. Diese Spielmodi bieten wiederum weitere Möglichkeiten, wie beispielsweise die Integration einer Waage für das Gemüse oder einer Tüte für die verschiedenen Brötchen. Zweitens könnten verschiedene Ziele zu einem interessanteren Spielerlebnis beitragen. Die betagte Person müsste statt konkreten Gegenständen Zutaten für einen Fruchtsalat oder Festartikel für eine Geburtstagsfeier einkaufen. Dadurch kann auch dem Wunsch der Testpersonen nach Selbstbestimmung der Einkaufsgegenstände Folge geleistet werden. Als dritte Möglichkeit wird die Implementation der Videomodi 4 und 5 beschrieben. Diese Modi ermöglichen die Interaktion mit dem Video des Gegenübers, wodurch im Speziellen das Spielerlebnis des Kindes attraktiver wird. Zwei Vorschläge für mögliche Szenarien sind das Aufsetzen von Verkleidungsgegenständen oder das Beschmieren mit Eis. Diese Ideen sind als Designentwurf im Anhang \ref{designEntwurfVideomodi4und5} zu finden.\\\\
\todo{einige punkte evtl. mit quellen untermauern («Die Kunst des Game Designs»?)}

\todo{im milanote hat es ja auch noch ganz viele ideen... ganz vergessen. entweder hier mehr schreiben oder in den anhang}

Eine weitere Möglichkeit das Spiel aufregender zu gestalten ist die Implementierung einer Physik Engine.\\\\
\todo{Dani: Vorarbeit dokumentieren. Versuche mit ammojs und headless Möglichkeiten auf Server.} \\\\

\todo{Rollenverteilung als mögliche Weiterentwicklung}
In einer späteren Version von Palim-Palim könnte die Verteilung der Rollen wählbar sein und die Einkaufsliste zusätzlich mit Bildern ergänzt werden. Dies könnte die Kommunikation fördern, da das Kind eventuell Dinge beschreiben muss, welche es noch nicht kennt.

Von der technischen Seite her gibt es ebenfalls einige Punkte, welche noch beachtet werden sollten. Wird die Seite neu geladen, kann man zwar den Raum erneut beitreten und auch die Videokommunikation funktioniert noch einwandfrei, jedoch kehrt man nicht zum gestarteten Spiel zurück. Da der andere Spielende das Spiel nicht allein abschliessen kann und auch keine Möglichkeit für die Rückkehr in die Gamelobby oder für einen Neustart des Spiels existiert, besteht ein ungewollter Zustand ohne Rückkehrmöglichkeit. Da der Zustand des Spiels verteilt auf beiden Clients liegt, kann dieser nach einem Reload nicht wiederhergestellt werden. Die effektivste Lösung ist es, das Spiel neu zu starten. Der Neustart des Spiels muss ebenfalls noch sauber implementiert werden, dafür fehlt noch das Aufräumen der Gamescene, also das Löschen der 3D-Elemente. Ein weiteres noch unbehandeltes Problem ist ein Fehler, welcher beim Verschieben von Objekten auftreten kann. Wenn beide Spielenden gleichzeitig ein Objekt verschieben, führt dies teilweise dazu, dass das Objekt dupliziert wird.\\\\

\chapter{Literaturverzeichnis}
\printbibliography[heading=none]

\appendix

\chapter{Ehrlichkeitserklärung}

Hiermit erklären wir, die vorliegende Bachelorthesis selbständig, ohne Hilfe Dritter und nur unter Benutzung der angegebenen Quellen verfasst zu haben.\\\\
\begin{center}
\begin{tabular}{p{5cm}p{5cm}}
\centering
\includegraphics[width=4cm]{signature_dani}
Daniel Obrist\\
Brugg, 20.08.2021	&
\centering
\includegraphics[width=4cm]{signature_severin}
Severin Peyer\\
Brugg, 20.08.2021
\end{tabular}
\end{center}


\chapter{Weitere Dokumente}

\section{Readme}\label{readme}

\section{Weitere Ideen für intergenerationelle Videospiele mit integriertem Videochat}\label{WeitereIdeen}
\paragraph{Bombe entschärfen} Die betagte Person sieht nur die Anleitung und nicht die Bombe, das Kind sieht nur die Bombe und kann mit dieser interagieren. Zusammen soll die Bombe dann entschärft werden. Das Spielprinzip ist auch für andere Use Cases denkbar.
\paragraph{«Brain Out»} Viele kleine Levels mit kniffligen Denksportaufgaben. Diese könnten zusätzlich mit Sensoren des Smartphones (Magnetometer, Rotationssensor...) und nur gemeinsam von beiden Mitspielenden gelöst werden.
\paragraph{Dr. Bibber mit Filter} Der eine Spielende muss virtuelle Objekte aus dem Gesicht/Video des anderen herausoperieren, indem das Objekt beispielsweise einer Linie entlang in einen Zielbereich gezogen wird. Die Person, welche den Patienten spielt, muss dabei möglichst ruhig halten.
\paragraph{«Drecksau»} Simples Kartenspiel mit der «Realität entsprechenden» Kartenhandlungen. Ein Beispiel für so ein Spiel ist «Drecksau», bei welchem eine Regenkarte das dreckige Schweinchen wieder sauber wäscht \cite{noauthor_drecksau_nodate}.
\paragraph{Escape Room} Die beiden Spielenden müssen Rätsel lösen, um aus einem Level zu entkommen. Dabei sieht jeder Spielende nur seinen "Teil" des Rätsels.
\paragraph{FaceAPI} Ein Spiel das mittels Gesichtsausdrücken gespielt wird. Verschiedene Möglichkeiten sind: Try not to laugh challenge, Gesichtsausdrücke von einem Foto nachmachen, usw. Mittels face-api.js könnte die Gesischtserkennung implementiert werden.
\paragraph{Filter-Mania} Das Gegenüber kann einem das Gesicht mit Filter "verschönern".
\paragraph{Filter-Tabu} Eine Person bekommt ein zufälliges Objekt als Gesichts-Filter. Die andere Person muss es beschreiben, ohne gewisse Wörter zu verwenden, während die "befilterte" Person erraten muss, was sie ist.
\paragraph{Gesicht zusammensetzen} Aus einem Webcamscreenshot werden automatisch die Gesichtsteile ausgeschnitten. Diese müssen nun in einem sinnvollen Grössenverhältnis wieder zusammengesetzt werden.
\paragraph{Gesichtsberührungen} Ein Spiel mit Bewegungen: Zufällig wird Nase, Mund, Wange, Stirn angezeigt/angesagt und dieser Gesichtsteil muss dann berührt werden.
\paragraph{Grössen-/Gewichtsverhältnisse} Die Spieler müssen zusammen diverse Gegenstände  gemäss dem Gewicht ordnen (Baum, Elefant, Katze, Haus, Fernseher…). Das eigene Webcam-Bild kann auch einer dieser Gegenstände sein.
\paragraph{Ich sehe was, was du nicht siehst} Eine Person kann auf einem Bild etwas markieren. Die andere Person muss dieses Objekt finden. Es kann nach Hinweisen gefragt werden.
\paragraph{Karikatur (Körperknickbilder)} Die Spielenden setzen zusammen eine Karikatur zusammen. Der Kopf ist immer sichtbar und besteht aus dem Webcambild eines Partizipanden. Jeder baut oder zeichnet dann unterschiedliche Körperteile für die Karikatur. Am Schluss wird das hoffentlich lustige kombinierte Ergebnis präsentiert.
\paragraph{Komponieren} Es werden auch wie bei Gesichtsberührungen Teile des Gesichts angefasst. Dabei entstehen unterschiedliche Töne. Als freie Komponierversion oder als Herausforderung mit Nachspielen denkbar. 
\paragraph{Kooperatives Tetris} Tetris spielen, aber zusammen im Videochat.
\paragraph{Labyrinth} Ein halb-transparentes Labyrinth auf dem Screen und beide Spieler starten auf verschiedenen Seiten. Mit dem Finger wird ein Pfad gezeichnet, und man muss sich treffen.
\paragraph{Live Puzzle} Das Webcam-Video des Gegenübers muss als Puzzle gelöst werden.
\paragraph{Montagsmaler} Montagsmaler spielen, aber zusammen im Videochat.
\paragraph{Moorhuhn} Gemeinsam müssen z.B. Ballone zerplatzt oder sonstige Gegenstände abgewehrt werden. Mit dem Videochat als Hintergrund.
\paragraph{Pokerface} Die Teilnehmer müssen zwei erfundene und eine echte Story/Fakten über sich erzählen. Das Gegenüber muss erraten, welche Geschichte/Aussage wahr ist. Via Videochat können dabei Emotionen besser interpretiert werden.
\paragraph{Schere - Stein - Papier} Mit Computer Vision und Hand-Gesten könnte das Spiel einfach den Sieger eines Schrere-Stein-Papier ermitteln. Wäre wohl eher als Mini-Game einer grösseren Kollektion geeignet.
\paragraph{Schweizer Reise} Enkelkind und Grosseltern gehen zusammen auf eine Reise durch die Schweiz. Das Webcambild wird freigestellt und als Hintergrund können Schweizer Sehenswürdigkeiten angeschaut werden.
\paragraph{Turm bauen} Zusammen wird ein Turm aus Klötzchen gebaut. Dieser muss zum Beispiel auf der Hand des Gegenübers balanciert werden.
\paragraph{Virtuelles Bilderbuch} Ein virtuelles Bilderbuch, das man zusammen Anschauen kann. Das Kind hat dabei die Möglichkeit, mit den Szenen zu interagieren. Die ältere Person kann sich Zusatz-Informationen anzeigen lassen, und diese dem Kind vorlesen. Denkbar wären zwei Avatare, die das freigestellte Gesicht der Spielenden aus dem Videochat besitzen.
\paragraph{Wimmelbild} Ein animiertes Wimmelbild, auf dem man sich selbst und den Gesprächspartner suchen muss.
\paragraph{Zaubersprüche} Mittels Computer Vision werden Handbewegungen auf dem Videostream erkannt. In Kombination mit Audio-Erkennung könnte man so ein Repertoire an Zaubersprüchen aufbauen. Hat man diese gelernt, lassen sich lustige Filter auf den Video-Stream des Gegenüber «zaubern».


\section{Gameplayloop «Bezahlen»}

\begin{figure}[!htb]
\includegraphics[width=\textwidth]{PalimPalim_Gameplayloop_Bezahlen}
\caption[Caption for LOF]{Der angedachte Gameplayloop des Videospiels Palim-Palim für den Zahlungsprozess (eigene Darstellung).}
\label{fig:PalimPalimGameplayloopBezahlen}
\end{figure}

% -- single page landscape --
%\usepackage{lscape}
%\usepackage{pdflscape}
%\usepackage{geometry}
%\newgeometry{margin=1cm} % Ränder kleiner
%\begin{landscape}
%\end{landscape}
%\restoregeometry % Wieder die alten Ränder
%\includegraphics[width=\paperwidth]

\section{Sequenzdiagramme}
\label{Sequenzdiagramme}

\todo{grösser, damit lesbar, wie?}

\begin{figure}[!htb]
\includegraphics[width=\textwidth]{Sequenzdiagramm_Objekt_vom_Vorrat_auf_den_Tresen}
\caption[Caption for LOF]{Sequenzdiagramm für die Aktion, wenn ein Objekt vom Vorrat auf den Tresen gezogen wird.  (eigene Darstellung).}
\label{fig:SequenzdiagrammVorratTresen}
\end{figure}

\begin{figure}[!htb]
\includegraphics[width=\textwidth]{Sequenzdiagramm_Objekt_in_den_Einkaufskorb_und_Ende_des_Spiels}
\caption[Caption for LOF]{Sequenzdiagramm für die Aktion, wenn ein Objket in den Einkaufskorb gelegt wird und das Spiel danach eventuell beendet ist (eigene Darstellung).}
\label{fig:SequenzdiagrammEinkaufskorbSpielEnde}
\end{figure}

\section{Designentwürfe zu Videomodi 4 und 5}
\label{designEntwurfVideomodi4und5}

\todo{sind glaubs keine lizenzfreien bilder, was sollen wir da machen?}

\begin{figure}[!htb]
\includegraphics[width=\textwidth]{Designentwurf_Videomodi45_Glace}
\caption[Caption for LOF]{Designentwurf «Glacé» für Videomodi 4 und 5. Perspektive des Kindes (links) und der betagten Person (rechts) (eigene Darstellung).}
\label{fig:DesignentwurfGlace}
\end{figure}

\begin{figure}[!htb]
\includegraphics[width=\textwidth]{Designentwurf_Videomodi45_Verkleidung}
\caption[Caption for LOF]{Designentwurf «Verkleidung» für Videomodi 4 und 5. Perspektive des Kindes (eigene Darstellung).}
\label{fig:DesignentwurfVerkleidung}
\end{figure}

\end{document}